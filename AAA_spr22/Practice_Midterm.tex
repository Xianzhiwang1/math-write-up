\documentclass{article}

%
% Homework Details
%   - Title
%   - Due date
%   - Class
%   - Section/Time
%   - Instructor
%   - Author
%

\newcommand{\hmwkTitle}{AAA \ \#Practice Midterm}
\newcommand{\hmwkDueDate}{2022}
\newcommand{\hmwkClass}{Adv Abstract Algebra}
\newcommand{\hmwkClassTime}{Spr 2022}
\newcommand{\hmwkClassInstructor}{Prof. Peter Hermann}
\newcommand{\hmwkAuthorName}{\textbf{Xianzhi} }
\newcommand{\hmwkAuthor}{\textit{Xianzhi Wang}}


\usepackage{fancyhdr}
\usepackage{extramarks}
\usepackage{amsmath}
\usepackage{amsthm}
\usepackage{amsfonts}
\usepackage{tikz}
\usepackage[plain]{algorithm}
\usepackage{algpseudocode}

\usetikzlibrary{automata,positioning}

%
% Basic Document Settings
%

\topmargin=-0.45in
\evensidemargin=0in
\oddsidemargin=0in
\textwidth=6.5in
\textheight=9.0in
\headsep=0.25in

\linespread{1.1}

\pagestyle{fancy}
\lhead{\hmwkAuthorName}
\chead{\hmwkClass\ (\hmwkClassInstructor\ \hmwkClassTime): \hmwkTitle}
\rhead{\firstxmark}
\lfoot{\lastxmark}
\cfoot{\thepage}

\renewcommand\headrulewidth{0.4pt}
\renewcommand\footrulewidth{0.4pt}

\setlength\parindent{0pt}

%
% Create Problem Sections
%

\newcommand{\enterProblemHeader}[1]{
    \nobreak\extramarks{}{Problem \arabic{#1} continued on next page\ldots}\nobreak{}
    \nobreak\extramarks{Problem \arabic{#1} (continued)}{Problem \arabic{#1} continued on next page\ldots}\nobreak{}
}

\newcommand{\exitProblemHeader}[1]{
    \nobreak\extramarks{Problem \arabic{#1} (continued)}{Problem \arabic{#1} continued on next page\ldots}\nobreak{}
    \stepcounter{#1}
    \nobreak\extramarks{Problem \arabic{#1}}{}\nobreak{}
}

\setcounter{secnumdepth}{0}
\newcounter{partCounter}
\newcounter{homeworkProblemCounter}
\setcounter{homeworkProblemCounter}{1}
\nobreak\extramarks{Problem \arabic{homeworkProblemCounter}}{}\nobreak{}

%
% Homework Problem Environment
%
% This environment takes an optional argument. When given, it will adjust the
% problem counter. This is useful for when the problems given for your
% assignment aren't sequential. See the last 3 problems of this template for an
% example.
%
\newenvironment{homeworkProblem}[1][-1]{
    \ifnum#1>0
        \setcounter{homeworkProblemCounter}{#1}
    \fi
    \section{Problem \arabic{homeworkProblemCounter}}
    \setcounter{partCounter}{1}
    \enterProblemHeader{homeworkProblemCounter}
}{
    \exitProblemHeader{homeworkProblemCounter}
}


%
% Title Page
%

\title{
    \vspace{2in}
    \textmd{\textbf{\hmwkClass:\ \hmwkTitle}}\\
    \normalsize\vspace{0.1in}\small{Due\ on\ \hmwkDueDate\ at 11:59PM}\\
    \vspace{0.1in}\large{\textit{\hmwkClassInstructor\ \hmwkClassTime}}
    \vspace{3in}
}

\author{\hmwkAuthorName}
\date{2023}

\renewcommand{\part}[1]{\textbf{\large Part \Alph{partCounter}}\stepcounter{partCounter}\\}

%
% Various Helper Commands
%

% Useful for algorithms
\newcommand{\alg}[1]{\textsc{\bfseries \footnotesize #1}}

% For derivatives
\newcommand{\deriv}[1]{\frac{\mathrm{d}}{\mathrm{d}x} (#1)}

% For partial derivatives
\newcommand{\pderiv}[2]{\frac{\partial}{\partial #1} (#2)}

% Integral dx
\newcommand{\dx}{\mathrm{d}x}

% Alias for the Solution section header
\newcommand{\solution}{\textbf{\large Solution:}}

% Probability commands: Expectation, Variance, Covariance, Bias
\newcommand{\E}{\mathrm{E}}
\newcommand{\Var}{\mathrm{Var}}
\newcommand{\Cov}{\mathrm{Cov}}
\newcommand{\Bias}{\mathrm{Bias}}

% From xianzhi.sty

% Fancy
\newcommand{\cA}{\mathcal A}
\newcommand{\cB}{\mathcal B}
\newcommand{\cC}{\mathcal C}
\newcommand{\cD}{\mathcal D}
\newcommand{\cE}{\mathcal E}
\newcommand{\cF}{\mathcal F}
\newcommand{\cG}{\mathcal G}
\newcommand{\cH}{\mathcal H}
\newcommand{\cI}{\mathcal I}
\newcommand{\cJ}{\mathcal J}
\newcommand{\cL}{\mathcal L}
\newcommand{\cM}{\mathcal M}
\newcommand{\cN}{\mathcal N}
\newcommand{\cO}{\mathcal O}
\newcommand{\cP}{\mathcal P}
\newcommand{\cR}{\mathcal R}
\newcommand{\cS}{\mathcal S}
\newcommand{\cT}{\mathcal T}
\newcommand{\cU}{\mathcal U}
\newcommand{\cW}{\mathcal W}
\newcommand{\cX}{\mathcal X}
\newcommand{\cY}{\mathcal Y}


\newcommand{\bP}{\mathbb{P}}

\newcommand{\C}{\mathbb{C}}
\newcommand{\R}{\mathbb{R}}
\newcommand{\N}{\mathbb{N}}
\newcommand{\Q}{\mathbb{Q}}
\newcommand{\Z}{\mathbb{Z}}
\newcommand{\F}{\mathbb{F}}

% Brackets
\newcommand{\bigp}[1]{\left( #1 \right)} % (x)
\newcommand{\bigb}[1]{\left[ #1 \right]} % [x]
\newcommand{\bigc}[1]{\left\{ #1 \right\}} % {x}
\newcommand{\biga}[1]{\left\langle #1 \right\rangle} % <x>

%norm

% theorem 
\newtheorem{Proposition}{proposition}
\newtheorem{Assumption}{assumption}
\newtheorem{Definition}{definition}
\newtheorem{Corollary}{corollary}
\newtheorem{Question}{question}



% \begin{document}

% \usepackage{xianzhi}

\begin{document}

\maketitle
Practice Midterm

\pagebreak

\begin{homeworkProblem}
    Let $H \leq G$ and $A \leq G$ such that 
    $\lvert G:H \rvert = 3$ and $\lvert A \rvert = 85$.
    Prove that 
    \begin{enumerate}
        \item $A \leq H$.
        \item $\cap_{g \in G} g^{-1} H g \neq 1$.
    \end{enumerate}
    \solution

    \part 

    Observe that the right cosets of $H$ partitions $G$ into equivalent classes,
    so $x \sim y \iff xy^{-1} \in H$.\\
    (Why? say $x,y \in Hb$ for some $b \in G$, then $x = h_1 b$ and $y = h_2 b$ 
    then 
    \begin{align}
        xy^{-1} &= h_1 b (h_2 b)^{-1}\\
        &= h_1 b b^{-1} h_2^{-1} \\
        &= h_1 h_2^{-1} \in H
    \end{align}

    Thus, if we restrict this partition into 
    equivalent class to $A$, it's still a partition.
    Let $u, v \in A$, since $A \cap H \leq A$,
    we have $u,v$ in some coset w.r.t. $A \cap H \iff u v^{-1} \in A \cap H$ 
    which means $u v^{-1}$ in $A$ and in $H$.\\
    since $u v^{-1}$ is automatically in $A$ because 
    $A$ is a group, 
    \begin{align}
        uv^{-1} \in A \cap H &\iff u v^{-1} \in H \\
        &\iff u, v \ \text{in the same coset w.r.t.} \ H \ \text{in} \ G
    \end{align}
    This observation implies the number of $A \cap H$ cosets in $A$
    is at most the \# of $H$ cosets in $G$, which is $3$, since
    $\lvert G: H \rvert = 3$. \\
    Since \# of $A \cap H$ cosets in $A$ need to divide the order
    of $\lvert A \rvert = 85$, this number cannot be $3$ or $2$, 
    so it must be $1$. Thus,
    $\lvert A : A \cap H \rvert  =  1$, and since 
    $A \cap H \leq A$, we conclude
    $A \cap H = A \implies A \cap H \supseteq A$
    which implies $A \leq H$. \\

    \part

    We show $a \in \cap_{g \in G} g^{-1} H g$.
    Take any $a \in A$. 
    Observe that if we let $\cR$ 
    be the set of all right cosets of $H$,
    then for $g \in G$, we define $g^{\rho} \in Sym \cR = S_3$
    by $g^{\rho}: Hx \mapsto Hxg$
    then $\rho: G \xrightarrow[]{\rho} Sym \cR = S_3$ 
    is a homomorphism because 
    $g \xrightarrow[]{} g^{\rho}$.
    and $g \in G$ is acting on the right cosets of $H$ in $G$ by right multiplication,
    and $ker \rho = \cap_{g \in G} g^{-1} H g$.\\
    If $x \in \cap_{g \in G} g^{-1} H g$, then for all cosets of $H, Hb$, we can write 
    $x = b^{-1} h_1 b$ for some $h_1 \in H$. So $Hb \rightarrow Hbx = H b b^{-1} h_1 b = H h_1 b = H b$,
    which is the identity permutation, so $x \in ker \rho$ implies 
    $ker \rho \supseteq \cap_{g \in G} g^{-1} H g$.
    If $x \in ker \rho$, then for all $Hb$, we have 
    $Hb = Hbx \implies H = H b x b^{-1}$,
    thus, $b x b^{-1} \in H \implies b x b^{-1} = h_2$ for some $h_2 \in H$.\\
    $\implies x = b^{-1} h_2 b \in \cap_{g \in G} g^{-1} H g$\\
    $\implies ker \rho \subseteq \cap_{g \in G} g^{-1} H g$\\
    Thus, $ker \rho = \cap_{g \in G} g^{-1} H g$.\\
    We show for any $a \in A, a \in ker \rho$.
    Let $a \in A \implies \lvert a \rvert \mid \lvert A \rvert = 85$,
    Also $\lvert \rho(a) \rvert$ divide $\lvert S_3 \rvert = 3! = 6$.\\
    Let $\lvert a \rvert = k$, so $a^k = 1$.\\
    then $\left( \rho (a) \right)^k = \rho (a^k) = \rho(1) = 1$,
    so $\lvert \rho (a) \rvert$ divide $k$.\\
    Thus, $\lvert \rho (a) \rvert$ divide $gcd(85,6)=1$.\\
    so $\lvert \rho (a) \rvert = 1 \implies \rho (a) =1$\\
    so $a$ is in $ker \rho$.\\
    Thus, $A \subseteq ker \rho$,
    so $ker \rho \neq 1 \implies \cap_{g \in G} g^{-1} H g \neq 1$. 

    

\end{homeworkProblem}

\pagebreak

\begin{homeworkProblem}
    Assume that $\langle c \rangle \lhd G, \ \lvert G : \langle c \rangle \rvert = 3$,
    $c$ has infinite order, and some $b \in G$
    has order $3$. 
    Prove that $G = \langle c, b \rangle$
    and $G \cong \mathbb{Z}^{+} \times \mathbb{Z}_3^{+}$.
    (For any ring $S$, the additive group of $S$
    is denoted by $S^{+}$.\\

    \solution

    $\langle c \rangle \lhd G, \ \lvert G : \langle c \rangle \rvert = 3$,
    since $\langle c \rangle$ is infinite cyclic group
    ($c$ has infinite order), any element that 
    is non identity has infinite order. Thus,
    $b \not\in \langle c \rangle$, since $\lvert b \rvert = 3.$\\
    Thus, we can form distinct cosets $\langle c \rangle, \ b \langle c \rangle, \ b^2 \langle c \rangle$.\\
    $b \langle c \rangle$ and $b^2 \langle c \rangle$ are distinct, 
    since if $b \langle c \rangle = b^2 \langle c \rangle$,
    then $b^{-1} b^2 \in \langle c \rangle$,\\
    $\implies b \in \langle c \rangle$. Contradiction.\\
    Since $\lvert G: \langle c \rangle \rvert = 3$,
    we found all $\langle c \rangle$ cosets.\\
    since $G$ is partitioned into cosets,
    we have $G = \langle c \rangle \cup b \langle c \rangle \cup b^2 \langle c \rangle$,
    and elements $g \in G$ is in one of the cosets,
    and can be written as $b^i c^j$,
    and since $\langle c \rangle \lhd G$,
    the left, right cosets of $\langle c \rangle$ in $G$
    are the same, so $G$ is indeed generated by $\langle c, b \rangle$.\\
    It's clear that $\langle c \rangle \lhd G, \langle b \rangle \cap \langle c \rangle = 1$,
    and since $G$ is generated by $\langle c, b \rangle$, 
    and $\langle c \rangle$ is normal,
    so $\langle b \rangle \cdot \langle c \rangle$ is a group,
    so $\langle b \rangle \cdot \langle c \rangle \leq G$.
    but any $g \in G$ can be written as $b^i c^j$, if it's 
    $c^j b^i$, then use normality of $\langle c \rangle$ 
    to get $b^{i'} c^{j'}$. Thus, we only need to show 
    $\langle b \rangle \lhd G$,
    then $G$ is the internal direct product of $\langle c \rangle$ and 
    $\langle b \rangle$, so it's isomorphic to the 
    external direct product $\langle c \rangle \times \langle b \rangle$,
    then since $\langle c \rangle \times \langle b \rangle \cong \mathbb{Z}^{+} \times \mathbb{Z}_3^{+}$,
    by sending $c \mapsto 1$ in $\mathbb{Z}^{+}$,
    and $b \mapsto 1$ in $\mathbb{Z}_3^{+}$.\\
    so $G \cong \langle c \rangle \times \langle b \rangle \cong \mathbb{Z}^{+} \times \mathbb{Z}_3^{+}$.\\
    Show $\langle b \rangle \lhd G$.\\







    
\end{homeworkProblem}

\pagebreak

\begin{homeworkProblem}
    Prove that $\mathbb{Q}^{+}$ (the additive group of the rational numbers) is hopfian.\\
    \textit{Hint. If $H \lneqq \mathbb{Q}^{+}$ and 
        $\lvert H \rvert > 1 \ \implies \ \exists n \in H$ 
        such that $n$ is a positive integer and $\frac{ 1 }{ n } \not\in H$.
    Then $\frac{ 1 }{ n } + H$ has finite order in $\mathbb{Q}^{+} / H$.}\\
    \solution

    We want to show take any subgroup that is not $1$, and not $\mathbb{Q}^{+}, H \leq \mathbb{Q}^+$,
    then $\mathbb{Q}^{+} / H \not\cong \mathbb{Q}^+$.\\
    since $\mathbb{Q}^{+}$ is abelian, anysubgroup is normal.\\
    Thus, we take a proper $H \leq \mathbb{Q}^+, H \neq \mathbb{Q}^+, \lvert H \rvert > 1$,
    observe that $\lvert H \rvert = \infty$, since if $h \neq e, h \in H$, 
    then all $(h+h+h+ \ldots)$ with $nh \in H$, so $H$ has 
    infinite many elements. Thus, for some sufficient large positive integer $m \in H$,
    we have $1/m \not\in H$. 
    This is possible since $H \neq \mathbb{Q}^T$. Thus, we could consider coset $1/m + H$.
    This coset has finite order in $\mathbb{Q}^+/H$, 
    since $m^2 (\frac{ 1 }{ m } + H) = m^2 \frac{ 1 }{ m } + H = m + H = H$,
    so $\lvert \frac{ 1 }{ m } + H \rvert$ divides $m^2$, so $\lvert \frac{ 1 }{ m } + H \rvert < \infty$.
    However, every non identity element $q \in Q^{-1}$ has infinite order, since 
    adding $q$ to itself many times will never get to zero. Thus, there cannot be 
    an isomorphism between $\frac{ Q^+ }{ H }$ and $\mathbb{Q}^+$, 
    since Isomorphism preserves order of the element. 
    We can also obtain a direct contradiction. say $\exists \phi \frac{ Q^+ }{ H } \mapsto Q^+$.\\
    such that $\phi(\frac{ 1 }{ m } + H) = q \in \mathbb{Q}$ and $q \neq 1$.
    Let $\lvert \frac{ 1 }{ m } + H \rvert = k < \infty$,
    then $1=\phi(1)=\phi \left( (\frac{ 1 }{ m } + H )^k \right) = \phi (\frac{ 1 }{ m }+H)^k = q^k$
    thus $q$ has finite order. Contradiction.


    
\end{homeworkProblem}

\pagebreak

\begin{homeworkProblem}
    Let a group $G$ have order $5^2 \cdot 7 \cdot 37$. Show that 
    \begin{enumerate}
        \item $G$ has a unique Sylow $37$-subgroup $\mathbb{Q}$.
        \item $Q \leq Z(G)$.
        \item The mapping $f: G \rightarrow G, \ \forall g \in G, \ g \rightarrow g^{175}$
            is a homomorphism. 
            What is the image of $f$?
    \end{enumerate}

    \solution

    $G$ has order $5^2 \cdot 7 \cdot 37$, show 
    $G$ has unique Sylow $37$-subgroup $Q$, 
    let $n_{37} = \#$ of Sylow $37$ subgroup in $G$, by Sylow theorem,
    \begin{align}
        n_{37} \equiv 1 \mod 37 \ \text{and} \\
        n_{37} \ \text{divide} \ 5^2 \cdot 7 = 175, 
    \end{align}
    possible choices: $1,5,7,25,35, 175$, but among those,
    only $1$ satisfy $1 \equiv 1 \mod 37$. so $n_{37} = 1$, 
    so $G$ has unique Sylow $37$ subgroup, we call $\mathbb{Q}$.\\
    \part

    $G$ acts on $\mathbb{Q}$ by conjugation, and since unique Sylow-$p$ subgroup 
    is also normal, $\mathbb{Q} \lhd G$, thus, $\forall g \in G$,
    $g^{-1} Q g \subseteq Q$, and 
    since 
    $h^{-1}(g^{-1} Q g) h = (gh)^{-1} Q gh$,
    this conjugation action could be associated 
    with automorphism on $Q$.
    \begin{align}
        G &\xrightarrow[]{\varphi} Aut (Q) \leq Sym (Q)\\
        g &\xrightarrow[]{\varphi} fg \ \text{where} \ Q^{fg} = g^{-1} Q g \subset Q
    \end{align}
    similar to a proof in class, $Q \leq Z(G) \iff$ this action is trivial $\iff \varphi$
    is trivial: $\varphi$ maps $G$ to the identity automorphism on $Q$.\\
    $\iff Im \varphi = \left\{ e\right\}$.\\
    Thus, we show $Q \leq Z(G)$ by showing $Im \varphi = \left\{ e \right\}$.
    By 1st Iso Thm
    \begin{align}
        G/ ker \varphi \cong Im \varphi &\implies \frac{ \lvert G \rvert }{ \lvert ker \varphi \rvert } = \lvert Im \varphi \rvert
        &\implies \lvert Im \varphi \rvert \ \text{divide} \ \lvert G \rvert = 5^2 \cdot 7 \cdot 37
    \end{align}
    On the other hand, $Im \varphi \leq Aut (Q)$.\\
    Since $\lvert Q \rvert = 37$, which is prime $\implies Q$
    is cyclic $\implies$ $Q$ is abelian.\\
    thus, except identity, all elements 
    of $Q$ has order $37$, and there are $36$
    of such elements, so $\lvert Aut(Q) \rvert = 36$.\\
    Let $Q = \langle a \rangle$.\\
    \begin{align}
        37 = \text{order} (a^i) = \frac{ \text{order}(a) = 37 }{ gcd(37,i) } \implies gcd(37,i)=1
    \end{align}
    implies there are $36$ such $i$ satisfy this.\\
    Thus, $\lvert Im \varphi \rvert$ divide $\lvert Aut (Q) \rvert = 36$.\\
    Thus, $\lvert Im \varphi \rvert$ divide both $36$ and $5^2 \cdot 7 \cdot 37$\\
    $\implies \lvert Im \varphi \rvert$ divide $gcd(36, 5^2 \cdot 7 \cdot 37) = 1$
    $\implies \lvert Im \varphi \rvert = 1 \implies Im \varphi = \left\{ e\right\}$.\\
    \part

    Since $Q$ is the unique Sylow $37$ subgroup, $Q$ is normal, 
    so get factor group $\lvert G/Q\rvert = 175$,
    since $Im f = \left\{ g^{175} \mid g \in G \right\}$.
    Let $g \in G$ 
    \begin{align}
        Qg^{175} = (Qg)^{175} = e_{G/Q} = Q
    \end{align}
    which implies $g^{175} \in Q$,
    $\implies Im f \subset Q$.
    Since $f$ is homomorphism, \textit{why?}
    $Im f$ is a subgroup of $G \implies Im f \leq Q$, 
    thus, $\lvert Im f\rvert$ divide $\lvert Q \rvert = 37 \implies \lvert Im f\rvert = 1$ or $37$.\\
    Assume $\lvert Im f\rvert = 1 \implies Im f = \left\{ e\right\} \implies \left\{ g^{175} | g \in G\right\} = \left\{ e \right\}$,
    so for all $g \in G, g^{175} = e \implies \lvert g \rvert$ divide $175$, but 
    we know $e \neq q \in Q$ has order $37$ since $Q$ is 
    prime order cyclic group, and $37 \hspace{-4pt}\not| \hspace{2pt} 175$. 
    Contradiction.\\
    $\implies \lvert Im f\rvert = 37$. Combine
    with $Im f \leq Q \implies Im f = Q$.\\



    
\end{homeworkProblem}

\pagebreak

\begin{homeworkProblem}
    Let $G$ be a group (infinite or finite). Let $A \neq 1$ be an abelian subgroup of $G$
    such that $\lvert G:A \rvert = 5$.\\
    Show that $G$ has some nontrivial normal subgroup.\\

    \solution
    
    Consider the action of $G$ on the set of right cosets of $A: \cR$
    \begin{align}
        G &\xrightarrow[]{\rho} Sym \cR\\
        g &\rightarrow g^{\rho}\\
        g^{\rho}: Ab &\rightarrow Abg
    \end{align}
    By 1st iso thm:
    \begin{align}
        G/ ker \rho &\cong Im \rho \leq S_5\\
        \lvert G / ker \rho \rvert &= \lvert Im \rho \rvert \ \text{divide} \ \lvert S_5\rvert = 5! = 120
    \end{align}
    first, observe that $ker \rho$ cannot be the whole group $G$, because if 
    $\rho (G) = 1$,
    then this contradicts $\rho$ action transitively on the five cosets 
    of $A$ in $G$. \\
    If $\lvert G \rvert > 5!$ then since $\lvert G / ker \rho \rvert$ divides $5!$,
    we conclude $ker \rho > 1$. So we are done in this case.\\

    If $\lvert G \rvert \leq 5!$, then we observe if we let 
    $\lvert G \rvert = k \leq 5!$, $k$ must have a factor of $5$,
    since $\lvert G:A \rvert = 5, \ \implies 5 \mid k$.
    Thus, there exists a Sylow $5$-group, we call $S$.\\
    If \# of $S$ is $1$, we are done, since unique $S$ is normal.\\
    If \# of $S$ is not $1$, let's consider it's possibilities.\\
    \begin{align}
        \# \ \text{of} \ S &\equiv 1 \mod 5\\
        \# \ \text{of} \ S &\mid \frac{ k }{ 5 },\\
        \frac{ k }{ 5 } &\leq \frac{ 120 }{ 5 } = 24
    \end{align}
    Since $\frac{ k }{ 5 }$ can be any number in $\left\{ 1, 2, \ldots, 24 \right\}$,\\
    \# of $S$ could be $6, 11, 17, 22$.\\

    If \# of $S$ is $6$, then we consider $Syl_3$ subgroup in $A$,
    since $\lvert A \rvert$ has a factor of $3$. 
    (because $6 \mid \frac{ k }{ 5 }$). 
    Then since $Syl_3$ subgroup are conjugates of each other, 
    if $\exists ! \ Syl_3$ subgroup, we done, if not, there are 
    $5$ of them, because there are $5$ cosets of $A$,
    but $5 \equiv 2 \mod 3$, 
    which is a contradiction, so $\exists ! \ Syl_3$ 
    subgroup which is normal. 
    Other case can be done similarly.\\
    \textit{for p instead of 3, provided $5 \not\equiv 1 \mod p,$
    i.e., $p \neq 2$.}\\
    



    
\end{homeworkProblem}

\end{document}







