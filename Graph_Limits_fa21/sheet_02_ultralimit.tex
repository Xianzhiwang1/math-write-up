\documentclass[12pt]{article}
\usepackage{amsfonts}
\usepackage{fancyhdr}
\usepackage[a4paper, top=2.5cm, bottom=2.5cm, left=2.2cm, right=2.2cm]{geometry}
\usepackage{times}
\usepackage{amsmath}
\usepackage{amssymb}
\usepackage{graphicx}

\newtheorem{theorem}{Theorem}
\newtheorem{acknowledgement}[theorem]{Acknowledgement}
\newtheorem{algorithm}[theorem]{Algorithm}
\newtheorem{axiom}{Axiom}
\newtheorem{case}[theorem]{Case}
\newtheorem{claim}[theorem]{Claim}
\newtheorem{conclusion}[theorem]{Conclusion}
\newtheorem{condition}[theorem]{Condition}
\newtheorem{conjecture}[theorem]{Conjecture}
\newtheorem{corollary}[theorem]{Corollary}
\newtheorem{criterion}[theorem]{Criterion}
\newtheorem{definition}[theorem]{Definition}
\newtheorem{example}[theorem]{Example}
\newtheorem{exercise}[theorem]{Exercise}
\newtheorem{lemma}[theorem]{Lemma}
\newtheorem{notation}[theorem]{Notation}
\newtheorem{problem}[theorem]{Problem}
\newtheorem{proposition}[theorem]{Proposition}
\newtheorem{remark}[theorem]{Remark}
\newtheorem{solution}[theorem]{Solution}
\newtheorem{summary}[theorem]{Summary}
\newenvironment{proof}[1][Proof]{\textbf{#1.} }{\ \rule{0.5em}{0.5em}}
\input{tcilatex}

\begin{document}

\title{Sheet 2: Ultralimits and the amenability of $\mathbb{Z}$ }
\author{BSM\ }
\date{}
\maketitle

The goal of this sheet is to prove the existence of a `universal density
function' on the integers. \bigskip

For a set $X$ let $P(X)$ denote the set of subsets of $X$.

\begin{theorem}
There exists a function $m:P(\mathbb{Z})\rightarrow\lbrack0,1]$ that
satisfies the following: \newline
1) $m(\mathbb{Z)}=1$;\newline
2) $m(A\cup B)=m(A)+m(B)$ for all $A,B\subseteq\mathbb{Z}$ with $A\cap
B=\not \emptyset $; \newline
3) for every integer $z\in\mathbb{Z}$ and $A\subseteq\mathbb{Z}$ we have $%
m(A+z)=m(A)$.
\end{theorem}

Such a function is called an \emph{invariant mean}. One way to produce it is
as follows.

\begin{definition}
A set system $U\subseteq P(X)$ is a \emph{filter}, if the following hold: 
\newline
1) $\emptyset\notin U$; \newline
2) If $A\in U$ and $A\subseteq B$ then $B\in U$; \newline
3) For every $A,B\in U$ we have $A\cap B\in U$.
\end{definition}

\begin{definition}
The set system $U\subseteq P(X)$ is an \emph{ultrafilter}, if it is a filter
and: \newline
4) For every $A\in P(X)$, either $A\in U$ or $X\diagdown A\in U$.
\end{definition}

\begin{exercise}
Show that $A\in U$ or $X\diagdown A\in U$ can not both hold in an
ultrafilter.
\end{exercise}

You should think of an ultrafilter, as a `perfect voting scheme' - here,
elements of the ultrafilter are `majority' and non-elements are `minority'.
A trivial example for an ultrafilter is to take some fixed $x\in X$ and let $%
A\in U$ if and only if $x\in A$. (This would be a dictatorship). One can
prove that there exists nontrivial ultrafilters, but no one has seen one.
Here is a way to get one.

\begin{lemma}
Let $X$ be an infinite set. Then the set of cofinite subsets 
\begin{equation*}
F=\left\{ A\in P(X)\mid X\diagdown A\text{ is finite}\right\} 
\end{equation*}
forms a filter.
\end{lemma}

\begin{theorem}
Let $X$ be an infinite set. Then every filter on $X$ is contained in an
ultrafilter.
\end{theorem}

Hint: Use Zorn's lemma.

\begin{exercise}
The cofinite filter can not be a subset of any trivial ultrafilter.
\end{exercise}

So, there exists a nontrivial ultrafilter. This allows us to define a really
cool notion, called the ultralimit.

\begin{definition}
Let $U\subseteq P(\mathbb{N})$ be an ultrafilter. Let $(a_{n})$ be a real
sequence and $a$ a real number. We say that 
\begin{equation*}
\lim\nolimits_{U}(a_{n})=a 
\end{equation*}
(or, that the \emph{ultralimit} of $a_{n}$ is $a$) if for all $\varepsilon>0$
the set 
\begin{equation*}
\left\{ n\in\mathbb{N}\mid\left\vert a_{n}-a\right\vert <\varepsilon\right\}
\in U\text{.}
\end{equation*}
\end{definition}

That is, if every open neighborhood of $a$ contains the majority of the
sequence, according to $U$.

In the following, we fix a nontrivial ultrafilter $U\subseteq P(\mathbb{N})$
for good.

\begin{lemma}
The ultralimit, if exists, is unique.
\end{lemma}

\begin{theorem}
Every bounded sequence has an ultralimit.
\end{theorem}

Hint: Remember why every bounded sequence has a convergent subsequence.
\bigskip

So, EVERY\ bounded sequence is convergent here! :) It turns out that the
ultralimit also behaves nicely under arithmetic operations.

\begin{theorem}
Let $(a_{n})$ and $(b_{n})$ be bounded real sequences. Then we have: \newline
1) $\lim\nolimits_{U}(a_{n}+b_{n})=\lim\nolimits_{U}(a_{n})+\lim%
\nolimits_{U}(b_{n})$; \newline
2) $\lim\nolimits_{U}(a_{n}b_{n})=\lim\nolimits_{U}(a_{n})\lim%
\nolimits_{U}(b_{n})$;\newline
3)\ if $\lim (a_{n})=a$ then $\lim_{U}(a_{n})=a$.
\end{theorem}

\begin{exercise}
What do we get when $U$ is a trivial ultrafilter? Does all the above hold?
\end{exercise}

Now, we can take the well-known density notion on $\mathbb{Z}$ and `ultra'
it.

\begin{definition}
For a subset $A\subseteq Z$ let 
\begin{equation*}
m_{n}(A)=\frac{\left\vert A\cap\lbrack-n,\ldots,n]\right\vert }{2n+1}
\end{equation*}
be the $n$-th density of $A$.
\end{definition}

Of course, not every subset has a density.

\begin{exercise}
Find a subset of $A$ such that $\lim m_{n}(A)$ does not exist.
\end{exercise}

And, here we go.

\begin{theorem}
Let $U$ be a nontrivial ultrafilter on $N$. Then 
\begin{equation*}
m(A)=\lim\nolimits_{U}(m_{n}(A)) 
\end{equation*}
is an invariant mean on $\mathbb{Z}$.
\end{theorem}

\begin{exercise}
Show that for any invariant mean $m$ on $\mathbb{Z}$, we have $m(d\mathbb{Z}%
)=1/d$.
\end{exercise}

\end{document}
