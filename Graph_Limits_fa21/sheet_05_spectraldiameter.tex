\documentclass[12pt]{article}%
\usepackage{amsfonts}
\usepackage{fancyhdr}
\usepackage[a4paper, top=2.5cm, bottom=2.5cm, left=2.2cm, right=2.2cm]{geometry}
\usepackage{times}
\usepackage{amsmath}
\usepackage{amssymb}
\usepackage{graphicx}%
\setcounter{MaxMatrixCols}{30}

\newtheorem{theorem}{Theorem}
\newtheorem{acknowledgement}[theorem]{Acknowledgement}
\newtheorem{algorithm}[theorem]{Algorithm}
\newtheorem{axiom}{Axiom}
\newtheorem{case}[theorem]{Case}
\newtheorem{claim}[theorem]{Claim}
\newtheorem{conclusion}[theorem]{Conclusion}
\newtheorem{condition}[theorem]{Condition}
\newtheorem{conjecture}[theorem]{Conjecture}
\newtheorem{corollary}[theorem]{Corollary}
\newtheorem{criterion}[theorem]{Criterion}
\newtheorem{definition}[theorem]{Definition}
\newtheorem{example}[theorem]{Example}
\newtheorem{exercise}[theorem]{Exercise}
\newtheorem{lemma}[theorem]{Lemma}
\newtheorem{notation}[theorem]{Notation}
\newtheorem{problem}[theorem]{Problem}
\newtheorem{proposition}[theorem]{Proposition}
\newtheorem{remark}[theorem]{Remark}
\newtheorem{solution}[theorem]{Solution}
\newtheorem{summary}[theorem]{Summary}
\newenvironment{proof}[1][Proof]{\textbf{#1.} }{\ \rule{0.5em}{0.5em}}


\begin{document}

\title{Sheet 5: A spectral diameter bound }
\author{Xianzhi Wang}
\date{fall 2021}
\maketitle






Let $G$ be a finite undirected $d$-regular graph on $n$ vertices. Let $A$
denote the adjacency operator, and let $b_{0},b_{1},\ldots,b_{n-1}$ be an
orthonormal eigenbasis with real eigenvalues $\lambda_{0}\geq\lambda_{1}\geq%
\ldots\geq\lambda_{n-1}$. Let $\rho=\rho_{0}(G)$ be the spectral radius. On
this sheet, we give an upper bound on the diameter using $\rho$.

Of course, $b_{i}^{\top}b_{i}=1$ ($0\leq i<n$). Let us force product the
other way round. Let 
\begin{align}
U_{i}=b_{i}b_{i}^{\top}\text{ \ (}0\leq i<n\text{).}
\end{align}


These matrices may look boring at first, but they are not. It turns out that
they give an `orthonormal eigendecomposition' for $A$.

\begin{lemma}
Let $A,B$ be $n\times n$ matrices such that $Av_{i}=Bv_{i}$ ($0\leq i<n$)
for a basis $(v_{i})$ for $\mathbb{R}^{n}$. Then $A=B$.
\end{lemma}

\begin{proof}
    We show $A=B$ by showing $Ax = Bx$ for all $x \in \mathbb{R}^n$.
    For all vector $x$, decompose $x = \sum_{i=0}^{n-1} \alpha_i v_i$,
    \begin{align}
        Ax &= A(\sum_i \alpha_i v_i)\\
        &= \sum_i \alpha_i A(v_i)\\
        &= \sum_i \alpha_i B v_i\\
        &=  B (\sum_i \alpha_i v_i)\\
        &= B(x)
    \end{align}
    for all $x \in \mathbb{R}^n$.
    
\end{proof}



\begin{theorem}
The following hold for $1\leq i,j<n$: \newline
1) $U_{i}^{2}=U_{i}$; \newline
2) $U_{i}U_{j}=0$ ($i\neq j$);\newline
3) $AU_{i}=\lambda _{i}U_{i}$; \newline
4) $\dsum\limits_{k=1}^{n}U_{k}=I$;\newline
5) $\dsum\limits_{k=1}^{n}\lambda _{k}U_{k}=A$;
\end{theorem}

\begin{proof}
1) 
\begin{align}
    U_i^2 = (b_i b_i^T)^2 = b_i b_i^T b_i b_i^T = b_i b_i^T = U_i
\end{align}
2)
\begin{align}
    b_i b_i^T b_j b_j^T = b_i \cdot 0 \cdot b_j^T = 0 \ (i \neq j)
\end{align}
3)
\begin{align}
    A U_i = A b_i b_i^T = \lambda_i b_i b_i^T = \lambda_i U_i
\end{align}
since $b_0, b_1, \ldots, b_{n-1}$ is eigenbasis.\\
4)
\begin{align}
    \sum_{k=1}^n U_k &= U_1 + U_2 + \ldots + U_n\\
    &= b_1 b_1^T + b_2 b^2^T + \ldots + b_n b_n^T\\
    &=
    \begin{bmatrix}
        b_1 & b_2 & \ldots & b_n\\
    \end{bmatrix} \cdot
    \begin{bmatrix}
        b_1^T\\
        b_2^T\\
        \vdots\\
        b_n^T
    \end{bmatrix} = I
\end{align}
since it's the product of a matrix with orthonormal columns with its transpose.\\
5)
\begin{align}
    \sum_{k=1}^n \lambda_k U_k &= A\\
    \sum_{k=1}^n \lambda_k U_k &= \sum_k \lambda_k b_k b_k^T\\
    &= \sum_k A b_k b_k^T\\
    &= A \sum_k b_k b_k^T\\
    &= A \sum_k U_k
    = A I
    = A
\end{align}


\end{proof}


\begin{definition}
For $x,y\in V(G)$ let the \emph{distance} $d(x,y)$ be the minimal length of
a path going from $x$ to $y$. 
\end{definition}

\begin{lemma}
Let $G$ be an undirected, connected graph. Then $d$ is a metric on $V(G)$. 
\end{lemma}

\begin{proof}
    We show $d$ is a metric on $V(G)$. 
    where $G$ undirected connected graph.\\
    (1) $d (x,y) \geq 0$, since the length of a path is always positive or zero,
    when $x$ and $y$ are the same vertex. 
    $x=y \implies d(x,y)=0$ since the shortest path is of zero length
    when we are going from a vertex to itself.
    $d(x,y)=0 \implies x=y$, the shortest path is zero length, so we 
    never moved, it's the same vertex.\\
    (2) $d(x,y) = d(y,x)$ since traversing the shortest path backwards,
    the length is the same. (The graph is undirected)\\
    (3) $d(x,z) \leq d(x,y) + d(y,z) \ \forall \ x,y,z \in V(G)$. the 
    shortest path from $x$ to $y$, then the shortest path from 
    $y$ to $z$ could only be a longer path than the shortest path
    from $x$ to $z$.\\
    Since graph $G$ is connected, $d(x,y) < \infty$.
\end{proof}


\begin{definition}
Let the \emph{diameter} of $G$ be 
\begin{equation*}
\mathrm{diam}(G)=\max \left\{ d(x,y)\mid x,y\in V(G)\right\} .
\end{equation*}
\end{definition}

One can give a general lower bound on the diameter as follows.

\begin{lemma}
Let $G$ be a $d$-regular undirected graph on $n$ vertices with $d\geq 3$.
Then 
\begin{align}
\mathrm{diam}(G)\geq \log _{d-1}(n/3).
\end{align}
\end{lemma}

\begin{proof}
    Let $G$ be $d$-regular, undirected graph on $n$ vertices $(d \geq 3)$
    \begin{align}
        diam (G) &\geq \log_{d-1} \left( \frac{ n }{ 3 }\right) \\
    \end{align}
    Let $r$ denote $diam(G)$, then showing 
    \begin{align}
        r &\geq \log_{d-1} \left( \frac{ n }{ 3 }\right)
    \end{align}
    is the same as showing $(d-1)^r \geq \frac{ n }{ 3 }$,
    which is 
    \begin{align}
        3(d-1)^r &\geq n.\\
        n &\leq 3 (d-1)^r
    \end{align}
    we could have an upper bound on $n$ by fixing 
    a vertex $v_0$ first, 
    then since $G$ is $d$ regular, $v_0$ has $d$ neighbors,
    at $1$st level, as we expand outwards.
    At 2nd level, each of those neighbors of $v_0$ has
    $d-1$ new neighbors, so we have $d(d-1)$ 
    in total (at max). At 3rd level, $d(d-1)^2$ vertices 
    in total (at max). As we go on, we could iterate at maximum
    $r-1$ levels, since the diameter of $G$ is $r$. 
    If our fixed vertex $v_0$ happens to be at the ``center''
    of the graph, then $n$ might be small compared to the 
    upper bound we obtained in this way.
    \begin{align}
        n &\leq 1 + d + d(d-1) + d(d-1)^2 + \ldots + d(d-1)^{r-1}\\
        &= 1 + \frac{ d(d-1)^r - d }{ d-2 }\\
        &\leq 3(d-1)^r\\
    \end{align}
    if we show the last $\leq$, then we are done.\\
    Now, we have $3 \leq d$.\\
    Use $X := (d-1)^r$ as short hand.\\
    \begin{align}
        3X-1 &< 3X < dX\\
        6X - 2 &< 2dX\\
        dX-2 &< 3X(d-2)\\
        \frac{ dX-2 }{ d-2 } &< 3X \\
        \frac{ d-2 + dX -d }{ d-2 } &< 3X\\
        1 + \frac{ dX }{ d-2 } - \frac{ d }{ d-2 } &< 3X \\
        1 + \frac{ d(d-1)^r }{ d-2 } - \frac{ d }{ d-2 } &< 3 (d-1)^r
    \end{align}
    as desired. and we are done.
    
    
\end{proof}


So, for a fixed degree, the diameter is at least logarithmic in the number
of vertices.

It turns out that for good expanders, one can get a logarithmic upper bound
for the diameter. 

\begin{lemma}
Let $x,y\in V(G)$. Then 
\begin{align}
d(x,y)=\min \left\{ k\mid X_{\{x\}}^{\top }(A^{k})X_{\{y\}}\neq 0\right\} .
\end{align}

\end{lemma}

\begin{theorem}
We have 
\begin{align}
\mathrm{diam}(G)\leq \frac{\log (n)}{\log (d/\rho _{0}(G))}\text{.}
\end{align}

\end{theorem}

Hint: decompose $X_{\{x\}}$ and $X_{\{y\}}$.\\
\textbf{solution:}\\

  \begin{align}
      diam (G) &= \max_{x,y \in V(G)} \ \ \min_{\chi^T_{\left\{ x \right\}} A^k \chi_{\left\{ y \right\}} > 0 } k
  \end{align}
  (since $A^k$ has non-negative entries, so $\neq 0$ 
  is the same as $>0$).

   \begin{align}
        \chi^T_{\left\{ x \right\}} A^k \chi_{\left\{ y \right\}} &= 
       \chi^T_{\left\{ x \right\}} \sum_{i=0}^{n-1} \lambda_i^k U_i \chi_{\left\{ y \right\}} \\
       &= \sum_{i=0}^{n-1} \lambda_i^k (U_i)_{x,y} \\
       &= \lambda_0^k (U_0)_{x,y} + \sum_{i=1}^{n-1} \lambda_i^k (U_i)_{x,y} \\
   \end{align}
    since $\lambda_0 = d$, and 
    \begin{align}
       U_0 = b_0 b_0^T =
     \begin{bmatrix}
         \frac{ 1 }{ n } & \ldots & \ldots & \frac{ 1 }{ n }\\
          \vdots & \ddots & & \vdots \\
          \vdots & & \ddots & \vdots \\
          \frac{ 1 }{ n } & \ldots & \ldots & \frac{ 1 }{ n }
      \end{bmatrix}
  \end{align}
  and 
    \begin{align}
        (U_i)_{x,y} &= (b_i b_i^T)_{x,y}\\
        &= (b_i)_x \cdot (b_i)_y\\
        &= x_i \cdot y_i
    \end{align}
    where 
    \begin{align}
        x_i &:= (b_i)_x\\
        y_i &:= (b_i)_y\\
        \text{for} \ i &\in \left\{ 0,1,2, \ldots, n-1 \right\}
    \end{align}
    is the $x$-th entry in vector $b_i$, the 
    $y$-th entry in vector $b_i$.
   \begin{align}
\chi^T_{\left\{ x \right\}} A^k \chi_{\left\{ y \right\}} &= 
\lambda_0^k (U_0)_{x,y} + \sum_{i=1}^{n-1} \lambda_i^k (U_i)_{x,y} \\
&= \frac{ d^k }{ n } + \sum_{i=1}^{n-1} \lambda_i^k x_i \cdot y_i\\
&\geq \lvert \frac{ d^k }{ n }\rvert - 
\lvert \sum_{i=1}^{n-1} \lambda_i^k x_i \cdot y_i \rvert >\\
&\geq \frac{ d^k }{ n } - \rho^k \sum_{i=1}^{n-1} \lvert x_i \rvert \lvert y_i \rvert\\
&\geq \frac{ d^k }{ n } -
\rho^k \sqrt[]{\sum_{i=1}^{n-1} \lvert x_i\rvert^2 } \sqrt[]{\sum_{i=1}^{n-1} \lvert y_i \rvert^2}\\
&\geq \frac{ d^k }{ n } -
\rho^k \sqrt[]{\sum_{i=0}^{n-1} \lvert x_i\rvert^2 } \sqrt[]{\sum_{i=0}^{n-1} \lvert y_i \rvert^2}\\
   \end{align}
   where we used inverse triangle inequality, and then Cauchy-Shwarz inequality. \\
   since 
   \begin{align}
       \sum_{i=0}^{n-1} \lvert x_i \rvert^2 &=
       \sum_{i=0}^{n-1} (b_i)_x (b_i)_x =  
       \sum_{i=0}^{n-1} \left( b_i b_i^T \right)_{x,x}
       = (I)_{x,x} = 1 \\
       &= \frac{ d^k }{ n } - \rho^k \cdot 1 \cdot 1\\
       &= \frac{ d^k }{ n } - \rho^k
   \end{align}
   Thus,
   \begin{align}
       \chi_{\left\{ x \right\}}^T A^k \chi_{\left\{ y \right\}}
       \geq \frac{ d^k }{ n } - \rho^k \ (*)
   \end{align}
   Since max is taken over $x,y \in V(G)$, we want 
   $\frac{ d^k }{ n } - \rho^k > 0$. So
   \begin{align}
       \frac{ d^k }{ n } &> \rho^k \\
       \frac{ d^k }{ \rho^k } &> n \\
       \log \left( \frac{ d }{ \rho }\right)^k &> \log n\\
       k &> \frac{ \log n }{ \log \left( d/ \rho \right) }
   \end{align}
   Min of $k$ is taken over all
   $\chi_{\left\{ x\right\}}^T A^k \chi_{\left\{ y\right\}} > 0$,
   so $k$ has $\frac{ \log n  }{ \log \left(d/\rho \right) }$ growth.
    
    \textbf{end solution}

When $G$ is bipartite, the above theorem (and also the theorem on the
previous sheet) do not give anything reasonable, as $\rho_{0}(G)=d$.
However, there is a similar diameter bound for bipartite graphs where all
the other eigenvalues (in absolute value) are bounded away from $d$.

\begin{theorem}
Let $G$ be a finite, connected, undirected $d$-regular bipartite graph on $n$ vertices. 
Assume that $\left\vert \lambda_{i}\right\vert \leq r$ ($1\leq i<n-1$). 
Then $G$ has logarithmic diameter.
\end{theorem}

\end{document}



