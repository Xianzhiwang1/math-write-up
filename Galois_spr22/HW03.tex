\documentclass[12pt,english]{article}
\usepackage{xianzhi}

\title{Galois Theory HW03}
\author{xianzhi wang}
\date{Due Mar 04, 2022}

\begin{document}

\maketitle

\section*{Exercise 3.2.12}
\begin{question}
Show $\pi - \sqrt{\pi}$ is not algebraic over $\Q$.
\end{question}

First we show $\sqrt{\pi}$ is not algebraic over $\Q$. Assume for contradiction that $\sqrt{\pi}$ is algebraic over $\Q$, so $\exists$ $a_n,\cdot, a_0\in \Q$ not all zero such that
\begin{align*}
    a_n(\sqrt{\pi})^n +\cdots+a_1 \sqrt{\pi} +a_0 = 0 \quad \text{denote by } p(\sqrt{\pi}) = 0
\end{align*}
Consider 
\begin{align*}
    0 = p(\sqrt{\pi})p(-\sqrt{\pi}) = q(\pi) \text{ for some polynomial } q
\end{align*} some algebra omitted.
Thus, $q(\pi) = 0$, so $\pi$ is algebraic over $\Q$, contradiction. \\

Now assume for contradiction $\pi-\sqrt{\pi}$ is algebraic over $\Q$, so $\exists$ $b_n \cdots, b_0 \in \Q$ not all zero s.t.
\begin{align*}
    b_n(\pi-\sqrt{\pi})^n+\cdots +b_1(\pi-\sqrt{\pi})+b_0 = 0
\end{align*} multiply out this expression, we obtain a not all zero polynomial of $\sqrt{\pi}$ such that it equals $0$. A contradiction. 


\section*{Exercise 5.3.7}
\begin{question}
    Let $L$ be the splitting field of a polynomial $f \in K[X].$ Show that $[L:K] \leq n!$, where $n$ is the degree of $f$.
\end{question}


\section*{Exercise 5.3.8}

$i$ is not contained in the splitting field of $x^3-2$ over $\Q$.\\

\begin{proof}
We know the splitting field of $X^3-2$ is $\Q(\sqrt[3]{2}, \omega)$, 
first we show $i \not\in \Q(\omega)$. 
Assume for contradiction $i \in \Q(w) \implies \sqrt{3} \in \Q(\omega)$, which implies 
\begin{align*}
    \Q(i,\sqrt{3}) \subset \Q(\omega),
\end{align*} but the LHS has degree 4 over $\Q$, and RHS has degree 2 over $\Q$. \\

$i \not\in \Q(\omega)$, so $\deg_{\Q(\omega)}i >1$. 
Thus, $\Q(\omega,i)$ is degree $4$ extension over $\Q$. 
However, if assume for contradiction $i \in \Q(\omega,\sqrt[3]{2}$, 
then $\Q(\omega,i) \subset \Q(\omega, \sqrt[3]{2}),$ 
but the former has degree 4, and the latter has degree 6, and $4 \nmid 6$.
\end{proof}







\end{document}
