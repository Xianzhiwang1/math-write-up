\documentclass[12pt,english]{article}
\usepackage{xianzhi}

\title{Galois Theory HW04}
\author{Xianzhi}
\date{Due March 11, 2022}

\begin{document}

\maketitle

\section*{Exercise 5.3.9}
\begin{question}
    Is the polynomial $X^4 -2$ irreducible over the field $\Q(\sqrt{3})$?
\end{question}

\textbf{Soln}

Assume $X^4 - 2$ is reducible over $\Q(\sqrt{3})$.
Then $X^4 -2$ either factor into 1 degree one factor and 1 degree three factor, or 
factor into 2 degree two factor (factor means polynomial).

\textbf{Case 1}

$X^4 -2$ has a degree one factor in $\Q(\sqrt{3})$. 
so it has a root in $\Q(\sqrt{3})$. The roots of $X^4 - 2$ are
\begin{align}
    \sqrt[4]{2}, \ i \sqrt[4]{2}, \ - \sqrt[4]{2}, \ - i \sqrt[4]{2}. 
\end{align}
since $\Q (\sqrt{3})$ is a degree $2$ extension, 
(Because $\sqrt{3}$ has minimal polynomial $X^2 -3$, which is irreducible by Eisenstein,)
\begin{align}
    \mathbb{Q}(\sqrt[2]{3}) = \left\{ a + b \sqrt[2]{3} \ | \ a,b \in \mathbb{Q} \right\} \subseteq \mathbb{R},
\end{align}
since $i \sqrt[4]{2}, - i \sqrt[4]{2} \not\in \mathbb{R}$,
we conclude that they are not in $\mathbb{Q}(\sqrt[2]{3})$.\\
Now, observe $\sqrt[4]{2}$ and $- \sqrt[4]{2}$ 
has $X^4 - 2$ as their minimal polynomial over $\mathbb{Q}$,
and $X^4 -2$ is irreducible over $\mathbb{Q}$ by Eisenstein,
so if $\sqrt[4]{2}$ were to be in $\mathbb{Q}(\sqrt[2]{3})$,
then necessarily $\mathbb{Q}(\sqrt[4]{2}) \subseteq \mathbb{Q}(\sqrt[2]{3}),$
but $\mathbb{Q}(\sqrt[4]{2})$ is degree 4 extension,
and $\mathbb{Q}(\sqrt[2]{3})$ is degree 2 extension,
by the tower law, we have a contradiction.
The argument for $- \sqrt[4]{2}$ is the same. End of Case 1.\\
\textbf{Case 2}\\
$X^4 - 2$ factor into 2 degree two polynomial over $\mathbb{Q}(\sqrt[2]{3})$, thus,
\begin{align}
    (X^2 - \sqrt{2})(X^2 + \sqrt{2})
\end{align} 
and $\sqrt[2]{2} \in \mathbb{Q}(\sqrt[2]{3}).$\\
\textit{Comment from instructor: There are other ways to factor
$X^4 - 2$ as the product of two quadratic polynomials.}\\
Since $\mathbb{Q}(\sqrt{3})$ is degree 2 extension, we could write
$\sqrt{2} = a + b \sqrt{3}$ for $a,b \in \mathbb{Q}$.\\
So $2 = a^2 + 3 b^2 + 2ab \sqrt{3}$. Thus,
\begin{align}
    2ab &= 0\\
    2 &= a^2 + 3b^2
\end{align}
If $a=0$, then $2/3 = b^2$, which implies $b=\sqrt{\frac{ 2 }{ 3 }}$, 
which is a contradiction with $b \in \mathbb{Q}$.\\
If $b=0$, then $2=a^2$ which implies $a = \sqrt{2}$, 
which is a contradiction with $a \in \mathbb{Q}$.\\
If $a,b$ both zero, then $2=0$, which is a contradiction.\\
Thus $\sqrt{2} \not\in \mathbb{Q}(\sqrt{3})$. 
So $X^4 - 2$ cannot factor into 2 degree two polynomial over $\mathbb{Q}(\sqrt{3})$.
End of Case 2.

\section*{Exercise 6.4.6}
\begin{question}
    Let $L$ be the splitting field over $\Q$ of $X^5 - 2$ over $\Q$. 
    Show that the Galois group $G:=\Gamma(L:\Q)$ has order $20$,
    and $G$ has a normal subgroup $N$ with $\lvert N \rvert = 5$
    such that the factor group $G/N$ is cyclic.
\end{question}

Let $L$ be the splitting field of $X^5 - 2$ over $\mathbb{Q}$.
\begin{align}
    L = \mathbb{Q}(\sqrt[5]{2}, \omega) \ \text{where} \ \omega = e^{\frac{ 2 \pi i }{5}}
\end{align}
we have
\begin{align}
    L \subseteq \mathbb{Q}(\sqrt[5]{2}, \sqrt[5]{2}\omega, \sqrt[5]{2}\omega^2, \sqrt[5]{2}\omega^3, \sqrt[5]{2} \omega^4)
\end{align}
because $\sqrt[5]{2}$ is in there, and $\omega = \frac{ \sqrt[5]{2} \omega }{ \sqrt[5]{2} }$ is in there.
\begin{align}
    L \supseteq \mathbb{Q}(\sqrt[5]{2}, \sqrt[5]{2}\omega, \sqrt[5]{2}\omega^2, \sqrt[5]{2}\omega^3, \sqrt[5]{2} \omega^4)
\end{align}
because we can multiply $\sqrt[5]{2}$ and $\omega$ to generate the roots.\\
Claim: $[L:\mathbb{Q}]=20$.\\
First $[\mathbb{Q}(\sqrt[5]{2}:\mathbb{Q}]=5$ since $X^5 - 2$ has $\sqrt[5]{2}$ as a root,
and $X^5 - 2$ is irreducible by Eisenstein (let $p=2$), so 
$X^5 - 2$ is the minimal polynomial of $\sqrt[5]{2}$ over $\mathbb{Q}$,
and its degree 5.\\
$[\mathbb{Q}(\omega):\mathbb{Q}]=4$ since $\omega$ is a root of 
$X^4 + X^3 + X^2 + X +1$,
so the degree of $\mathbb{Q}(\omega)$ is at most 4, since $X^5 - 1$ has 4
primitive roots of unity, the degree $[\mathbb{Q}(\omega):\mathbb{Q}]$ is 4.\\
Thus, since $\mathbb{Q}(\omega)$ and $\mathbb{Q}(\sqrt[5]{2})$
are intermediate fields in $L$,
their degree devides degree of $L$ over $\mathbb{Q}$.
so $[L:\mathbb{Q}]$ is a multiple of 20. but by previous result
\begin{align}
    [K(\alpha_1,\ldots, \alpha_n):K] &\leq deg_K(\alpha_1) \cdots deg_K (\alpha_n)\\
    [L:\mathbb{Q}] &\leq 5 \cdot 4 = 20\\
    \implies [L:\mathbb{Q}] &=20
\end{align}





\section*{Exercise 6.4.7}
\begin{question}
    Let $p$ be an irreducible polynomial over a subfield $K$ of $\C$, and denote by $L$ the splitting field of $p$ over $K$. 
    Show that if the Galois group $\Gamma (L:K)$ is abelian (i.e. commutative), 
    then its order equals the degree of $p$.
\end{question}

\begin{proof}
    Let $p$ be irreducible polynomial over $K \subseteq \C$.
    Let $L$ be the splitting field of $p$ over $K$.
    Let $\alpha$ be a root of $p$.
    Let $m = m_K^{\alpha}$ be the minimal polynomial having $\alpha$ as a root over $K$.
    Then $m$ divide $p$. But $p$ is already irreducible, so we conclude that $m = p$. 
    (We can assume $p$ is monic, because if not, we could scale by a constant
    from $K$ to make it monic.)
    Since $L$ is the splitting field of $p$ over $K$, and $K \subseteq L \subseteq \C$, 
    so $p$ has no multiple roots in $L$, we apply the equivalence theorem
    to say $L$ of $K$ is a Galois extension. Since $\Gamma(L:K)$ is abelian,
    all subgroups are normal. We apply Galois correspondence.
    \begin{align}
        \Gamma(K(\alpha):K) \cong \Gamma(L:K)/\Gamma(L:K(\alpha))
    \end{align}
    and $K(\alpha):K$ is Galois extension by Galois correspondence.
    so $K(\alpha):K$ is normal and separable.
    Thus, since we established $m_K^{\alpha} = p$, $K(\alpha)$ is normal,
    so $K(\alpha)$ contain all the roots of $m_K^{\alpha}=p$, 
    so $K(\alpha) \supset L,$ and since $K(\alpha) \subseteq L$,
    we conclude $K(\alpha) = L$. Thus,
    \begin{align}
        \lvert \Gamma(L:K) \rvert = [L:K] = [K(\alpha):K] = deg \ m_K^{\alpha} = deg \ p
    \end{align}
    and the first equal sign is because extension is Galois.
    
\end{proof}





\section*{A question from HW02}
\begin{question}
Show number of automorphisms of a finite degree field extension divides the degree of the field extension. 
\end{question}
Let $K \subset L, L:K$ be a finite degree field extension. Recall
\begin{align*}
    \Gamma(L:K) &= \{g \in \Gamma(L): g(x) = x \quad \forall x \in K\}\\
    \text{WTS:} \quad |\Gamma(L:K)| &\mid [L:K].
\end{align*} Recall Artin's theorem, let $\Gamma(L:K)$ be the finite subgroup. (Since $|\Gamma(L:K)|$ is bounded by $[L:K]<\infty$.) and 
\begin{align*}
    M = \{x \in L: \forall g \in \Gamma(L:K): g(x)= x\}
\end{align*} so $K \subset M$, and $[L:M]=|\Gamma(L:K)|$. Thus, consider $K \subset M \subset L$,
\begin{align*}
    [L:K] = [L:M][M:K]
\end{align*} where $[L:M] = |\Gamma(L:K)|$, so $|\Gamma(L:K)|$ divides $[L:K]$.










\end{document}
