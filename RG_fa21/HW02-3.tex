\documentclass{article}

%
% Homework Details
%   - Title
%   - Due date
%   - Class
%   - Section/Time
%   - Instructor
%   - Author
%

\newcommand{\hmwkTitle}{RG \ \#02-3}
\newcommand{\hmwkDueDate}{Sept 2021}
\newcommand{\hmwkClass}{RG}
\newcommand{\hmwkClassTime}{fa21}
\newcommand{\hmwkClassInstructor}{Professor Csik\'os Bal\'azs}
\newcommand{\hmwkAuthorName}{\textbf{Xianzhi} }
\newcommand{\hmwkAuthor}{\textit{Xianzhi Wang}}


\usepackage{fancyhdr}
\usepackage{extramarks}
\usepackage{amsmath}
\usepackage{amsthm}
\usepackage{amsfonts}
\usepackage{tikz}
\usepackage[plain]{algorithm}
\usepackage{algpseudocode}

\usetikzlibrary{automata,positioning}

%
% Basic Document Settings
%

\topmargin=-0.45in
\evensidemargin=0in
\oddsidemargin=0in
\textwidth=6.5in
\textheight=9.0in
\headsep=0.25in

\linespread{1.1}

\pagestyle{fancy}
\lhead{\hmwkAuthorName}
\chead{\hmwkClass\ (\hmwkClassInstructor\ \hmwkClassTime): \hmwkTitle}
\rhead{\firstxmark}
\lfoot{\lastxmark}
\cfoot{\thepage}

\renewcommand\headrulewidth{0.4pt}
\renewcommand\footrulewidth{0.4pt}

\setlength\parindent{0pt}

%
% Create Problem Sections
%

\newcommand{\enterProblemHeader}[1]{
    \nobreak\extramarks{}{Problem \arabic{#1} continued on next page\ldots}\nobreak{}
    \nobreak\extramarks{Problem \arabic{#1} (continued)}{Problem \arabic{#1} continued on next page\ldots}\nobreak{}
}

\newcommand{\exitProblemHeader}[1]{
    \nobreak\extramarks{Problem \arabic{#1} (continued)}{Problem \arabic{#1} continued on next page\ldots}\nobreak{}
    \stepcounter{#1}
    \nobreak\extramarks{Problem \arabic{#1}}{}\nobreak{}
}

\setcounter{secnumdepth}{0}
\newcounter{partCounter}
\newcounter{homeworkProblemCounter}
\setcounter{homeworkProblemCounter}{1}
\nobreak\extramarks{Problem \arabic{homeworkProblemCounter}}{}\nobreak{}

%
% Homework Problem Environment
%
% This environment takes an optional argument. When given, it will adjust the
% problem counter. This is useful for when the problems given for your
% assignment aren't sequential. See the last 3 problems of this template for an
% example.
%
\newenvironment{homeworkProblem}[1][-1]{
    \ifnum#1>0
        \setcounter{homeworkProblemCounter}{#1}
    \fi
    \section{Problem \arabic{homeworkProblemCounter}}
    \setcounter{partCounter}{1}
    \enterProblemHeader{homeworkProblemCounter}
}{
    \exitProblemHeader{homeworkProblemCounter}
}


%
% Title Page
%

\title{
    \vspace{2in}
    \textmd{\textbf{\hmwkClass:\ \hmwkTitle}}\\
    \normalsize\vspace{0.1in}\small{Due\ on\ \hmwkDueDate\ at 11:59PM}\\
    \vspace{0.1in}\large{\textit{\hmwkClassInstructor\ \hmwkClassTime}}
    \vspace{3in}
}

\author{\hmwkAuthorName}
\date{2023}

\renewcommand{\part}[1]{\textbf{\large Part \Alph{partCounter}}\stepcounter{partCounter}\\}

%
% Various Helper Commands
%

% Useful for algorithms
\newcommand{\alg}[1]{\textsc{\bfseries \footnotesize #1}}

% For derivatives
\newcommand{\deriv}[1]{\frac{\mathrm{d}}{\mathrm{d}x} (#1)}

% For partial derivatives
\newcommand{\pderiv}[2]{\frac{\partial}{\partial #1} (#2)}

% Integral dx
\newcommand{\dx}{\mathrm{d}x}

% Alias for the Solution section header
\newcommand{\solution}{\textbf{\large Solution:}}

% Probability commands: Expectation, Variance, Covariance, Bias
\newcommand{\E}{\mathrm{E}}
\newcommand{\Var}{\mathrm{Var}}
\newcommand{\Cov}{\mathrm{Cov}}
\newcommand{\Bias}{\mathrm{Bias}}

% From xianzhi.sty

% Fancy
\newcommand{\cA}{\mathcal A}
\newcommand{\cB}{\mathcal B}
\newcommand{\cC}{\mathcal C}
\newcommand{\cD}{\mathcal D}
\newcommand{\cE}{\mathcal E}
\newcommand{\cF}{\mathcal F}
\newcommand{\cG}{\mathcal G}
\newcommand{\cH}{\mathcal H}
\newcommand{\cI}{\mathcal I}
\newcommand{\cJ}{\mathcal J}
\newcommand{\cL}{\mathcal L}
\newcommand{\cM}{\mathcal M}
\newcommand{\cN}{\mathcal N}
\newcommand{\cO}{\mathcal O}
\newcommand{\cP}{\mathcal P}
\newcommand{\cR}{\mathcal R}
\newcommand{\cS}{\mathcal S}
\newcommand{\cT}{\mathcal T}
\newcommand{\cU}{\mathcal U}
\newcommand{\cW}{\mathcal W}
\newcommand{\cX}{\mathcal X}
\newcommand{\cY}{\mathcal Y}


\newcommand{\bP}{\mathbb{P}}

\newcommand{\C}{\mathbb{C}}
\newcommand{\R}{\mathbb{R}}
\newcommand{\N}{\mathbb{N}}
\newcommand{\Q}{\mathbb{Q}}
\newcommand{\Z}{\mathbb{Z}}
\newcommand{\F}{\mathbb{F}}

% Brackets
\newcommand{\bigp}[1]{\left( #1 \right)} % (x)
\newcommand{\bigb}[1]{\left[ #1 \right]} % [x]
\newcommand{\bigc}[1]{\left\{ #1 \right\}} % {x}
\newcommand{\biga}[1]{\left\langle #1 \right\rangle} % <x>

%norm

% theorem 
\newtheorem{Proposition}{proposition}
\newtheorem{Assumption}{assumption}
\newtheorem{Definition}{definition}
\newtheorem{Corollary}{corollary}
\newtheorem{Question}{question}



% \begin{document}

% \usepackage{xianzhi}

\begin{document}

\maketitle
RG HW 02 Question 3
\pagebreak

\begin{homeworkProblem}
    First we fix a chart 
    \begin{align}
        \Phi^{-1} : \mathbb{R}^2 &\xrightarrow[]{} S^2 \\
        (\phi, \theta) &\xrightarrow[]{} \Phi^{-1}(\phi, \theta) := (R \cos \phi \sin \theta, R \sin \phi \sin \theta, R \cos \theta)
    \end{align}
    where $R$ is the radius of sphere $S^2$.\\
    Now we calculate the Riemannian Metric induced from $\mathbb{R}^3$.
    \begin{align}
        (\phi, \theta) &\xrightarrow[]{\Phi^{-1}} (x,y,z)\\
        \frac{ \partial }{ \partial \phi } \left( \Phi^{-1} \left( \phi, \theta \right)\right) 
        &= \frac{ \partial x }{ \partial \phi } \frac{ \partial  }{ \partial x } \left( \Phi^{-1} \left( \phi, \theta \right)\right) 
        + \frac{ \partial y }{ \partial \phi } \frac{ \partial  }{ \partial y } \left( \Phi^{-1} \left( \phi, \theta \right)\right) 
        + \frac{ \partial z }{ \partial \phi } \frac{ \partial }{ \partial z } \left( \Phi^{-1} \left( \phi, \theta \right)\right) \\
        &= (-R \sin \phi \sin \theta, R \cos \phi \sin \theta, 0)
    \end{align}
    In basis $\frac{ \partial }{ \partial x }, \frac{ \partial }{ \partial y }, \frac{ \partial }{ \partial z }$.\\
    \begin{align}
        \frac{ \partial }{ \partial \theta } &= \frac{ \partial x }{ \partial \theta } \frac{ \partial }{ \partial x } 
        + \frac{ \partial y }{ \partial \theta } \frac{ \partial }{ \partial y }
        + \frac{ \partial z }{ \partial \theta } \frac{ \partial  }{ \partial z }\\
        &= (R \cos \phi \cos \theta, R \sin \phi \cos \theta, -R \sin \theta)\\
        g_{\theta, \theta} &= \left\langle \partial_{\theta}, \partial_{\theta} \right\rangle 
        = \left\langle \frac{ \partial  }{ \partial \theta }, \frac{ \partial  }{ \partial \theta }\right\rangle
        = R^2 \cos^2 \phi \cos^2 \theta + R^2 \sin^2 \phi \cos^2 \theta + R^2 \sin^2 \theta = R^2
    \end{align}
    similarly,
    \begin{align}
        g_{\phi \phi} = \left\langle \partial_{\phi}, \partial_{\phi} \right\rangle
        = \left\langle \frac{ \partial }{ \partial \phi }, \frac{ \partial }{ \partial \phi } \right\rangle
        = R^2 \sin^2 \phi \sin^2 \theta + R^2 \cos^2 \phi \sin^2 \theta 
        = R^2 \sin^2 \theta
    \end{align}
    similarly,
    \begin{align}
        g_{\theta \phi} = g_{\phi \theta}
        = \left\langle \frac{ \partial }{ \partial \phi }, \frac{ \partial }{ \partial \theta } \right\rangle 
        = - R^2 \sin \phi \sin \theta \cos \phi \cos \theta
        + R^2 \cos \phi \sin \theta \sin \phi \cos \theta
        + 0
        = 0
    \end{align}
    Thus,
    \begin{align}
        \left[ g_{\phi \theta} \right] =
        \begin{bmatrix}
            g_{\phi \phi} & g_{\phi \theta} \\
            g_{\theta \phi} & g_{\theta \theta}\\
        \end{bmatrix}
        =
        \begin{bmatrix}
            R^2 \sin^2 \theta & 0 \\
            0 & R^2
        \end{bmatrix}
    \end{align}
    $\phi$ is (encoded 1) $x$-axis, $\theta$ (encoded 2) is $y$-axis.\\
    \begin{align}
        \begin{bmatrix}
        g_{\phi \phi} & g_{\phi \theta} \\
        g_{\theta \phi} & g_{\theta \theta} \\
        \end{bmatrix}
        =
        \begin{bmatrix}
            R^2 \sin^2 \theta & 0 \\
            0 & R^2
        \end{bmatrix}
        =: \left[ g_{\phi \theta} \right]
    \end{align}
    similarly
    \begin{align}
        \begin{bmatrix}
            g^{\phi \phi} & g^{\phi \theta}\\
            g^{\theta \phi} & g^{\theta \theta}\\
        \end{bmatrix}
        =
        \begin{bmatrix}
            \frac{ 1 }{ R^2 \sin^{2} \theta } & 0 \\
            0 & \frac{ 1 }{ R^{2} }\\
        \end{bmatrix} =: \left[ g^{\phi \theta}\right]
    \end{align}
    Now, 
    \begin{align}
        \Gamma_{\phi \phi}^{\phi} &= \sum_{\ell = 1}^2 \frac{ 1 }{ 2 } \left( \partial_{\phi} g_{\phi \ell} + \partial_{\phi} g_{\phi \ell} 
        - \partial_{\ell} g_{\phi \phi}\right)
        g^{\ell \phi}\\
        &= \frac{ 1 }{ 2 } \left( \frac{ \partial }{ \partial \phi } g_{\phi \phi} + 
        \frac{ \partial }{ \partial \phi } g_{\phi \phi}
        - \frac{ \partial }{ \partial \phi } g_{\phi \phi}\right)
        g^{\phi \phi}\\
        &+ \frac{ 1 }{ 2 }
        \left(\frac{ \partial }{ \partial \phi } g_{\phi \theta}
        + \frac{ \partial }{ \partial \phi }g_{\phi \theta}
    -\frac{ \partial }{ \partial \theta } g_{\phi \phi}\right)
    g^{\theta \phi}\\
    &= 0 \cdot \frac{ 1 }{ R^2 \sin^{2} \theta } + \frac{ 1 }{ 2 }\left( - \frac{ \partial }{ \partial \theta }(R^2 \sin^2 \theta) \right) \cdot 0 = 0
    \end{align}
    similarly,
    \begin{align}
        \Gamma^{\phi}_{\theta \theta} &= \sum_{\ell = 1}^2 \frac{ 1 }{ 2 }
        \left( \partial_{\theta} g_{\theta \ell}
        + \partial_{\theta} g_{\theta \ell}
    -\partial_{\ell} g_{\theta \theta}\right)
    g^{\ell \phi}\\
    &= \frac{ 1 }{ 2 }
    \left( \partial_{\theta} g_{\theta \phi}
    + \partial_{\theta} g_{\theta \phi}
- \partial_{\phi} g_{\theta \theta}\right)
g^{\phi \phi}\\
&+ \frac{ 1 }{ 2 }
\left( \partial_{\theta} g_{\theta \theta}
+ \partial_{\theta} g_{\theta \theta}
- \partial_{\theta} g_{\theta \theta}\right)
g^{\theta \phi}\\
&= \frac{ 1 }{ 2 } (0+0-0)\frac{ 1 }{ R^2 \sin^2 \theta } + \frac{ 1 }{ 2 }(0+0-0)\cdot 0
=0
    \end{align}
    Similarly,
    \begin{align}
        \Gamma^{\theta}_{\theta \theta} 
        &= \sum_{\ell = 1}^2 \frac{ 1 }{ 2 }
        \left( \partial_{\theta} g_{\phi \ell}
    +\partial_{\phi} g_{\theta \ell}
-\partial_{\ell} g_{\theta \phi} \right) 
g^{\ell \theta}\\
&= \frac{ 1 }{ 2 }
\left(
\partial_{\theta} g_{\phi \phi} 
+\partial_{\theta} g_{\theta \phi} 
-\partial_{\phi} g_{\theta \phi} 
\right)
g^{\phi \theta}\\
&+ \ldots\\
&= \frac{ 1 }{ 2 }(0+0-0) \cdot 0 
+ \frac{ 1 }{ 2 } (0 + 0 - 0) \frac{ 1 }{ R^2 } =0
    \end{align}
    
    
    
    
    
    
    
    

             
    


\end{homeworkProblem}

\pagebreak



\begin{homeworkProblem}
    Find the derivative of \(f(x) = x^4 + 3x^2 - 2\)
\end{homeworkProblem}


\end{document}
