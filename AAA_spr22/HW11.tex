\documentclass{article}

%
% Homework Details
%   - Title
%   - Due date
%   - Class
%   - Section/Time
%   - Instructor
%   - Author
%

\newcommand{\hmwkTitle}{AAA \ \#11}
\newcommand{\hmwkDueDate}{May 2022}
\newcommand{\hmwkClass}{Adv Abstract Algebra}
\newcommand{\hmwkClassTime}{Spr 2022}
\newcommand{\hmwkClassInstructor}{Prof. Peter Hermann}
\newcommand{\hmwkAuthorName}{\textbf{Xianzhi} }
\newcommand{\hmwkAuthor}{\textit{Xianzhi Wang}}


\usepackage{fancyhdr}
\usepackage{extramarks}
\usepackage{amsmath}
\usepackage{amsthm}
\usepackage{amsfonts}
\usepackage{tikz}
\usepackage[plain]{algorithm}
\usepackage{algpseudocode}

\usetikzlibrary{automata,positioning}

%
% Basic Document Settings
%

\topmargin=-0.45in
\evensidemargin=0in
\oddsidemargin=0in
\textwidth=6.5in
\textheight=9.0in
\headsep=0.25in

\linespread{1.1}

\pagestyle{fancy}
\lhead{\hmwkAuthorName}
\chead{\hmwkClass\ (\hmwkClassInstructor\ \hmwkClassTime): \hmwkTitle}
\rhead{\firstxmark}
\lfoot{\lastxmark}
\cfoot{\thepage}

\renewcommand\headrulewidth{0.4pt}
\renewcommand\footrulewidth{0.4pt}

\setlength\parindent{0pt}

%
% Create Problem Sections
%

\newcommand{\enterProblemHeader}[1]{
    \nobreak\extramarks{}{Problem \arabic{#1} continued on next page\ldots}\nobreak{}
    \nobreak\extramarks{Problem \arabic{#1} (continued)}{Problem \arabic{#1} continued on next page\ldots}\nobreak{}
}

\newcommand{\exitProblemHeader}[1]{
    \nobreak\extramarks{Problem \arabic{#1} (continued)}{Problem \arabic{#1} continued on next page\ldots}\nobreak{}
    \stepcounter{#1}
    \nobreak\extramarks{Problem \arabic{#1}}{}\nobreak{}
}

\setcounter{secnumdepth}{0}
\newcounter{partCounter}
\newcounter{homeworkProblemCounter}
\setcounter{homeworkProblemCounter}{1}
\nobreak\extramarks{Problem \arabic{homeworkProblemCounter}}{}\nobreak{}

%
% Homework Problem Environment
%
% This environment takes an optional argument. When given, it will adjust the
% problem counter. This is useful for when the problems given for your
% assignment aren't sequential. See the last 3 problems of this template for an
% example.
%
\newenvironment{homeworkProblem}[1][-1]{
    \ifnum#1>0
        \setcounter{homeworkProblemCounter}{#1}
    \fi
    \section{Problem \arabic{homeworkProblemCounter}}
    \setcounter{partCounter}{1}
    \enterProblemHeader{homeworkProblemCounter}
}{
    \exitProblemHeader{homeworkProblemCounter}
}


%
% Title Page
%

\title{
    \vspace{2in}
    \textmd{\textbf{\hmwkClass:\ \hmwkTitle}}\\
    \normalsize\vspace{0.1in}\small{Due\ on\ \hmwkDueDate\ at 11:59PM}\\
    \vspace{0.1in}\large{\textit{\hmwkClassInstructor\ \hmwkClassTime}}
    \vspace{3in}
}

\author{\hmwkAuthorName}
\date{2023}

\renewcommand{\part}[1]{\textbf{\large Part \Alph{partCounter}}\stepcounter{partCounter}\\}

%
% Various Helper Commands
%

% Useful for algorithms
\newcommand{\alg}[1]{\textsc{\bfseries \footnotesize #1}}

% For derivatives
\newcommand{\deriv}[1]{\frac{\mathrm{d}}{\mathrm{d}x} (#1)}

% For partial derivatives
\newcommand{\pderiv}[2]{\frac{\partial}{\partial #1} (#2)}

% Integral dx
\newcommand{\dx}{\mathrm{d}x}

% Alias for the Solution section header
\newcommand{\solution}{\textbf{\large Solution:}}

% Probability commands: Expectation, Variance, Covariance, Bias
\newcommand{\E}{\mathrm{E}}
\newcommand{\Var}{\mathrm{Var}}
\newcommand{\Cov}{\mathrm{Cov}}
\newcommand{\Bias}{\mathrm{Bias}}

% From xianzhi.sty

% Fancy
\newcommand{\cA}{\mathcal A}
\newcommand{\cB}{\mathcal B}
\newcommand{\cC}{\mathcal C}
\newcommand{\cD}{\mathcal D}
\newcommand{\cE}{\mathcal E}
\newcommand{\cF}{\mathcal F}
\newcommand{\cG}{\mathcal G}
\newcommand{\cH}{\mathcal H}
\newcommand{\cI}{\mathcal I}
\newcommand{\cJ}{\mathcal J}
\newcommand{\cL}{\mathcal L}
\newcommand{\cM}{\mathcal M}
\newcommand{\cN}{\mathcal N}
\newcommand{\cO}{\mathcal O}
\newcommand{\cP}{\mathcal P}
\newcommand{\cR}{\mathcal R}
\newcommand{\cS}{\mathcal S}
\newcommand{\cT}{\mathcal T}
\newcommand{\cU}{\mathcal U}
\newcommand{\cW}{\mathcal W}
\newcommand{\cX}{\mathcal X}
\newcommand{\cY}{\mathcal Y}


\newcommand{\bP}{\mathbb{P}}

\newcommand{\C}{\mathbb{C}}
\newcommand{\R}{\mathbb{R}}
\newcommand{\N}{\mathbb{N}}
\newcommand{\Q}{\mathbb{Q}}
\newcommand{\Z}{\mathbb{Z}}
\newcommand{\F}{\mathbb{F}}

% Brackets
\newcommand{\bigp}[1]{\left( #1 \right)} % (x)
\newcommand{\bigb}[1]{\left[ #1 \right]} % [x]
\newcommand{\bigc}[1]{\left\{ #1 \right\}} % {x}
\newcommand{\biga}[1]{\left\langle #1 \right\rangle} % <x>

%norm

% theorem 
\newtheorem{Proposition}{proposition}
\newtheorem{Assumption}{assumption}
\newtheorem{Definition}{definition}
\newtheorem{Corollary}{corollary}
\newtheorem{Question}{question}



% \begin{document}

% \usepackage{xianzhi}

\begin{document}

\maketitle
HW11
\pagebreak

\begin{homeworkProblem}
    Which of the following actions is primitive?
    \begin{enumerate}
        \item 
        \begin{align}
            \left\langle \left( 1,2,3,4,5,6,7,8 \right), (1,2,3,4,5) \right\rangle \ \text{on} \ \left\{ 1,2,3,4,5,6,7,8 \right\};
        \end{align}
        \item 
        \begin{align}
            \left\langle \left( 1,2,3,4,5,6,7,8 \right), (2,4,6) \right\rangle \ \text{on} \ \left\{ 1,2,3,4,5,6,7,8 \right\};
        \end{align}
    \end{enumerate}
    \solution

    \part

    Let 
        \begin{align}
            G := \left\langle \left( 1,2,3,4,5,6,7,8 \right), (1,2,3,4,5) \right\rangle \ \text{act on} \ \Omega := \left\{ 1,2,3,4,5,6,7,8 \right\};
        \end{align}
        $G$ acts transitively on $\Omega$, since $G$ has a $8$-cycle that can take $i$ to $j$
        for any $i,j \in \Omega$ by raise this cycle to some power.\\
        Now, assume this action is not primitive, there are only 2 possibilities,
        \begin{enumerate}
            \item 4 blocks, each block has 2 elements
            \item 2 blocks, each has 4 elements
        \end{enumerate}
        $G$ act on the set of $4$ blocks: $X$, let
        $a = \left( 12345\right)$.
        \begin{align}
            &G \xrightarrow[]{\varphi} Sym (X) \cong S_4\\
            & order \ (\varphi(a)) \mid \ order \ (a) = 5\\
            & order \ (\varphi(a)) \mid \ \lvert S_4 \rvert = 4!\\
            &\implies \lvert \varphi (a) \rvert = 1
        \end{align}
        so $\varphi (a)$ is the identity permutation on $X$.\\
        Similarly, $a$ is identity permutation on 
        each element (block) in $B_i \in X$.
        So $a$ is the identity on 
        $\left\{ 1,2, \ldots, 8\right\}$. contradiction.\\

        $G$ act on the set of $2$ blocks: $Y$
        \begin{align}
            G \xrightarrow[]{\varphi} Sym ( Y ) \cong S_2
        \end{align}
        Similarly, let $a = \left( 12345 \right)$,
        then $\varphi(a)$ acts as identity permutation on $Y$,
        $\varphi(a)$ acts as identity on $4$ elements
        in each block $B_i \in Y$.
        Then $\varphi(a)$ is id on $\left\{ 1,2, \ldots, 8\right\}$.
        hence contradiction.\\
        Thus: the action is primitive.

        \part

        We could verify $\left\{ 2,4,6,8 \right\}$
        and $\left\{ 1,3,5,7 \right\}$ are 2 blocks, since
        \begin{align}
           \left\{ 1,3,5,7\right\} \xrightarrow[]{(12345678)} \left\{ 2,4,6,8 \right\} 
        \end{align}
        and $(246)$ fix the 2 blocks, since all other elements of the group 
        can be written as generators, those are indeed blocks.\\




\end{homeworkProblem}

\pagebreak


\begin{homeworkProblem}
    Consider the action of the dihedral group $D_n$ on the vertices. 
    Determine all values of $n$ for which the action is primitive.\\
    \solution 

    For $n=p$, a prime, the action is primitive 
    $(\lvert \Delta \rvert \mid \lvert \Omega \rvert)$
    If $n$ is a composite, $n=d \cdot r$ for some $1 < d < n$,
    if we mark $d$ vertices that all has $(r-1)$ vertices
    in between with the same color,
    then those $d$ vertices form a block.

\end{homeworkProblem}

\pagebreak

\begin{homeworkProblem}
    Let $A$ be a group and consider the \textit{right regular action} of $A$
    on $A$ [defined by $\alpha^g := \alpha \cdot g$ for all $\alpha, g \in A$].
    When will this action be primitive 
    (in terms of $A$)?\\
    \solution

    Let $A$ be a group, let $a\in A, \ a \neq e$, then 
    consider the cyclic group $\langle a \rangle \leq A$.
    If $\left\langle a \right\rangle = A$, then
    if further $A$ is prime order, then this right regular action is primitive.
    If $A$ has composite order then let $\lvert A \rvert = d \cdot r$,
    $1 < d < \lvert A \rvert$. We know there exist a cyclic subgroup of order $d$, 
    generated by $a^r$, so the cosets of $\left\langle a^r \right\rangle $ form a block system,
    (since cosets either disjoint or coincide completely).\\
    If $\langle a \rangle \lneqq A$,
    then consider the cosets of $\langle a \rangle$, which forms a block system.
    
\end{homeworkProblem}

\pagebreak

\begin{homeworkProblem}
    Let $G$ act on $\Omega$ and suppose that $N$ is a \textit{minimal normal subgroup}
    of $G$. [$1 \neq N \lhd G$ and there is no $1 \neq K \lneqq N$ 
    such that $K \lhd G$.]
    If $N$ is abelian and acts transitively  on $\Omega$ then 
    $G$ acts primitively on $\Omega$.\\
    \begin{enumerate}
        \item Hint: $N$ acts regularly on $\Omega$, i.e. $N_{\omega} = 1$ for all $\omega \in \Omega$.
            (Remember: in a transitively acting group stabilizers are 
            conjugate to each other.)
        \item Using that $N$ acts transitively prove that $G = G_{\alpha} \cdot N$
            for all $\alpha \in \Omega$; here, necessarily, $G_{\alpha} \cap N = 1$.
        \item If $G_{\alpha} \lneqq H \lneqq G$ then $G = H \cdot N$ and
            $H \cap N \neq 1$. In this case $H \cap N \lhd G$
            and $1 \lneqq H \cap N \lneqq N$.
    \end{enumerate}
    \solution

    \textbf{Lemma:} $G$ abelian, $G$ acts transitively, faithfully on $\Omega$,
    $\implies$ then this action is regular.\\

    WLOG, we assume $G$ acts faithfully on $\Omega$, since if $G$ is not faithfully acting, take
    \begin{align}
        \Bar{G} = G / ker (\varphi) \ \text{where} \ G \xrightarrow[]{\varphi} Sym(\Omega)
    \end{align}
    where $\varphi$ is the homomorphism.\\
    Then we prove $\Bar{G}$ acts primitively on $\Omega$ would imply $G$ acts 
    primitively on $\Omega$.\\
    Now $G \xrightarrow[]{\varphi} Sym (\Omega) \ ker (\varphi) = \left\{ 1 \right\}$.\\
    Then $N \lhd G$.
    \begin{align}
        N \xrightarrow[]{\varphi |_{N} } Sym (\Omega), \ ker (\varphi |_{N}) = \left\{ 1 \right\}.
    \end{align}
    The kernel of $\varphi$ restrict to $N$ is also $\left\{ 1 \right\}$, 
    so $N$ acts faithfully,
    $N$ is abelian, transitive on $\Omega$, then 
    $N$ acts regularly by lemma. 
    This implies $N_x = 1 \forall x \in \Omega$.\\
    $N$ transitive on $\Omega$ implies 
    for any $g \in G$ there exists $n \in N$ such that
    $x^g = x^n$ for all $x \in \Omega$.
    \begin{align}
        x^{gn^{-1}} &= x\\
        gn^{-1} &\in Stab_x\\
        g &\in Stab_x \cdot n \subseteq Stab_x \cdot N\\
        \implies G &\subseteq Stab_x \cdot N
    \end{align}
    since $Stab_x \cdot N \leq G$ ($N$ is normal)
    which implies $G = Stab_x \cdot N$ for all $x \in \Omega$\\
    so $G = G_x \cdot N$. $G_x$ denotes $Stab_x$ in $G$, and $N_x$ denotes $Stab_x$ in $N$.\\
    we have $G_x \cap N = N_x = \left\{ 1 \right\}$ since $N$ is regular.\\
    Claim: $G_x <_{max} G \ \text{where} \ (x \in \Omega)$.\\
    Suppose for contradiction $\exists K$ such that 
    \begin{align}
        G_x \lneqq K \lneqq G
    \end{align}
    then $K \cap N \lhd G$ (since $G = G_x \cdot N = K \cdot N$)
    and $N \leq N_G(K \cap N)$, $N$ in normalizer of $K \cap N$ since $N$ is abelian
    $K \leq N_G(K \cap N)$ since $N \lhd G$. So
    \begin{align}
        K \cdot N &\leq N_G(K \cap N)\\
        \implies G &\leq N_G(K \cap N)\\ 
        \implies G &= N_G(K \cap N)
    \end{align}
    Now, we consider $K \cap N$. If we show 
    \begin{align}
        e \lneqq K \cap N \lneqq N,
    \end{align}
    we obtain the desired contradiction, since $N$ is minimal normal subgroup.\\
    First, for sake of contradiction, if $K \cap N = N$, then $N \leq K$,
    which implies $G = K \cdot N =K$. Hence contradiction.\\
    Second we show $e \lneqq K \cap N$.\\
    Since $G = N \cdot G_x$, $G_x \lneqq K$, there exists $y \not\in G_x, y \in K$
    then since $y$ is also in $G$, we have $y = n \cdot h$ for $n \in N$,
    $h \in G_x \lneqq K$. $n \neq e$ since $y \not\in G_x$.
    Then $K \ni y h^{-1} = n \in N$, so $n = yh^{-1} \in K \cap N$,
    and $n \neq e$. Thus, $e \lneqq K \cap N$.\\

    Thus, we showed the claim $G_x <_{max} G \ (x \in \Omega)$.
    Use Thm: if $G$ act transitively on $\Omega$,
    then $G$ act primitively $\Leftrightarrow G_y <_{max} G, \ y \in \Omega$.\\
    ($G$ acts transitively since $N$ acts transitively, $N \lhd G$)\\

    
    
    
\end{homeworkProblem}

\begin{homeworkProblem}
    Consider the natural action of $S_n$ on $\Omega = \left\{ 1,2,3, \ldots, n \right\}$.
    Let $2 \leq k \leq \frac{ n }{ 2 }$ and define the action of $S_n$
    on the set $\cK := \left\{ T \mid T \subset \Omega \ \text{and} \ \lvert T \rvert = k \right\}$
    by means of $T^g := \left\{ t^g \mid t \in T \right\}$.
    Prove that $S_n$ acts primitively on $\cK$ 
    if and only if $k \neq \frac{ n }{ 2 }$.\\
    \solution

    $S_n$ acts primitively on $\cK \Leftrightarrow k \neq \frac{ n }{ 2 }.$\\
    $\implies$ \\
    To prove this direction, consider the contrapositive: 
    $k = \frac{ n }{ 2 } \implies S_n$ acts not primitively.\\
    Indeed, when $k = \frac{ n }{ 2 }$, we could form blocks:
    \begin{align}
        \left\{ a_1, a_2, \ldots, a_k \right\} \sim \left\{ b_1, b_2, \ldots b_k \right\}
    \end{align}
    where those two sets are disjoint set of numbers picked from $\Omega$.\\
    Since they have the same size, knowing $\left\{ a_1, a_2, \ldots, a_k \right\}$
    determines uniquely $\left\{ b_1, b_2, \ldots, b_k \right\}$,
    and since elements of $S_n$ are bijecctions, 
    they preserves this equivalence relation $\sim$.\\

    $\Leftarrow$\\
    Since $k \neq \frac{ n }{ 2 }$, and $2 \leq k < \frac{ n }{ 2 }, \ \implies k < n - k$ \\
    We have $\lvert T \rvert = k$, let $T \in \cK$, \\
    Claim: $(S_n)_T <_{max} S_n$.\\
    Observe the stabilizer of $T: \ (S_n)_T \cong S_k \times S_{n-k},$ 
    since we can freely permute the $k$ numbers in $T$ and the
    $(n-k)$ numbers outside $T$ among themselves.\\
    Let $g \not\in (S_n)_T$, Want To Show 
    $H := \langle (S_n)_T, g \rangle = S_n$.\\
    Since $g$ is some cycle, 
    \begin{align}
        &\exists \ \alpha_1 \in T \ s.t. \ \alpha_2 := \alpha_1^{g} \not\in T\\
        &\exists \ \beta_1 \not\in T \ s.t. \ \beta_2 := \beta_1^{g} \not\in T\\
    \end{align}
    consider transposition
    \begin{align}
        (\alpha_1 \beta_1) &= (\alpha_2^{g^{-1}} \beta_2^{g^{-1}}) 
        = g^{-1}(\alpha_2 \beta_2) g \in H = \langle (S_n)_T, g \rangle
    \end{align}
    $(\alpha_2, \beta_2) \in (S_n)_T$.
    I think for $n \in \mathbb{N}$, 
    
    

    
    
\end{homeworkProblem}


\end{document}
