\documentclass{article}

%
% Homework Details
%   - Title
%   - Due date
%   - Class
%   - Section/Time
%   - Instructor
%   - Author
%

\newcommand{\hmwkTitle}{GAL \ \#07}
\newcommand{\hmwkDueDate}{Apr 08, 2022}
\newcommand{\hmwkClass}{Galois Theory}
\newcommand{\hmwkClassTime}{Section 11}
\newcommand{\hmwkClassInstructor}{Prof Matyas Domokos}
\newcommand{\hmwkAuthorName}{\textbf{Xianzhi} }
\newcommand{\hmwkAuthor}{\textit{Xianzhi Wang}}


\usepackage{fancyhdr}
\usepackage{extramarks}
\usepackage{amsmath}
\usepackage{amsthm}
\usepackage{amsfonts}
\usepackage{tikz}
\usepackage[plain]{algorithm}
\usepackage{algpseudocode}

\usetikzlibrary{automata,positioning}

%
% Basic Document Settings
%

\topmargin=-0.45in
\evensidemargin=0in
\oddsidemargin=0in
\textwidth=6.5in
\textheight=9.0in
\headsep=0.25in

\linespread{1.1}

\pagestyle{fancy}
\lhead{\hmwkAuthorName}
\chead{\hmwkClass\ (\hmwkClassInstructor\ \hmwkClassTime): \hmwkTitle}
\rhead{\firstxmark}
\lfoot{\lastxmark}
\cfoot{\thepage}

\renewcommand\headrulewidth{0.4pt}
\renewcommand\footrulewidth{0.4pt}

\setlength\parindent{0pt}

%
% Create Problem Sections
%

\newcommand{\enterProblemHeader}[1]{
    \nobreak\extramarks{}{Problem \arabic{#1} continued on next page\ldots}\nobreak{}
    \nobreak\extramarks{Problem \arabic{#1} (continued)}{Problem \arabic{#1} continued on next page\ldots}\nobreak{}
}

\newcommand{\exitProblemHeader}[1]{
    \nobreak\extramarks{Problem \arabic{#1} (continued)}{Problem \arabic{#1} continued on next page\ldots}\nobreak{}
    \stepcounter{#1}
    \nobreak\extramarks{Problem \arabic{#1}}{}\nobreak{}
}

\setcounter{secnumdepth}{0}
\newcounter{partCounter}
\newcounter{homeworkProblemCounter}
\setcounter{homeworkProblemCounter}{1}
\nobreak\extramarks{Problem \arabic{homeworkProblemCounter}}{}\nobreak{}

%
% Homework Problem Environment
%
% This environment takes an optional argument. When given, it will adjust the
% problem counter. This is useful for when the problems given for your
% assignment aren't sequential. See the last 3 problems of this template for an
% example.
%
\newenvironment{homeworkProblem}[1][-1]{
    \ifnum#1>0
        \setcounter{homeworkProblemCounter}{#1}
    \fi
    \section{Problem \arabic{homeworkProblemCounter}}
    \setcounter{partCounter}{1}
    \enterProblemHeader{homeworkProblemCounter}
}{
    \exitProblemHeader{homeworkProblemCounter}
}


%
% Title Page
%

\title{
    \vspace{2in}
    \textmd{\textbf{\hmwkClass:\ \hmwkTitle}}\\
    \normalsize\vspace{0.1in}\small{Due\ on\ \hmwkDueDate\ at 11:59PM}\\
    \vspace{0.1in}\large{\textit{\hmwkClassInstructor\ \hmwkClassTime}}
    \vspace{3in}
}

\author{\hmwkAuthorName}
\date{2023}

\renewcommand{\part}[1]{\textbf{\large Part \Alph{partCounter}}\stepcounter{partCounter}\\}

%
% Various Helper Commands
%

% Useful for algorithms
\newcommand{\alg}[1]{\textsc{\bfseries \footnotesize #1}}

% For derivatives
\newcommand{\deriv}[1]{\frac{\mathrm{d}}{\mathrm{d}x} (#1)}

% For partial derivatives
\newcommand{\pderiv}[2]{\frac{\partial}{\partial #1} (#2)}

% Integral dx
\newcommand{\dx}{\mathrm{d}x}

% Alias for the Solution section header
\newcommand{\solution}{\textbf{\large Solution:}}

% Probability commands: Expectation, Variance, Covariance, Bias
\newcommand{\E}{\mathrm{E}}
\newcommand{\Var}{\mathrm{Var}}
\newcommand{\Cov}{\mathrm{Cov}}
\newcommand{\Bias}{\mathrm{Bias}}

% From xianzhi.sty

% Fancy
\newcommand{\cA}{\mathcal A}
\newcommand{\cB}{\mathcal B}
\newcommand{\cC}{\mathcal C}
\newcommand{\cD}{\mathcal D}
\newcommand{\cE}{\mathcal E}
\newcommand{\cF}{\mathcal F}
\newcommand{\cG}{\mathcal G}
\newcommand{\cH}{\mathcal H}
\newcommand{\cI}{\mathcal I}
\newcommand{\cJ}{\mathcal J}
\newcommand{\cL}{\mathcal L}
\newcommand{\cM}{\mathcal M}
\newcommand{\cN}{\mathcal N}
\newcommand{\cO}{\mathcal O}
\newcommand{\cP}{\mathcal P}
\newcommand{\cR}{\mathcal R}
\newcommand{\cS}{\mathcal S}
\newcommand{\cT}{\mathcal T}
\newcommand{\cU}{\mathcal U}
\newcommand{\cW}{\mathcal W}
\newcommand{\cX}{\mathcal X}
\newcommand{\cY}{\mathcal Y}


\newcommand{\bP}{\mathbb{P}}

\newcommand{\C}{\mathbb{C}}
\newcommand{\R}{\mathbb{R}}
\newcommand{\N}{\mathbb{N}}
\newcommand{\Q}{\mathbb{Q}}
\newcommand{\Z}{\mathbb{Z}}
\newcommand{\F}{\mathbb{F}}

% Brackets
\newcommand{\bigp}[1]{\left( #1 \right)} % (x)
\newcommand{\bigb}[1]{\left[ #1 \right]} % [x]
\newcommand{\bigc}[1]{\left\{ #1 \right\}} % {x}
\newcommand{\biga}[1]{\left\langle #1 \right\rangle} % <x>

%norm

% theorem 
\newtheorem{Proposition}{proposition}
\newtheorem{Assumption}{assumption}
\newtheorem{Definition}{definition}
\newtheorem{Corollary}{corollary}
\newtheorem{Question}{question}



% \begin{document}

% \usepackage{xianzhi}

\begin{document}

\maketitle
HW07 \\
Apr 08, 2022 \\
Exercise 11.4.2\\
Exercise 11.4.6\\
Exercise 11.4.8\\
\pagebreak

\begin{homeworkProblem}
    \textbf{Exercise 11.4.2} Show that $f \in K[X]$ (where $K$ is a subfield of $\C$) has a root
    in a radical extension of $K$ $\iff$ $f$ has an irreducible factor $p$ in $K[X]$ 
    such that $Gal_K(p)$ is solvable.\\
    \solution 

    ``$\Leftarrow$''\\
    Assume $f$ has irreducible factor $p$ such that $Gal_K(p)$ is solvable. Then apply
    Galois's theorem, $\exists$ a radical extension $L$ of $K$ contain all roots 
    of $p$, so $L$ must contain at least one root of $p$, call it $\alpha$.
    Since $p$ is a factor of $f$, $\alpha$ is also a root of $f$.
    Thus, $f$ has a root $\alpha$ in radical extension $L$.\\
    ``$\Rightarrow$''\\
    Assume $f\in K[X]$ has a root $\alpha$ in a radical extension $L$ of $K$.
    Thus, $L = K(\beta_1,\beta_2, \cdots, \beta_m)$ with $\beta_1, \cdots, \beta_m$
    a radical sequence.\\
    Let $p = m_K^{\alpha}$, then $p$ is automatically an irreducible factor of $f$,
    since $\alpha$ is a root of $f$.\\
    We want to show all the roots of $p$ are in some radical extension of $K$,
    but the radical extension we have, $L$, is not normal, so we modify it.
    Since 
    \begin{align}
        \beta_i^{n_i} \in K(\beta_1, \ldots, \beta_{i-1})
    \end{align}
    The sequence $\beta_i$ has corresponding sequence $n_i \in \N$, let
    \begin{align}
        L' = K(\beta_1, \zeta_{n_1}, \beta_2, \zeta_{n_2}, \ldots, \beta_m, \zeta_{n_m}) 
    \end{align}
    where $\zeta_{n_i}$ is a primitive $n_i$th root of unity.\\
    Since we obtain $L'$ by adjoin $\beta_i$ and $\zeta_{n_i}$,
    we join all the roots of the polynomial 
    \begin{align}
        X^{n_i} - \beta_i^{n_i}
    \end{align}
    at each step, so at each step we obtain a splitting field,
    and since we are in $\C$, $L'$ is a splitting field, 
    hence a normal extension of $K$, thus, since $\alpha \in L \subset L'$,
    all the roots of $p = m_K^{\alpha}$ are in $L'$, so
    $\exists$ radical extension $L'$
    containing all the roots of $p \ \implies \ Gal_K(p)$ is solvable by Galois's thm.\\

    \textit{Comment from instructor: The way you try to extend $L$ to get a normal
    extension of $K$ is not correct, see the following exercise: a normal extension
    of a normal extension is not necessarily normal.
    Instead, we gave a proof in class that 
    the normal closure of a radical extension is normal.}


\end{homeworkProblem}

\pagebreak


\begin{homeworkProblem}
    \textbf{Exercise 11.4.6} Suppose that $L:K$ and $M:L$ are normal extensions.
    Does it follow that $M:K$ is a normal extension?\\
    \solution 

    We observe that degree $2$ extensions are normal,
    since by a previous exercise, degree $2$ extension is obtained
    by adding ``a square root,'' so we add the other root too.
    \begin{align}
        L &= K(\alpha), \alpha \notin K, \alpha^2 \in K, \alpha \in L. \\ 
        [L:K] &= 2
    \end{align}
    Take irreducible polynomial that has $\alpha$ as a root, assume $\beta$ is another root. 
    \begin{align}
        X^2 + a X + b \ \  a,b \in K
    \end{align}
    then 
    \begin{align}
        \alpha + \beta &= -a \ \text{by Vieta's theorem} \\
        \alpha \beta &= b
    \end{align}
    so if $\alpha \in L$, then $\beta = -a -\alpha \in L$.\\

    Thus consider
    \begin{align}
        \Q \subseteq \Q (\sqrt[2]{3}) \subseteq \Q(\sqrt[4]{3})
    \end{align}
    and we claim that both extensions are degree 2 extension, and thus normal.\\
    \begin{align}
        [\Q(\sqrt[2]{3}) : \Q] &= 2 \ \text{since} \ m_{\Q}^{\sqrt[2]{3}} = X^2 - 3\\
        [\Q(\sqrt[4]{3}) : \Q] &= 4 \ \text{since} \ m_{\Q}^{\sqrt[4]{3}} = X^4 - 3\\
    \end{align}
    and both polynomial are irreducible by Eisenstein. Now, by Tower Law,
    \begin{align}
        [\Q (\sqrt[4]{3}) : \Q(\sqrt[2]{3})] = 2
    \end{align}
    But $\Q(\sqrt[4]{3}) : \Q$ is not normal. $X^4 - 3$ has (non-real) complex roots
    $i \sqrt[4]{3}, - i \sqrt[4]{3}$ not in $\Q(\sqrt[4]{3})$. And we are done.
    


\end{homeworkProblem}

\pagebreak

\begin{homeworkProblem}
    \textbf{Exercise 11.4.8} Find a degree $6$ irreducible polynomial $f \in \Q[X]$
    whose Galois group is isomorphic to $S_3$.\\
    \solution 

    $X^6 + 3$ is a degree $6$ irreducible polynomial $f \in \Q[X]$
    Let $L$ be splitting field of $X^6 + 3$ over $\Q$, then
    \begin{align}
        \Gamma (L: \Q) \cong S_3
    \end{align}
    We claim 
    \begin{align}
        L = \Q (\sqrt[6]{-3}, \zeta),
    \end{align}
    where $\zeta$ is a primitive 6th root of unity.
    \begin{align}
        \zeta = \frac{ 1 }{ 2 } + \frac{ \sqrt{3} }{ 2 } i
    \end{align}
    since $(\sqrt[6]{-3})^3 = \sqrt[2]{-3} = i \sqrt[2]{3}$ and
    \begin{align}
        \zeta = \frac{ 1 }{ 2 } + \frac{ 1 }{ 2 } (\sqrt[6]{-3})^3 \in \Q(\sqrt[6]{-3}) 
    \end{align}
    Thus, $L = \Q (\sqrt[6]{-3})$.\\
    $[L:\Q] = 6$ since $m_{\Q} \sqrt[6]{-3} = X^6 + 3$ is irreducible by Eisenstein.
    Then, since $L$ is splitting field over $\Q$, $L:\Q$ is Galois extension. 
    so $\lvert \Gamma(L:\Q) \rvert = 6$.\\
    Let $a := \sqrt[6]{-3}$\\
    Then $\phi \in \Gamma (L:\Q)$ need to take $a$ to some other root $\zeta^{k_{\phi}} a$
    for $0 \leq k_{\phi} \leq 5$.\\
    Up to isomorphism, there are only 2 group of order 6, $\Z_6$ and $D_3 \cong S_3$.\\
    Thus, suffice to show $\Gamma (L:\Q)$ is not abelian.\\
    Suffice to show $\Gamma (L:\Q)$ has a subgroup that is not normal.\\

    Consider
    \begin{align}
        \Q &\subseteq \Q (\sqrt[3]{-3}) \subseteq \Q (\sqrt[6]{-3})\\
        [\Q (\sqrt[3]{-3}) : \Q] &= 3 \ \text{since} \ m_{\Q} \sqrt[3]{-3} = X^3 + 3
    \end{align}
    is irreducible by Eisenstein. But $\Q(\sqrt[3]{-3})$ is not normal over $\Q$, since
    $X^3 + 3$ has roots $\omega \cdot \sqrt[3]{-3}$  and $\omega^2 \cdot \sqrt[3]{-3}$
    for $\omega = (-1 + i\sqrt{3})/2$.
    \begin{align}
        \omega \cdot \sqrt[3]{-3} \notin \Q (\sqrt[3]{-3}) \iff \omega \notin \Q (\sqrt[3]{-3})
    \end{align}
    since $[\Q (\omega):\Q] = 2$, and $\omega$ is root of $X^2 + X + 1$.\\
    $2 \ndiv 3$ so $\omega \notin \Q (\sqrt[3]{-3})$\\ 
    $\implies \Q \subseteq \Q (\sqrt[3]{-3})$  not normal, by Galois correspondence,
    \begin{align}
        L &:= \Q(\sqrt[6]{-3})\\
        \Gamma (L : \Q (\sqrt[3]{-3})) \ &\text{not normal in} \ \Gamma(L:\Q)
    \end{align}
    so $\Gamma (L:\Q) \cong S_3$. And we are done.
    



    
    
    

    
    
    



\end{homeworkProblem}

\end{document}
