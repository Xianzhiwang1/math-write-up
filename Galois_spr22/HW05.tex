\documentclass{article}

%
% Homework Details
%   - Title
%   - Due date
%   - Class
%   - Section/Time
%   - Instructor
%   - Author
%

\newcommand{\hmwkTitle}{GAL \ \#05}
\newcommand{\hmwkDueDate}{Mar 18, 2022}
\newcommand{\hmwkClass}{Galois Theory}
\newcommand{\hmwkClassTime}{Section 7}
\newcommand{\hmwkClassInstructor}{Prof Matyas Domokos}
\newcommand{\hmwkAuthorName}{\textbf{Xianzhi} }
\newcommand{\hmwkAuthor}{\textit{Xianzhi Wang}}


\usepackage{fancyhdr}
\usepackage{extramarks}
\usepackage{amsmath}
\usepackage{amsthm}
\usepackage{amsfonts}
\usepackage{tikz}
\usepackage[plain]{algorithm}
\usepackage{algpseudocode}

\usetikzlibrary{automata,positioning}

%
% Basic Document Settings
%

\topmargin=-0.45in
\evensidemargin=0in
\oddsidemargin=0in
\textwidth=6.5in
\textheight=9.0in
\headsep=0.25in

\linespread{1.1}

\pagestyle{fancy}
\lhead{\hmwkAuthorName}
\chead{\hmwkClass\ (\hmwkClassInstructor\ \hmwkClassTime): \hmwkTitle}
\rhead{\firstxmark}
\lfoot{\lastxmark}
\cfoot{\thepage}

\renewcommand\headrulewidth{0.4pt}
\renewcommand\footrulewidth{0.4pt}

\setlength\parindent{0pt}

%
% Create Problem Sections
%

\newcommand{\enterProblemHeader}[1]{
    \nobreak\extramarks{}{Problem \arabic{#1} continued on next page\ldots}\nobreak{}
    \nobreak\extramarks{Problem \arabic{#1} (continued)}{Problem \arabic{#1} continued on next page\ldots}\nobreak{}
}

\newcommand{\exitProblemHeader}[1]{
    \nobreak\extramarks{Problem \arabic{#1} (continued)}{Problem \arabic{#1} continued on next page\ldots}\nobreak{}
    \stepcounter{#1}
    \nobreak\extramarks{Problem \arabic{#1}}{}\nobreak{}
}

\setcounter{secnumdepth}{0}
\newcounter{partCounter}
\newcounter{homeworkProblemCounter}
\setcounter{homeworkProblemCounter}{1}
\nobreak\extramarks{Problem \arabic{homeworkProblemCounter}}{}\nobreak{}

%
% Homework Problem Environment
%
% This environment takes an optional argument. When given, it will adjust the
% problem counter. This is useful for when the problems given for your
% assignment aren't sequential. See the last 3 problems of this template for an
% example.
%
\newenvironment{homeworkProblem}[1][-1]{
    \ifnum#1>0
        \setcounter{homeworkProblemCounter}{#1}
    \fi
    \section{Problem \arabic{homeworkProblemCounter}}
    \setcounter{partCounter}{1}
    \enterProblemHeader{homeworkProblemCounter}
}{
    \exitProblemHeader{homeworkProblemCounter}
}


%
% Title Page
%

\title{
    \vspace{2in}
    \textmd{\textbf{\hmwkClass:\ \hmwkTitle}}\\
    \normalsize\vspace{0.1in}\small{Due\ on\ \hmwkDueDate\ at 11:59PM}\\
    \vspace{0.1in}\large{\textit{\hmwkClassInstructor\ \hmwkClassTime}}
    \vspace{3in}
}

\author{\hmwkAuthorName}
\date{2023}

\renewcommand{\part}[1]{\textbf{\large Part \Alph{partCounter}}\stepcounter{partCounter}\\}

%
% Various Helper Commands
%

% Useful for algorithms
\newcommand{\alg}[1]{\textsc{\bfseries \footnotesize #1}}

% For derivatives
\newcommand{\deriv}[1]{\frac{\mathrm{d}}{\mathrm{d}x} (#1)}

% For partial derivatives
\newcommand{\pderiv}[2]{\frac{\partial}{\partial #1} (#2)}

% Integral dx
\newcommand{\dx}{\mathrm{d}x}

% Alias for the Solution section header
\newcommand{\solution}{\textbf{\large Solution:}}

% Probability commands: Expectation, Variance, Covariance, Bias
\newcommand{\E}{\mathrm{E}}
\newcommand{\Var}{\mathrm{Var}}
\newcommand{\Cov}{\mathrm{Cov}}
\newcommand{\Bias}{\mathrm{Bias}}

% From xianzhi.sty

% Fancy
\newcommand{\cA}{\mathcal A}
\newcommand{\cB}{\mathcal B}
\newcommand{\cC}{\mathcal C}
\newcommand{\cD}{\mathcal D}
\newcommand{\cE}{\mathcal E}
\newcommand{\cF}{\mathcal F}
\newcommand{\cG}{\mathcal G}
\newcommand{\cH}{\mathcal H}
\newcommand{\cI}{\mathcal I}
\newcommand{\cJ}{\mathcal J}
\newcommand{\cL}{\mathcal L}
\newcommand{\cM}{\mathcal M}
\newcommand{\cN}{\mathcal N}
\newcommand{\cO}{\mathcal O}
\newcommand{\cP}{\mathcal P}
\newcommand{\cR}{\mathcal R}
\newcommand{\cS}{\mathcal S}
\newcommand{\cT}{\mathcal T}
\newcommand{\cU}{\mathcal U}
\newcommand{\cW}{\mathcal W}
\newcommand{\cX}{\mathcal X}
\newcommand{\cY}{\mathcal Y}


\newcommand{\bP}{\mathbb{P}}

\newcommand{\C}{\mathbb{C}}
\newcommand{\R}{\mathbb{R}}
\newcommand{\N}{\mathbb{N}}
\newcommand{\Q}{\mathbb{Q}}
\newcommand{\Z}{\mathbb{Z}}
\newcommand{\F}{\mathbb{F}}

% Brackets
\newcommand{\bigp}[1]{\left( #1 \right)} % (x)
\newcommand{\bigb}[1]{\left[ #1 \right]} % [x]
\newcommand{\bigc}[1]{\left\{ #1 \right\}} % {x}
\newcommand{\biga}[1]{\left\langle #1 \right\rangle} % <x>

%norm

% theorem 
\newtheorem{Proposition}{proposition}
\newtheorem{Assumption}{assumption}
\newtheorem{Definition}{definition}
\newtheorem{Corollary}{corollary}
\newtheorem{Question}{question}



% \begin{document}

% \usepackage{xianzhi}

\begin{document}

\maketitle
HW05 \\
Exercise 7.2.5\\
Exercise 7.2.7\\
Exercise 7.2.8\\
\pagebreak

\begin{homeworkProblem}
    \textbf{Exercise 7.2.5} Let $\gamma = \sqrt{2 + \sqrt{2}}$.
    \begin{enumerate}
        \item Show that $\Q(\gamma):\Q$ is normal with cyclic Galois Group.\\
        \item Show that $\Q(\gamma, i) = \Q(\phi)$ with $\phi^4 = i$.
    \end{enumerate}

    \solution

    \part

    Let 
    \begin{align}
        \sqrt{2 + \sqrt{2}} &= X \\
        \sqrt{2} &= X^2 - 2 \\
        2 &= (X^2 - 2)^2 = X^4 - 4 X^2 + 4
    \end{align}
    Thus, $\sqrt{2+\sqrt{2}}$ is a root of $X^4 - 4X^2 + 2 =: f$
    Since $f$ is irreducible, by Eisenstein $(p=2)$, 
    $\mathbb{Q}(\gamma)$ is degree 4 over $\mathbb{Q}$,
    we could find the roots of $f$:
    \begin{align}
        f = \left(X + \sqrt{2+\sqrt{2}}\right) \left(X - \sqrt{2+\sqrt{2}} \right) \left(X + \sqrt{2-\sqrt{2}}\right) \left(X - \sqrt{2-\sqrt{2}}\right)
    \end{align}
    Let $\phi \in \Gamma(\mathbb{Q}(\gamma):\mathbb{Q})$, we know $\phi$ permutes the roots of $f$.\\
    If we can find an element $\phi$ with order strictly greater than $2$, 
    then since order of the element need to divide the order of the group,
    $\lvert \phi \rvert$ must be 4. Since up to isomorphism, group of order 4
    is $\mathbb{Z}_4$ and $\mathbb{Z}_2 \oplus \mathbb{Z}_2$, we know once 
    there exists $\lvert \phi \rvert = 4, \ \Gamma(\mathbb{Q}(\gamma):\mathbb{Q})$ must be cyclic.\\
    First, we show $ \mathbb{Q}(\gamma) $ is splitting field of $f$
    that has no multiple roots, which would imply it is normal.
    since 
    \begin{align}
        \sqrt{2+\sqrt{2}} = \gamma \in \mathbb{Q}(\gamma) \implies - \sqrt{2+\sqrt{2}} &\in \mathbb{Q}(\gamma)\\
        2+\sqrt{2} &= \gamma^2 \in \mathbb{Q}(\gamma) 
    \end{align}
    so $\sqrt{2} \in \Q(\gamma)$. Since
    \begin{align}
        \sqrt{2+\sqrt{2}} \cdot \sqrt{2 - \sqrt{2}} = \sqrt{2} \in \mathbb{Q}(\gamma) \implies \sqrt{2-\sqrt{2}} \in \mathbb{Q}(\gamma)  
    \end{align}
    so $-\sqrt{2-\sqrt{2}} \in \mathbb{Q}(\gamma)$.\\
    Thus, all 4 roots of $f$ are in $\mathbb{Q}(\gamma)$. And 
    these roots are all distinct. so
    \begin{align}
        \mathbb{Q} \left(\sqrt{2+\sqrt{2}}, -\sqrt{2+\sqrt{2}}, +\sqrt{2-\sqrt{2}}, - \sqrt{2-\sqrt{2}}\right) &\subseteq \mathbb{Q}(\gamma)\\
        \mathbb{Q} \left(\sqrt{2+\sqrt{2}}, -\sqrt{2+\sqrt{2}}, +\sqrt{2-\sqrt{2}}, - \sqrt{2-\sqrt{2}}\right) &\supseteq \mathbb{Q}(\gamma)\\
        \text{so} \ \mathbb{Q} \left(\sqrt{2+\sqrt{2}}, -\sqrt{2+\sqrt{2}}, +\sqrt{2-\sqrt{2}}, - \sqrt{2-\sqrt{2}}\right) &= \mathbb{Q}(\gamma)\\
    \end{align}
    Thus, $\mathbb{Q}(\gamma)$ is indeed splitting field of $f$ with no multiple roots,
    so $\mathbb{Q}(\gamma): \mathbb{Q}$ is Galois extension, 
    and is normal extension. 
    Now we find $\phi \in \Gamma(\mathbb{Q}(\gamma):\mathbb{Q})$ that has order $\geq 3$. Claim:
    \begin{align}
        \sqrt{2 + \sqrt{2}} \mapsto_{\phi} \sqrt{2 - \sqrt{2}}  
    \end{align}
    does the job.\\
    \textit{Why does such an automorphism exist?
    Answer: Galois group acts transitively on the roots of a minimal polynomial}



    Let 
    \begin{align}
    X_1 = - \sqrt{2 + \sqrt{2}}, \ X_2 = \sqrt{2 + \sqrt{2}}, \ X_3 = - \sqrt{2 - \sqrt{2}}, \ X_4 = \sqrt{2 - \sqrt{2}}
    \end{align}
    Hence 
    \begin{align}
        \phi \circ \phi \left( \sqrt{2 + \sqrt{2}}\right) &= \phi \left( \sqrt{2 - \sqrt{2}}\right) \\ 
        &= \phi \left( \frac{ \sqrt{2} }{ \sqrt{2+\sqrt{2}} }\right)\\
        &= \frac{ \phi (\sqrt{2}) }{ \phi \left( \sqrt{2 + \sqrt{2}}\right) }\\
        &= \frac{ - \sqrt{2} }{ \sqrt{2-\sqrt{2}} }\\
        &= - \sqrt{2 + \sqrt{2}}
    \end{align}
    We also have
    \begin{align}
        \phi(2) + \phi(\sqrt{2}) &= \phi(2 + \sqrt{2})\\
        &= \phi \left( \sqrt{2 + \sqrt{2}} \cdot \sqrt{2 + \sqrt{2}} \right)\\
        &= \phi \left( \sqrt{2 + \sqrt{2}} \right) \cdot \phi \left( \sqrt{2 + \sqrt{2}} \right)\\
        &= \left( \sqrt{2 - \sqrt{2}} \right) \cdot \left( \sqrt{2 - \sqrt{2}} \right)\\
        &= 2 - \sqrt{2}
    \end{align}
    so $\phi(2) + \phi(\sqrt{2}) = 2 - \sqrt{2}$,
    since $\phi(2) = 2$ implies $\phi(\sqrt{2}) = - \sqrt{2}$.\\
    Since $\phi^2$ is not identity automorphism, $\lvert \phi \rvert \geq 3,$ 
    and we found the desired $\phi$.\\
    We can check $\phi$ indeed permutes $X_1, X_2, X_3, X_4$.\\
    \begin{align}
        X_2 &\xrightarrow[]{\phi}    X_4\\
        \phi(X_1) &= \phi(-\sqrt[]{2+\sqrt[]{2}}) = - \sqrt[]{2-\sqrt[]{2}} = X_3\\
        \phi(X_3) &= \phi(- \sqrt[]{2-\sqrt[]{2}}) = (-1)\cdot (-\sqrt[]{2+\sqrt[]{2}}) = \sqrt[]{2+\sqrt[]{2}} = X_2\\
        \phi(X_4) &= - \sqrt[]{2+\sqrt[]{2}} = X_1\\
        \phi: X_1 &\xrightarrow[]{} X_3 \xrightarrow[]{} X_2 \xrightarrow[]{} X_4 \xrightarrow[]{} X_1
    \end{align}

    \part

    Show $\mathbb{Q}(\gamma, i) = \mathbb{Q}(\phi)$ with $\phi^4 = i$.\\
    Use formula
    \begin{align}
        \cos (A) &= 1 - 2 \sin^2 \frac{ A }{ 2 }\\
        1 - 2 \sin^2(22.5^{\circ}) &= \cos 45^{\circ}\\
        \sin^2 (22.5^{\circ}) &= \frac{ \sqrt[]{2}-1 }{ 2 \sqrt{2} }\\
        \sin^2 (22.5^{\circ}) &= \frac{ \sqrt[]{2-\sqrt[]{2}} }{ 2 } 
    \end{align}
    And also,
    \begin{align}
        \cos (A) &= 2 \cos^2 \frac{ A }{ 2 } - 1\\
        \cos 45^{\circ} &= 2 \cos^2 22.5^{\circ} - 1\\
        \frac{ 1 }{ \sqrt[]{2} }+1 &= 2 \cos^2 (22.5^{\circ})\\
        \sqrt[]{\frac{ 1+\sqrt{2} }{ 2 \sqrt{2} }} &= \cos (22.5^{\circ})\\
        \cos (22.5^{\circ}) &= \frac{ \sqrt[]{\sqrt[]{2}+2} }{ 2 }
    \end{align}
    Since 
    \begin{align}
        \phi = \frac{ \sqrt[]{\sqrt[]{2}+2} }{ 2 } + \frac{ \sqrt[]{2 - \sqrt[]{2}} }{ 2 } i \in \mathbb{Q}(\phi)
    \end{align}
    WTS $\phi \in \mathbb{Q}(\gamma, i)$.\\
    \begin{align}
        &i \in \mathbb{Q}(\gamma, i), \ \sqrt[]{\sqrt[]{2} + 2} = \gamma \in \mathbb{Q}(\gamma, i)\\
        &\sqrt[]{2}+2 = \gamma^2 \in \mathbb{Q}(\gamma, i) \implies \ \sqrt[]{2} \in \mathbb{Q}(\gamma, i)
    \end{align}
    Thus, since 
    \begin{align}
        \sqrt[]{\sqrt[]{2}+2} \cdot \sqrt[]{2 - \sqrt[]{2}} = \sqrt[]{2} 
        \implies \sqrt[]{2 - \sqrt[]{2}} \in \mathbb{Q}(\gamma, i)
    \end{align}
    so $\phi \in \mathbb{Q}(\gamma, i)$ as wanted.\\
    so $\mathbb{Q}(\phi) \subseteq \mathbb{Q}(\gamma, i)$.
    Now, we show $\mathbb{Q}(\phi)$ and $\mathbb{Q}(\gamma, i)$ has same degree over $\mathbb{Q}$,
    which implies they are equal.\\
    \begin{align}
        [\mathbb{Q}(\phi):\mathbb{Q}] = 8,
    \end{align}
    since $\varphi(16)=8$, there are 8 numbers less than 16 that
    coprime with 16:
    \begin{align}
        &1,3,5,7,9,11,13,15. \ \text{And also, we have }\\
        &[\mathbb{Q}(\gamma,i):\mathbb{Q}]=8
    \end{align}
    Why? Assume for contradiction $\sqrt[]{2 + \sqrt[]{2}} \in \mathbb{Q}(i)$,
    We know $\mathbb{Q}(i)$ is degree 2 extension with irreducible polynomial 
    $X^2 + 1$, with basis $\left\{ 1, i \right\}$.\\
    So there exists $a,b \in \mathbb{Q}$ such that
    \begin{align}
        \sqrt[]{2+\sqrt[]{2}} &= a + bi\\
        2 + \sqrt[]{2} &= a^2 - b^2 + 2abi
    \end{align}
    We have a contradiction since we have $\sqrt[]{2}$ in LHS,
    but no $\sqrt[]{2}$ in RHS.\\
    (If we let $\sqrt[]{2}=2abi$ then $2ab = (-i)\cdot \sqrt[]{2}$. contradiction)\\
    Hence $\sqrt[]{2+\sqrt[]{2}} \not\in \mathbb{Q}(i)$.\\ 
    So $[\mathbb{Q}(i)(\sqrt[]{2+\sqrt[]{2}}): \mathbb{Q}(i)] = 4$, 
    because 
    \begin{align}
        &\text{deg} \ \sqrt[]{2 + \sqrt[]{2}} \ \text{over} \ \mathbb{Q}(i) \\
        &= \ \text{deg} \ \sqrt[]{2+\sqrt[]{2}} \ \text{over} \ \mathbb{Q} 
        = 4.
    \end{align}
    So using the tower law:
    \begin{align}
        \left[ \mathbb{Q}\left( i, \sqrt[]{2 + \sqrt[]{2}} \right) : \mathbb{Q} \right] 
        &= 4 \cdot \left[ \mathbb{Q}(i) : \mathbb{Q} \right]\\
        &= 4 \cdot 2 = 8
    \end{align}
    so $\mathbb{Q}(i, \gamma) = \mathbb{Q}(\phi)$. And we are done.




\end{homeworkProblem}

\pagebreak


\begin{homeworkProblem}
    \textbf{Exercise 7.2.7} Find the degree of 
    \begin{align}
        \sqrt[5]{81} + 29\sqrt[5]{9} + 17\sqrt[5]{3} - 16
    \end{align}
    over $\Q$.

    \solution

    Observe that if we adjoint $\sqrt[5]{3}$ to $\mathbb{Q}$, then 
    \begin{align}
        \gamma := (\sqrt[5]{3})^4 + 29 (\sqrt[5]{3})^2 + 17 \sqrt[5]{3} - 16 \in \mathbb{Q}(\sqrt[5]{3}).
    \end{align}
    since $\sqrt[5]{3}$ is root of $X^5 - 3$, which is irreducible by Eisenstein.
    \begin{align}
        m_{\mathbb{Q}}(\sqrt[5]{3}) = X^5 - 3
    \end{align}
    and $\mathbb{Q}(\sqrt[5]{3})$ is degree $5$ extension.\\
    Since $\gamma \in \mathbb{Q}(\sqrt[5]{3})$, we have
    $[\mathbb{Q}(\gamma) : \mathbb{Q}]$ divides $[\mathbb{Q}(\sqrt[5]{3}:\mathbb{Q})]=5$
    so $\mathbb{Q}(\gamma)$ is either degree 1 or 5.\\
    If it's degree 1, then $\gamma \in \mathbb{Q},$ so there exists $q \in \mathbb{Q}$ such that 
    $\gamma = q$.
    \begin{align}
        \left( \sqrt[5]{3} \right)^4 + 29 \left( \sqrt[5]{3} \right)^2 + 17 \sqrt[5]{3} - 16 - q = 0
    \end{align}
    Thus, $\sqrt[5]{3}$ is a root of the above polynomial with coefficients in $\mathbb{Q}$,
    but this polynomial is degree $4$, contradicting the minimal polynomial 
    of $\sqrt[5]{3}$ having degree $5$.
    Thus, $\mathbb{Q}(\gamma)$ is degree 5. so the degree of $\gamma$ over $\mathbb{Q}$ is 5.

    


    
    

\end{homeworkProblem}

\pagebreak

\begin{homeworkProblem}
    \textbf{Exercise 7.2.8} Find the degree of $\sqrt[5]{81}$ over $\Q(\sqrt[81]{5})$.

    \solution

    First we show $\mathbb{Q}(\sqrt[5]{81}) = \mathbb{Q}(\sqrt[5]{3})$.\\
    Want to show $\sqrt[5]{81} \in \mathbb{Q}(\sqrt[5]{3})$. 
    Write $\sqrt[5]{81} = (\sqrt[5]{3})^4 \in \mathbb{Q}(\sqrt[5]{3})$.
    So $\mathbb{Q}(\sqrt[5]{81}) \subset \mathbb{Q}(\sqrt[5]{3})$.
    Want to show $\sqrt[5]{3} \in \mathbb{Q}(\sqrt[5]{81}):$
    write $3^{1/5} = \left( 3^{4/5} \right)^4 (3^{-1})^3 \in \mathbb{Q} \left( \sqrt[5]{81} \right)$
    so $\mathbb{Q}(\sqrt[5]{3}) \subseteq \mathbb{Q}(\sqrt[5]{81})$\\
    so $\mathbb{Q}(\sqrt[5]{81}) = \mathbb{Q}(\sqrt[5]{3})$\\
    Since $X^5 - 3$ is irreducible by eisenstein, 
    $\mathbb{Q}(\sqrt[5]{3})$ is degree $5$ over $\mathbb{Q}$,
    so $\mathbb{Q}(\sqrt[5]{81})$ is degree $5$ over $\mathbb{Q}$.\\
    $\mathbb{Q}(\sqrt[81]{5})$ is degree $81$ over $\mathbb{Q}$,
    since $X^{81} - 5$ is irreducible by Eisenstein.\\
    By earlier exercise
    \begin{align}
        [\mathbb{Q}(\sqrt[5]{81}, \sqrt[81]{5}) : \mathbb{Q}] 
        \leq deg m_{\mathbb{Q}}(\sqrt[5]{81}) \cdot deg m_{\mathbb{Q}}(\sqrt[81]{5}) = 5 \cdot 81
    \end{align}
    Also, since $[\mathbb{Q}(\sqrt[5]{81}):\mathbb{Q}] = 5$ and
    $[\mathbb{Q}(\sqrt[81]{5}: \mathbb{Q}] = 81$ divide
    $[\mathbb{Q}(\sqrt[5]{81}, \sqrt[81]{5}) : \mathbb{Q}]$
    by Tower law, $[\mathbb{Q}(\sqrt[5]{81}, \sqrt[81]{5}): \mathbb{Q}]$
    is a multiple of $5 \cdot 81$ so its exactly $5 \cdot 81$.\\
    Use tower law again,
    \begin{align}
        5 \cdot 81 &= [\mathbb{Q}(\sqrt[5]{81},\sqrt[81]{5}): \mathbb{Q}(\sqrt[81]{5})]
        \cdot \left[ \mathbb{Q}(\sqrt[81]{5} : \mathbb{Q} \right]\\
        &= 5 \cdot 81
    \end{align}
    Thus, $\sqrt[5]{81}$ is degree $5$ over $\mathbb{Q}(\sqrt[81]{5})$. 
    And we are done.
    
    


    



\end{homeworkProblem}


\end{document}
