\documentclass{article}

%
% Homework Details
%   - Title
%   - Due date
%   - Class
%   - Section/Time
%   - Instructor
%   - Author
%

\newcommand{\hmwkTitle}{AAA \ \#HW05}
\newcommand{\hmwkDueDate}{2022}
\newcommand{\hmwkClass}{Adv Abs Alg}
\newcommand{\hmwkClassTime}{Spr22}
\newcommand{\hmwkClassInstructor}{Prof. Peter Hermann}
\newcommand{\hmwkAuthorName}{\textbf{Xianzhi} }
\newcommand{\hmwkAuthor}{\textit{Xianzhi Wang}}


\usepackage{fancyhdr}
\usepackage{extramarks}
\usepackage{amsmath}
\usepackage{amsthm}
\usepackage{amsfonts}
\usepackage{tikz}
\usepackage[plain]{algorithm}
\usepackage{algpseudocode}

\usetikzlibrary{automata,positioning}

%
% Basic Document Settings
%

\topmargin=-0.45in
\evensidemargin=0in
\oddsidemargin=0in
\textwidth=6.5in
\textheight=9.0in
\headsep=0.25in

\linespread{1.1}

\pagestyle{fancy}
\lhead{\hmwkAuthorName}
\chead{\hmwkClass\ (\hmwkClassInstructor\ \hmwkClassTime): \hmwkTitle}
\rhead{\firstxmark}
\lfoot{\lastxmark}
\cfoot{\thepage}

\renewcommand\headrulewidth{0.4pt}
\renewcommand\footrulewidth{0.4pt}

\setlength\parindent{0pt}

%
% Create Problem Sections
%

\newcommand{\enterProblemHeader}[1]{
    \nobreak\extramarks{}{Problem \arabic{#1} continued on next page\ldots}\nobreak{}
    \nobreak\extramarks{Problem \arabic{#1} (continued)}{Problem \arabic{#1} continued on next page\ldots}\nobreak{}
}

\newcommand{\exitProblemHeader}[1]{
    \nobreak\extramarks{Problem \arabic{#1} (continued)}{Problem \arabic{#1} continued on next page\ldots}\nobreak{}
    \stepcounter{#1}
    \nobreak\extramarks{Problem \arabic{#1}}{}\nobreak{}
}

\setcounter{secnumdepth}{0}
\newcounter{partCounter}
\newcounter{homeworkProblemCounter}
\setcounter{homeworkProblemCounter}{1}
\nobreak\extramarks{Problem \arabic{homeworkProblemCounter}}{}\nobreak{}

%
% Homework Problem Environment
%
% This environment takes an optional argument. When given, it will adjust the
% problem counter. This is useful for when the problems given for your
% assignment aren't sequential. See the last 3 problems of this template for an
% example.
%
\newenvironment{homeworkProblem}[1][-1]{
    \ifnum#1>0
        \setcounter{homeworkProblemCounter}{#1}
    \fi
    \section{Problem \arabic{homeworkProblemCounter}}
    \setcounter{partCounter}{1}
    \enterProblemHeader{homeworkProblemCounter}
}{
    \exitProblemHeader{homeworkProblemCounter}
}


%
% Title Page
%

\title{
    \vspace{2in}
    \textmd{\textbf{\hmwkClass:\ \hmwkTitle}}\\
    \normalsize\vspace{0.1in}\small{Due\ on\ \hmwkDueDate\ at 11:59PM}\\
    \vspace{0.1in}\large{\textit{\hmwkClassInstructor\ \hmwkClassTime}}
    \vspace{3in}
}

\author{\hmwkAuthorName}
\date{2023}

\renewcommand{\part}[1]{\textbf{\large Part \Alph{partCounter}}\stepcounter{partCounter}\\}

%
% Various Helper Commands
%

% Useful for algorithms
\newcommand{\alg}[1]{\textsc{\bfseries \footnotesize #1}}

% For derivatives
\newcommand{\deriv}[1]{\frac{\mathrm{d}}{\mathrm{d}x} (#1)}

% For partial derivatives
\newcommand{\pderiv}[2]{\frac{\partial}{\partial #1} (#2)}

% Integral dx
\newcommand{\dx}{\mathrm{d}x}

% Alias for the Solution section header
\newcommand{\solution}{\textbf{\large Solution:}}

% Probability commands: Expectation, Variance, Covariance, Bias
\newcommand{\E}{\mathrm{E}}
\newcommand{\Var}{\mathrm{Var}}
\newcommand{\Cov}{\mathrm{Cov}}
\newcommand{\Bias}{\mathrm{Bias}}

% From xianzhi.sty

% Fancy
\newcommand{\cA}{\mathcal A}
\newcommand{\cB}{\mathcal B}
\newcommand{\cC}{\mathcal C}
\newcommand{\cD}{\mathcal D}
\newcommand{\cE}{\mathcal E}
\newcommand{\cF}{\mathcal F}
\newcommand{\cG}{\mathcal G}
\newcommand{\cH}{\mathcal H}
\newcommand{\cI}{\mathcal I}
\newcommand{\cJ}{\mathcal J}
\newcommand{\cL}{\mathcal L}
\newcommand{\cM}{\mathcal M}
\newcommand{\cN}{\mathcal N}
\newcommand{\cO}{\mathcal O}
\newcommand{\cP}{\mathcal P}
\newcommand{\cR}{\mathcal R}
\newcommand{\cS}{\mathcal S}
\newcommand{\cT}{\mathcal T}
\newcommand{\cU}{\mathcal U}
\newcommand{\cW}{\mathcal W}
\newcommand{\cX}{\mathcal X}
\newcommand{\cY}{\mathcal Y}


\newcommand{\bP}{\mathbb{P}}

\newcommand{\C}{\mathbb{C}}
\newcommand{\R}{\mathbb{R}}
\newcommand{\N}{\mathbb{N}}
\newcommand{\Q}{\mathbb{Q}}
\newcommand{\Z}{\mathbb{Z}}
\newcommand{\F}{\mathbb{F}}

% Brackets
\newcommand{\bigp}[1]{\left( #1 \right)} % (x)
\newcommand{\bigb}[1]{\left[ #1 \right]} % [x]
\newcommand{\bigc}[1]{\left\{ #1 \right\}} % {x}
\newcommand{\biga}[1]{\left\langle #1 \right\rangle} % <x>

%norm

% theorem 
\newtheorem{Proposition}{proposition}
\newtheorem{Assumption}{assumption}
\newtheorem{Definition}{definition}
\newtheorem{Corollary}{corollary}
\newtheorem{Question}{question}



% \begin{document}

% \usepackage{xianzhi}

\begin{document}

\maketitle
Homework Set 5
\pagebreak

\begin{homeworkProblem}
    Prove that none of $(\mathbb{Q}, +), \ (\mathbb{Q} \setminus \left\{ 0 \right\}, \cdot)$ 
    is finitely generated.\\
    \solution

    Assume for contradiction\\
    \begin{align}
        G := (\mathbb{Q} \setminus \left\{ 0 \right\}, \cdot)
    \end{align}
    is finitely generated. Then
    \begin{align}
        G = \langle \frac{ r_1 }{ s_1 }, \frac{ r_2 }{ s_2 }, \ldots, \frac{ r_n }{ s_n }\rangle 
    \end{align}
    where $r_i$ and $s_i$ are coprime.\\
    Now, take prime number $p > \max \left\{ r_i, s_j \mid 1 \leq i \leq n, 1 \leq j \leq n \right\}$
    since there are infinitely many primes numbers, 
    we can take such $p$, then $\frac{ 1 }{ p }$ cannot 
    be expressed using the generators. Because:
    A general element of $G$ is of the form:
    \begin{align}
        \frac{ r_1^{i_1}\ldots r_n^{i_n} }{ s_1^{i_1}\ldots s_n^{i_n} }
    \end{align}
    and if 
    \begin{align}
        \frac{ 1 }{ p } &= \frac{ r_1^{i_1}\ldots r_n^{i_n} }{ s_1^{i_1}\ldots s_n^{i_n} },\\
    p &= \frac{ s_1^{i_1}\ldots s_n^{i_n} }{r_1^{i_1}\ldots r_n^{i_n}} 
    \end{align}
    then it contradicts $p$ is prime.\\

    Assume for contradiction $G:= (\mathbb{Q}, +)$ is finitely generated,
    then $G = \langle \frac{ r_1 }{ s_1 }, \frac{ r_2 }{ s_2 }, \ldots \frac{ r_n }{ s_n }\rangle$
    where $r_i$ and $s_i$ are coprime.\\
    Again we take a prime number $p > \max \left\{ r_i, s_j \mid 1 \leq i \leq n, 1 \leq j \leq n \right\}$
    and $\frac{ 1 }{ p }$ cannot be expressed using generators. 
    Because: a general element is of the form $(k_1, k_2 \in \mathbb{Z})$.
    \begin{align}
        \frac{ k_2 }{ k_1 \cdot s_1 \cdot s_2 \ldots s_n }
    \end{align}
    so we have a contradiction if 
    \begin{align}
        \frac{ 1 }{ p } = \frac{ k_2 }{ k_1 \cdot s_1 \cdot s_2 \ldots s_n }
    \end{align}
    we could assume $k_2$ and $k_1 \cdot s_1 \ldots s_n$ are 
    coprime. Then $p = k_1 \cdot s_1 \cdot s_2 \ldots s_n$
    which is a contradiction, since $p$ cannot 
    have strictly smaller prime factors.
    
    


    
    
    
    



    
\end{homeworkProblem}

\pagebreak

\begin{homeworkProblem}
    Let $A$ be abelian.
    \begin{enumerate}
        \item Prove that $A$ is finitely generated if and only if 
        there exist finitely many subgroups $A_i$ such that 
        \begin{align}
            A = A_0 \geq A_1 \geq A_2 \geq \ldots \geq A_n \geq A_{n+1} = 1\\
        \end{align}
        and all the factor groups $A_i / A_{i+1}$ are cyclic.
    \item Let $B \leq A$ and assume that $A$ is finitely generated.
        Show that $B$ is finitely generated.
    \end{enumerate}
    \solution

    \part

    ``$\implies$''\\
    $A$ is finitely generated, so
    \begin{align}
        A = \left\langle a_1, a_2, \ldots, a_k \right\rangle
    \end{align}
    let 
    \begin{align}
        A = A_0 &= \left\langle a_1, \ldots, a_k \right\rangle\\
        A_1 &= \left\langle a_1, \ldots, a_{k-1} \right\rangle\\
        &\vdots\\
        A_{k-1} &= \left\langle a_1 \right\rangle\\
        A_k &= \left\{ e \right\}
    \end{align}
    so this shows existance.\\
    We define $\varphi$ from the cyclic group $\langle a_k \rangle$ to factor group:
    \begin{align}
        \langle a_k \rangle &\xrightarrow[]{\varphi} A_0 / A_1\\
        a_k^n &\xrightarrow[]{\varphi} A_1 \cdot a_k^n\\
    \end{align}
    In general
    \begin{align}
        \left\langle a_{k-i} \right\rangle &\xrightarrow[]{\varphi} A_i / A_{i+1}\\
        (a_{k-i})^n &\rightarrow A_{i+1} \cdot (a_{k-i})^n
    \end{align}
    Since $A$ is abelian, all subgroup is normal.\\
    This is well defined, since if we take an element
    in $A_0/A_1 : A_1 \cdot x$, where $x \in A_0$,
    then
    \begin{align}
        x = a_1^{n_1} \cdot a_2^{n_2} \cdot \ldots \cdot a_k^{n_k}
    \end{align}
    so 
    \begin{align}
        A_1 \cdot x = A_1 a_k^{n_k}
    \end{align}
    so coset representative is indeed of the form 
    $a_{k}$ raised to some power. This also shows
    $\varphi$ is onto. So $Im \varphi = A_0 / A_1$.\\
    $\varphi$ is homomorphism:
    \begin{align}
        \varphi (a_k^n)\varphi (a_k^m) &= (A_1 \cdot a_k^n) (A_1 \cdot a_k^m)\\
        &= A_1 \cdot a_k^{n+m}\\
        \varphi (a_k^n \cdot a_k^m) &= A_1 a_k^{n+m}
    \end{align}
    Thus, since image of a cyclic group under homomorphism 
    is cyclic, we conclude $A_0 / A_1$ is cyclic.\\
    The same proof shows $A_i / A_{i+1}$ is cyclic.\\
    \part

    ``$\Leftarrow$''\\
    \begin{align}
        \left\{ e\right\} = A_k \leq A_{k-1} \leq \ldots \leq A_1 \leq A_0 = A
    \end{align}
    since all $A_i / A_{i+1}$ are cyclic,
    let $a_{i+1}$ denote some generator
    \begin{align}
        A_i / A_{i+1} = \left\langle A_{i+1} \cdot a_{i+1} \right\rangle
    \end{align}
    so an arbitrary coset is of the form 
    \begin{align}
        A_{i+1} \cdot a_{i+1}^n
    \end{align}
    let $x \in A$. Then $x$ is in some $A_1$ coset.
    \begin{align}
        x \in A_1 \cdot a_1^{n_1}
    \end{align}
    Thus $x = x_1 \cdot a_1^{n_1}$ for some $x_1 \in A_1$. Again
    \begin{align}
        x_1 \in A_2 \cdot a_2^{n_2}
    \end{align}
    implies $x_1 = x_2 \cdot a_2^{n_2}$ for some $x_2 \in A_2$
    and so on.\\
    Thus, $x \in A_k \cdot a_{k-1}^{n_{k-1}} \cdot \ldots \cdot a_1^{n_1}$
    but $A_k = \left\{ e \right\}$, so 
    \begin{align}
        x = a_{k-1}^{n_{k-1}} \cdot \ldots \cdot a_1^{n_1}
    \end{align}
    so $A$ is finitely generated. And we are done.\\
    \part

    $A$ is finitely generated $\implies$ 
    \begin{align}
        \exists A = A_0 \geq \ldots \geq A_k = \left\{ e \right\}
    \end{align}
    and $A_i / A_{i+1}$ is cyclic.\\
    Let $B \leq A$. If $B=A$ or $B= \left\{ e\right\}$,
    then we are done. So assume $B$ 
    is a nontrivial proper subgroup. Let
    \begin{align}
        B_0 &= B \cap A_0\\
        B_1 &= B \cap A_1\\
        B_i &= B \cap A_i
    \end{align}
    in general.
    Thus, $B_i$ are chain of subgroups
    \begin{align}
        B = B_0 \geq B_1 \geq \ldots \geq B_k = \left\{ e\right\}
    \end{align}
    If we show $B_i / B_{i+1}$ is cyclic,
    we are done. To simplify notation, 
    we consider $B_1/ B_2$. Write
    \begin{align}
        \frac{ (B \cap A_1) }{ (B \cap A_2) } = \frac{ (B \cap A_1) }{ \underbrace{A_2}_{N} \cap \underbrace{(B \cap A_1)}_{H} }  
    \end{align}
    where the equal sign is because $A_2 \subseteq A_1$.
    Recall iso thm (we can use it since every subgroup is normal)
    \begin{align}
        \frac{ H }{ H \cap N } \cong \frac{ H \cdot N }{ N }
    \end{align}
    Thus
    \begin{align}
        B_1/B_2 &= \frac{ \underbrace{B \cap A_1 }_{H} }{ \underbrace{A_2}_{N} \cap \underbrace{B \cap A_1}_{H}  } 
        \cong \frac{ \underbrace{B \cap A_1}_{H} \cdot \underbrace{A_2}_{N} }{ A_2 }
    \end{align}
    since $M := (B \cap A_1)\cdot A_2 \subseteq A_1$,
    and $M$ is a subgroup, and $A_2 \leq M$, implies 
    $M$ is a subgroup of $A_1$,
    \begin{align}
        \implies M / A_2 \leq A_1 / A_2
    \end{align}
    since subgroup of cyclic group is cyclic, $M / A_2$ is cyclic.\\
    Hence, $B_1 / B_2$ is cyclic. $B_i / B_{i+1}$ is exactly the same.\\
    
    
    

\end{homeworkProblem}

\pagebreak

\begin{homeworkProblem}
Let $n \geq 2$ be an integer and $A = C_1 \times C_2 \times \ldots \times C_n$,
the direct product of $n$ infinite cyclic groups.
Show that (for any $a \in A$) $\lvert A : \langle a \rangle \rvert$ is infinite.\\
This completes the proof of the following theorem: $C \leq G$,
$C$ cyclic, $\lvert G: C \rvert$ finite, $G$ torsion-free $\implies G$ is cyclic.\\
\solution

Let 
\begin{align}
    A = C_1 \times C_2 \times \ldots \times C_n
\end{align}
where $C_i$ are infinite cyclic groups.\\
then $A \cong \mathbb{Z} \times \mathbb{Z} \times \ldots \times \mathbb{Z} = \mathbb{Z}^n$\\
take $a \in A$.\\
Then $a$ could be identified as 
\begin{align}
    (a_1, a_2, \ldots, a_n) \ \text{for} \ a_i \in \mathbb{Z}
\end{align}
using additive notation, then
\begin{align}
    \left\langle a \right\rangle &= \left\{ k \cdot a \mid k \in \mathbb{Z} \right\}\\
    &= \left\{ (k a_1, k a_2, \ldots, k a_n ) \mid k \in \mathbb{Z} \right\}
\end{align}
we could see this has infinite index in $\mathbb{Z}^d$, since 
it is just a ``line'' in the integer lattice $\mathbb{Z}^d$,
and we could have infinitely shifted copy of it, 
which are cosets.
\begin{align}
    (z_1, z_2, \ldots, z_n) + (k a_1, k a_2, \ldots, k a_n) 
\end{align}
for $z_i \in \mathbb{Z}, \ k \in \mathbb{Z}$.
Thus, we have infinitely many distinctive cosets, 
so $\left\langle a \right\rangle$ has infinite index.






    
\end{homeworkProblem}


\end{document}







