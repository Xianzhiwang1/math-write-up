\documentclass{article}

%
% Homework Details
%   - Title
%   - Due date
%   - Class
%   - Section/Time
%   - Instructor
%   - Author
%

\newcommand{\hmwkTitle}{AAA \ \#Final}
\newcommand{\hmwkDueDate}{May 2022}
\newcommand{\hmwkClass}{Adv Abstract Algebra}
\newcommand{\hmwkClassTime}{All Section}
\newcommand{\hmwkClassInstructor}{Prof. Peter Hermann}
\newcommand{\hmwkAuthorName}{\textbf{Xianzhi} }
\newcommand{\hmwkAuthor}{\textit{Xianzhi Wang}}


\usepackage{fancyhdr}
\usepackage{extramarks}
\usepackage{amsmath}
\usepackage{amsthm}
\usepackage{amsfonts}
\usepackage{tikz}
\usepackage[plain]{algorithm}
\usepackage{algpseudocode}

\usetikzlibrary{automata,positioning}

%
% Basic Document Settings
%

\topmargin=-0.45in
\evensidemargin=0in
\oddsidemargin=0in
\textwidth=6.5in
\textheight=9.0in
\headsep=0.25in

\linespread{1.1}

\pagestyle{fancy}
\lhead{\hmwkAuthorName}
\chead{\hmwkClass\ (\hmwkClassInstructor\ \hmwkClassTime): \hmwkTitle}
\rhead{\firstxmark}
\lfoot{\lastxmark}
\cfoot{\thepage}

\renewcommand\headrulewidth{0.4pt}
\renewcommand\footrulewidth{0.4pt}

\setlength\parindent{0pt}

%
% Create Problem Sections
%

\newcommand{\enterProblemHeader}[1]{
    \nobreak\extramarks{}{Problem \arabic{#1} continued on next page\ldots}\nobreak{}
    \nobreak\extramarks{Problem \arabic{#1} (continued)}{Problem \arabic{#1} continued on next page\ldots}\nobreak{}
}

\newcommand{\exitProblemHeader}[1]{
    \nobreak\extramarks{Problem \arabic{#1} (continued)}{Problem \arabic{#1} continued on next page\ldots}\nobreak{}
    \stepcounter{#1}
    \nobreak\extramarks{Problem \arabic{#1}}{}\nobreak{}
}

\setcounter{secnumdepth}{0}
\newcounter{partCounter}
\newcounter{homeworkProblemCounter}
\setcounter{homeworkProblemCounter}{1}
\nobreak\extramarks{Problem \arabic{homeworkProblemCounter}}{}\nobreak{}

%
% Homework Problem Environment
%
% This environment takes an optional argument. When given, it will adjust the
% problem counter. This is useful for when the problems given for your
% assignment aren't sequential. See the last 3 problems of this template for an
% example.
%
\newenvironment{homeworkProblem}[1][-1]{
    \ifnum#1>0
        \setcounter{homeworkProblemCounter}{#1}
    \fi
    \section{Problem \arabic{homeworkProblemCounter}}
    \setcounter{partCounter}{1}
    \enterProblemHeader{homeworkProblemCounter}
}{
    \exitProblemHeader{homeworkProblemCounter}
}


%
% Title Page
%

\title{
    \vspace{2in}
    \textmd{\textbf{\hmwkClass:\ \hmwkTitle}}\\
    \normalsize\vspace{0.1in}\small{Due\ on\ \hmwkDueDate\ at 11:59PM}\\
    \vspace{0.1in}\large{\textit{\hmwkClassInstructor\ \hmwkClassTime}}
    \vspace{3in}
}

\author{\hmwkAuthorName}
\date{2023}

\renewcommand{\part}[1]{\textbf{\large Part \Alph{partCounter}}\stepcounter{partCounter}\\}

%
% Various Helper Commands
%

% Useful for algorithms
\newcommand{\alg}[1]{\textsc{\bfseries \footnotesize #1}}

% For derivatives
\newcommand{\deriv}[1]{\frac{\mathrm{d}}{\mathrm{d}x} (#1)}

% For partial derivatives
\newcommand{\pderiv}[2]{\frac{\partial}{\partial #1} (#2)}

% Integral dx
\newcommand{\dx}{\mathrm{d}x}

% Alias for the Solution section header
\newcommand{\solution}{\textbf{\large Solution:}}

% Probability commands: Expectation, Variance, Covariance, Bias
\newcommand{\E}{\mathrm{E}}
\newcommand{\Var}{\mathrm{Var}}
\newcommand{\Cov}{\mathrm{Cov}}
\newcommand{\Bias}{\mathrm{Bias}}

% From xianzhi.sty

% Fancy
\newcommand{\cA}{\mathcal A}
\newcommand{\cB}{\mathcal B}
\newcommand{\cC}{\mathcal C}
\newcommand{\cD}{\mathcal D}
\newcommand{\cE}{\mathcal E}
\newcommand{\cF}{\mathcal F}
\newcommand{\cG}{\mathcal G}
\newcommand{\cH}{\mathcal H}
\newcommand{\cI}{\mathcal I}
\newcommand{\cJ}{\mathcal J}
\newcommand{\cL}{\mathcal L}
\newcommand{\cM}{\mathcal M}
\newcommand{\cN}{\mathcal N}
\newcommand{\cO}{\mathcal O}
\newcommand{\cP}{\mathcal P}
\newcommand{\cR}{\mathcal R}
\newcommand{\cS}{\mathcal S}
\newcommand{\cT}{\mathcal T}
\newcommand{\cU}{\mathcal U}
\newcommand{\cW}{\mathcal W}
\newcommand{\cX}{\mathcal X}
\newcommand{\cY}{\mathcal Y}


\newcommand{\bP}{\mathbb{P}}

\newcommand{\C}{\mathbb{C}}
\newcommand{\R}{\mathbb{R}}
\newcommand{\N}{\mathbb{N}}
\newcommand{\Q}{\mathbb{Q}}
\newcommand{\Z}{\mathbb{Z}}
\newcommand{\F}{\mathbb{F}}

% Brackets
\newcommand{\bigp}[1]{\left( #1 \right)} % (x)
\newcommand{\bigb}[1]{\left[ #1 \right]} % [x]
\newcommand{\bigc}[1]{\left\{ #1 \right\}} % {x}
\newcommand{\biga}[1]{\left\langle #1 \right\rangle} % <x>

%norm

% theorem 
\newtheorem{Proposition}{proposition}
\newtheorem{Assumption}{assumption}
\newtheorem{Definition}{definition}
\newtheorem{Corollary}{corollary}
\newtheorem{Question}{question}



% \begin{document}

% \usepackage{xianzhi}

\begin{document}

\maketitle
\#Final Exam Spr 2022\\
\pagebreak

\begin{homeworkProblem}
    Assume that $H \leq G, \ \lvert G:H \rvert = 4,$ 
    $L \leq G, \ \lvert L \rvert = 77,$ show that $L \leq H$. \\
    \solution 

    Let $G$ act on the set of right cosets of $H, \ \cR$ by right action
    \begin{align}
        G &\xrightarrow{\varphi} Sym \cR \cong S_4.\\
        g &\rightarrow \left( Ha \rightarrow Ha \cdot g \right)
    \end{align}
    Consider the Image of $L$ under $\varphi$. Since $L \leq G$.
    \begin{align}
        &\implies \varphi(L) \leq S_4\\
        &\implies \lvert \varphi(L) \rvert \ \text{divide} \ \lvert S_4 \rvert = 4! = 24
    \end{align}
    so for $\ell \in L$, $\varphi (\ell) \mid order(\ell)$, and
    $order(\ell) \mid \lvert L \rvert$, so $\varphi (\ell) \mid \lvert L \rvert$
    which implies $\varphi(\ell) \mid 77$.
    \begin{align}
        \frac{ \lvert L \rvert }{ \lvert \restr{ker \varphi}{L} \rvert } = \lvert \text{Im of L under} \varphi \rvert
    \end{align}
    which implies 
    \begin{align}
        \varphi(L) &\mid \lvert L \rvert \\
        \implies \varphi(L) &\mid 77 \\
        \implies \lvert \varphi(\ell) \rvert &\mid gcd(77,24)=1\\
        \implies L &\in ker \varphi\\
        \implies L \in ker \varphi &= \cap_{g \in G} gHg^{-1} \leq H\\
        \implies L &\leq H
    \end{align}
    

\end{homeworkProblem}

\pagebreak


\begin{homeworkProblem}
    Let $\lvert G \rvert = 3 \cdot 5 \cdot 59$. Prove:
    \begin{enumerate}
        \item $\lvert Z(G) \rvert > 1$.
        \item $G$ is abelian.
        \item $G$ is cyclic.
    \end{enumerate}
    \solution 

    \part
    Consider Sylow 59 subgroup
    \begin{align}
        &n_{59} \equiv 1 \mod 59\\
        &n_{59} \mid 3 \cdot 5 
    \end{align}
    which implies $n_{59} = 1$.\\
    Hence $\exists$ unique Sylow 59 subgroup we call $Q$,\\
    $\implies Q \triangleleft G$, consider $G$ act on $Q$ by conjugation.
    \begin{align}
        G &\xrightarrow[]{\varphi} Aut (Q) \ (q \in Q)\\
        g &\xrightarrow[]{} \ (q \xrightarrow[]{} gqg^{-1})
    \end{align}

    WTS: $Q \leq Z(G)$\\
    WTS: $ker \varphi = G$
    Let $g \in G \ \varphi(g) \mid Aut(Q) = 58$ 
    since $|Q| = 59$ is prime, $Q$ is cyclic. Also,
    \begin{align}
        &\frac{ \lvert G \rvert }{ \lvert ker \varphi \rvert } = \lvert Im \varphi \rvert\\
        &\varphi(g) \ \text{divides} \ \lvert Im \varphi \rvert 
        \ \text{which divides} \ \lvert G \rvert \\
        &\implies \varphi(g) \mid \lvert G \rvert = 3 \cdot 5 \cdot 59
    \end{align}
    
    
    
    

    

\end{homeworkProblem}

\pagebreak

\begin{homeworkProblem}
    Let $\lvert G \rvert = 2 \cdot 5 \cdot 7 \cdot 79^3$.
    Show that $G$ is solvable.\\
    \solution


\end{homeworkProblem}

\pagebreak

\begin{homeworkProblem}
    Let a solvable group $G$ act faithfully and
    transitively on the set $\Omega$, where $\lvert \Omega \rvert = 35$.
    \begin{enumerate}
        \item Prove that this action is not primitive.
        \item Show that if $G$ is abelian then it must be cyclic.
    \end{enumerate}
    \solution

    \part

    $G$ has a subnormal chain with abelian factor
    \begin{align}
        1 &= G_n \lhd \ldots \lhd G_1 \lhd G_0 = G\\  
        G &\xrightarrow[]{\varphi} \ Sym(\Omega) \cong S_{35}
    \end{align}
    $ker \varphi = 1$ since $G$ is faithful,
    \begin{align}
        G \cong \frac{ G }{ ker \varphi } \cong Im \varphi \leq S_{35}
    \end{align}
    We have $G \leq S_{35}, \ G \hookrightarrow S_{35}, \ \lvert G \rvert \leq 35!$
    $G$ finite\\
    $G$ transitive on $\Omega$:\\
    Primitive $\iff G_y <_{max} G, \ y \in \Omega$ 
    Try to show $G_y$ is not max subgroup in $G, y \in \Omega$?\\

    Transitive, Orbit Stabilizer Theorem $\implies \lvert \Omega \rvert$ divide
    $\lvert G \rvert$, so $35 \mid \lvert G \rvert$, now
    \begin{align}
        \lvert \text{orbit of} \ y\rvert \cdot \lvert G_y \rvert &= \lvert G \rvert\\
        \lvert \Omega \rvert &= 5 \cdot 7
    \end{align}
    there exist minimal normal subgroup $N \lhd G$, 
    \textit{(comment from instructor) suppose: primitive, faithful, solvable, 
        $\implies$ $\lvert N \rvert$ is a prime power, 
    but $\lvert N \rvert = 35$, contradiction}\\

    $\exists$ blocks,
    \begin{enumerate}
        \item $7$ blocks $\lvert B_i \rvert = 5$
        \item $5$ blocks $\lvert B_i \rvert = 7$
    \end{enumerate}
    \begin{align}
        G \xrightarrow[]{} Sym(B_i) \cong S_5
    \end{align}
    $G$ acts transitively on $B_i$'s,
    O-S implies $5 \mid \lvert G \rvert $\\

    $G$ abelian, transitive, faithful on $\Omega$ implies 
    $G$ acts regularly on $\Omega$.\\
    $Stab_x = 1 \ \forall \ x \in \Omega$.\\
    $G$ transitive, $G \xrightarrow[]{} Sym (\Omega)$,
    \begin{align}
        1 &= \lvert \# \ \text{of orbits} \rvert = \frac{ 1 }{ \lvert G \rvert } 
        \sum_{g \in G} \lvert \text{fixed elts by} \ g \rvert \\
        \lvert G \rvert &= \sum_{g \in G} \lvert \text{fixed elts by} \ g\rvert
    \end{align}
    only identity can fix some elts, hence all $\Omega$.\\
    $e \neq g$ cannot fix any element $\implies$ since action is regular.\\
    Hence $\lvert G \rvert = 35 + 0$, so $\lvert G \rvert = 35$.\\
    Since $G$ is abelian, use fundamental theorem for abelian group
    \begin{align}
        G \cong \mathbb{Z}_5 \times \mathbb{Z}_7 \cong \mathbb{Z}_{35}
    \end{align}
    is cyclic, since 5,7 coprime.\\
    


    
    
    

    
    


    
    


    
\end{homeworkProblem}

\pagebreak

\begin{homeworkProblem}
    Let $G$ be a finite group, $e$ the identity of $G$
    and $a$ some nonidentity element. Suppose that $\chi$
    is a character of $G$ such that $(\forall g \in G)$
    \begin{align}
        \chi (g) = 
        \begin{cases}
            5, \ \text{if} \ g=e,\\
            3, \ \text{if} \ g = a,\\
            0, \ \text{otherwise}
        \end{cases}
    \end{align}
    Determine $G$.\\
    \solution

    We have $\lvert G \rvert < \infty$, and
    \begin{align}
        [\chi, \chi] &= \frac{ 1 }{ \lvert G \rvert } \sum_{g \in G} 
        \chi(g) \overline{\chi(g)}\\
        &= \frac{ 1 }{ \lvert G \rvert }(25+9+0)\\
        &= \frac{ 34 }{ \lvert G \rvert }
    \end{align}
    is an integer. so $\lvert G \rvert \mid 34 = 2 \cdot 17$.\\
    \begin{align}
        [\chi, \mathbb{1}_G] &= \frac{ 1 }{ \lvert G \rvert }(5+3)
        = \frac{ 8 }{ \lvert G \rvert } \ \text{is integer}\\
        \implies \lvert G \rvert &\mid 8\\
        \implies \lvert G \rvert &\mid gcd(8, 2 \cdot 17) = 2
    \end{align}
    implies $G$ is the unique group of $2$ elements,
    the cyclic group of $2$ elements.

    
    


    
    
\end{homeworkProblem}

\end{document}
