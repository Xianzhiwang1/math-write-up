\documentclass{article}

%
% Homework Details
%   - Title
%   - Due date
%   - Class
%   - Section/Time
%   - Instructor
%   - Author
%

\newcommand{\hmwkTitle}{AAA \ \#06}
\newcommand{\hmwkDueDate}{2022}
\newcommand{\hmwkClass}{AAA}
\newcommand{\hmwkClassTime}{Spr 2022, Free Graph}
\newcommand{\hmwkClassInstructor}{Prof Peter Hermann}
\newcommand{\hmwkAuthorName}{\textbf{Xianzhi} }
\newcommand{\hmwkAuthor}{\textit{Xianzhi Wang}}


\usepackage{fancyhdr}
\usepackage{extramarks}
\usepackage{amsmath}
\usepackage{amsthm}
\usepackage{amsfonts}
\usepackage{tikz}
\usepackage[plain]{algorithm}
\usepackage{algpseudocode}

\usetikzlibrary{automata,positioning}

%
% Basic Document Settings
%

\topmargin=-0.45in
\evensidemargin=0in
\oddsidemargin=0in
\textwidth=6.5in
\textheight=9.0in
\headsep=0.25in

\linespread{1.1}

\pagestyle{fancy}
\lhead{\hmwkAuthorName}
\chead{\hmwkClass\ (\hmwkClassInstructor\ \hmwkClassTime): \hmwkTitle}
\rhead{\firstxmark}
\lfoot{\lastxmark}
\cfoot{\thepage}

\renewcommand\headrulewidth{0.4pt}
\renewcommand\footrulewidth{0.4pt}

\setlength\parindent{0pt}

%
% Create Problem Sections
%

\newcommand{\enterProblemHeader}[1]{
    \nobreak\extramarks{}{Problem \arabic{#1} continued on next page\ldots}\nobreak{}
    \nobreak\extramarks{Problem \arabic{#1} (continued)}{Problem \arabic{#1} continued on next page\ldots}\nobreak{}
}

\newcommand{\exitProblemHeader}[1]{
    \nobreak\extramarks{Problem \arabic{#1} (continued)}{Problem \arabic{#1} continued on next page\ldots}\nobreak{}
    \stepcounter{#1}
    \nobreak\extramarks{Problem \arabic{#1}}{}\nobreak{}
}

\setcounter{secnumdepth}{0}
\newcounter{partCounter}
\newcounter{homeworkProblemCounter}
\setcounter{homeworkProblemCounter}{1}
\nobreak\extramarks{Problem \arabic{homeworkProblemCounter}}{}\nobreak{}

%
% Homework Problem Environment
%
% This environment takes an optional argument. When given, it will adjust the
% problem counter. This is useful for when the problems given for your
% assignment aren't sequential. See the last 3 problems of this template for an
% example.
%
\newenvironment{homeworkProblem}[1][-1]{
    \ifnum#1>0
        \setcounter{homeworkProblemCounter}{#1}
    \fi
    \section{Problem \arabic{homeworkProblemCounter}}
    \setcounter{partCounter}{1}
    \enterProblemHeader{homeworkProblemCounter}
}{
    \exitProblemHeader{homeworkProblemCounter}
}


%
% Title Page
%

\title{
    \vspace{2in}
    \textmd{\textbf{\hmwkClass:\ \hmwkTitle}}\\
    \normalsize\vspace{0.1in}\small{Due\ on\ \hmwkDueDate\ at 11:59PM}\\
    \vspace{0.1in}\large{\textit{\hmwkClassInstructor\ \hmwkClassTime}}
    \vspace{3in}
}

\author{\hmwkAuthorName}
\date{2023}

\renewcommand{\part}[1]{\textbf{\large Part \Alph{partCounter}}\stepcounter{partCounter}\\}

%
% Various Helper Commands
%

% Useful for algorithms
\newcommand{\alg}[1]{\textsc{\bfseries \footnotesize #1}}

% For derivatives
\newcommand{\deriv}[1]{\frac{\mathrm{d}}{\mathrm{d}x} (#1)}

% For partial derivatives
\newcommand{\pderiv}[2]{\frac{\partial}{\partial #1} (#2)}

% Integral dx
\newcommand{\dx}{\mathrm{d}x}

% Alias for the Solution section header
\newcommand{\solution}{\textbf{\large Solution:}}

% Probability commands: Expectation, Variance, Covariance, Bias
\newcommand{\E}{\mathrm{E}}
\newcommand{\Var}{\mathrm{Var}}
\newcommand{\Cov}{\mathrm{Cov}}
\newcommand{\Bias}{\mathrm{Bias}}

% From xianzhi.sty

% Fancy
\newcommand{\cA}{\mathcal A}
\newcommand{\cB}{\mathcal B}
\newcommand{\cC}{\mathcal C}
\newcommand{\cD}{\mathcal D}
\newcommand{\cE}{\mathcal E}
\newcommand{\cF}{\mathcal F}
\newcommand{\cG}{\mathcal G}
\newcommand{\cH}{\mathcal H}
\newcommand{\cI}{\mathcal I}
\newcommand{\cJ}{\mathcal J}
\newcommand{\cL}{\mathcal L}
\newcommand{\cM}{\mathcal M}
\newcommand{\cN}{\mathcal N}
\newcommand{\cO}{\mathcal O}
\newcommand{\cP}{\mathcal P}
\newcommand{\cR}{\mathcal R}
\newcommand{\cS}{\mathcal S}
\newcommand{\cT}{\mathcal T}
\newcommand{\cU}{\mathcal U}
\newcommand{\cW}{\mathcal W}
\newcommand{\cX}{\mathcal X}
\newcommand{\cY}{\mathcal Y}


\newcommand{\bP}{\mathbb{P}}

\newcommand{\C}{\mathbb{C}}
\newcommand{\R}{\mathbb{R}}
\newcommand{\N}{\mathbb{N}}
\newcommand{\Q}{\mathbb{Q}}
\newcommand{\Z}{\mathbb{Z}}
\newcommand{\F}{\mathbb{F}}

% Brackets
\newcommand{\bigp}[1]{\left( #1 \right)} % (x)
\newcommand{\bigb}[1]{\left[ #1 \right]} % [x]
\newcommand{\bigc}[1]{\left\{ #1 \right\}} % {x}
\newcommand{\biga}[1]{\left\langle #1 \right\rangle} % <x>

%norm

% theorem 
\newtheorem{Proposition}{proposition}
\newtheorem{Assumption}{assumption}
\newtheorem{Definition}{definition}
\newtheorem{Corollary}{corollary}
\newtheorem{Question}{question}



% \begin{document}

% \usepackage{xianzhi}

\begin{document}

\maketitle
HW06 \\
\pagebreak

\begin{homeworkProblem}
    Without using any reference to the Cayley Graphs prove that 
    free groups are torsion free.

    \solution

    prove free groups are torsion free.\\
    Let $W \in F(X)$ be a non-identity reduced word,
    so $W = X_{i_1}^{k_1} \cdots X_{i_m}^{k_m}$
    We observe 
    \begin{align}
        \left( X_{i_1}^{k_1} \cdots X_{i_m}^{k_m} \right) \cdot \left( X_{i_1}^{k_1} \cdots X_{i_m}^{k_m} \right) \left( \ldots \right) \ldots
    \end{align}
    The potential place cancelation can happen is between 
    \begin{align}
        X_{i_m}^{k_m} X_{i_1}^{k_1},
    \end{align}
    If there is no cancelation here, then it's clear we 
    raise $W^n$ to the power $n$ for all $n \in \mathbb{N}$
    will not get the empty word, so $W$ has infinite order.\\
    On the other hand, if there  is cancelation between 
    \begin{align}
        X_{i_m}^{k_m} \ X_{i_1}^{k_1},
    \end{align}
    we can write $W = a^{-1} b a$ where there is 
    no cancelation between $a, b$ or 
    $a^{-1}, b$ or $b$ and $b$. We know
    we can write $W$ in this way since 
    $W$ is assumed not to be the identity.\\
    If 
    \begin{align}
        \left( X_{i_1}^{k_1} \cdots X_{i_m}^{k_m} \right) \cdot \left( X_{i_1}^{k_1} \cdots X_{i_m}^{k_m} \right) = 1
    \end{align}
    then it is clear that at some point during cancelation, we get
    \begin{align}
        \left( X_{i_1}^{k_1} \cdots X_{i_{\ell}}^{k_{\ell}} \right) \left( X_{i_{\ell'}}^{k_{\ell'}} \cdots X_{i_m}^{k_m}\right)
    \end{align}
    where $1 \leq \ell, \ell' \leq m$, then the fact we can keep on performing 
    cancelation means $W$ is of the form $a^{-1}a$, so 
    $a^{-1} a a^{-1} a = 1$, so $W$ is not reduced. Contradiction.\\
    $W$ is also identity, which is another contradiction.\\

    Thus, $W = a^{-1} b a$, and $W^n = a^{-1} b^n a$, 
    and since there is no cancelation between $b$ and $b$. 
    $W$ has infinite order. 
    

\end{homeworkProblem}

\pagebreak


\begin{homeworkProblem}
    Let $X \subset G$ and suppose $G = \langle X \rangle$. 
    Denote the corresponding (directed) Cayley graph by $\Gamma = \Gamma(G; X)$
    such that $V(\Gamma) = G,\ E(\Gamma) = \cup_{x \in X} E_x (\Gamma)$,
    where for each $x \in X$, we have 
    $E_x(\Gamma) = \left\{ (a, xa) | a \in G \right\}$
    (the set of all directed edges having color $x$).
    A bijective mapping $\alpha: V(\Gamma) \mapsto V(\Gamma)$ 
    is called an \textit{automorphism} of $\Gamma$ provided
    \begin{align}
        &\forall a, b \in V(\Gamma) \ \text{and} \ x \in X\\
        &\bar{(a,b)} \in E_x (\Gamma) \Leftrightarrow \bar{(a^{\alpha}, b^{\alpha})} \in E_x(\Gamma).
    \end{align}
    Prove that 
    \begin{align}
        Aut \Gamma = G^* = \left\{ g^* \mid g \in G \right\},
    \end{align}
    where $g^*$ denotes multiplication by the element $g$ on the right.

    \solution

    Since a bijective mapping $\alpha: V(\Gamma) \rightarrow V(\Gamma)$ is automorphism 
    if it preserve edge, direction and label:\\
    $\forall a,b \in V(\Gamma), x \in X,$ then 
    \begin{align}
        (a,b) \in E_x(\Gamma) \Leftrightarrow (a^{\alpha}, b^{\alpha}) \in E_x (\Gamma)
    \end{align}
    so multiplication by element $G$ on the right clearly satisfy this,
    let $(a, xa) \in E_x(\Gamma) = \left\{ (a, xa) | a \in G \right\}$
    \begin{align}
        (a, xa) \in E_x(\Gamma) \Leftrightarrow (ag, x(ag)) = (a \cdot g, (xa) \cdot g) \in E_x(\Gamma)
    \end{align}
    so $Aut \Gamma \supset G^{*}$.\\
    Now let $\phi \in Aut \Gamma$, we have (since $\phi$ preserve this for all $a,b \in V(\Gamma)$)
    \begin{align}
        (1, X \cdot 1) \in E_x (\Gamma) \Leftrightarrow (\phi(1), \phi(x \cdot 1)) \in E_x(\Gamma)
    \end{align}
    since $E_x(\Gamma) = \left\{ (a, xa) | a \in G\right\} $
    could only be $(\phi(1), x \cdot \phi(1)) \in E_x(\Gamma)$
    Thus, for all generators $\phi(x) = x \cdot \phi(1)$, and we are right multiplying by 
    $g = \phi(1)$. so for arbitrary $a \in V(\Gamma)$,
    \begin{align}
        (a, xa) \in E_x (\Gamma) &\Leftrightarrow (\phi(a), \phi(x \cdot a)) \in E_x (\Gamma)\\
        &= (\phi(a), x \phi(a)) \in E_x (\Gamma)
    \end{align}
    similarly $\phi (x^{-1}) = x^{-1} \phi(1)$, for all generators, 
    so since $a = X_{i_1}^{k_1} \cdots X_{i_m}^{k_m}$,
    we could perform this finitely many times to have 
    $\phi(a) = X_{i_1}^{k_1} \cdots X_{i_m}^{k_m} \phi(1)$ so
    \begin{align}
        (\phi(a), \phi(x \cdot a)) 
        = (X_{i_1}^{k_1} \cdots X_{i_m}^{k_m} \phi(1), X \cdot X_{i_1}^{k_1} \cdots X_{i_m}^{k_m} \phi(1)) 
        \in E_x(\Gamma)
    \end{align}
    Thus, $\phi$ is indeed right multiply by element $g = \phi(1)$.\\

    
    
    
    

    

\end{homeworkProblem}

\pagebreak

\begin{homeworkProblem}
    Let $F(X)$ denote the free group with $X$ being the set of free generators.
    \begin{enumerate}
        \item Determine the center of $F(X)$ when $\lvert X \rvert > 1$.
        \item What is $Z(F(X))$ when $\lvert X \rvert = 1$ ?
    \end{enumerate}
    

    \part

    center of $F(X)$ when $\lvert X \rvert > 1$.
    We know $1 \in Z(F(X))$. Now assume 
    $z \neq  1$, and $z \in Z(F(X))$.
    so $ZW = WZ$ (assume $Z$ and $W$ are reduced.) so
    \begin{align}
        X^{k_1}_{i_1} \cdots X_{i_m}^{k_m} \cdot X_{j_1}^{\ell_1} \cdots X_{j_m}^{\ell_m} 
        &= X_{j_1}^{\ell_1} \cdots X_{j_m}^{\ell_m} \cdots X_{i_1}^{k_1} \cdots X_{i_m}^{k_m}\\
        ZW &= WZ\\
    \end{align}
    since $W$ can be any word in $F(X)$, it is possible we 
    pick an $W$ such that there is no cancelation between 
    \begin{align}
        X_{i_m}^{k_m} \cdot X_{j_1}^{\ell_1}
    \end{align}
    and between 
    \begin{align}
        X_{j_m}^{\ell_m} \cdot X_{i_1}^{k_1}
    \end{align}
    so $ZW$ and $WZ$ are clearly different words.
    So cannot equal.
    \textit{It is somewhat simpler to choose $w:= x_i (\forall i),$ etc}
    so $Z(F(X)) = \left\{ 1\right\}$.\\

    \part

    $Z(F(X)) = F(X)$ when $\lvert X \rvert$, so $F(X)$ is 
    generated by a single generator $X$, 
    Thus, fix any $Z \in F(X)$.
    $Z = X^k$ then for any other element $W = X^{\ell} \in F(X)$
    (assume reduced),
    \begin{align}
        Z \cdot W = X^k \cdot X^{\ell}
        = X^{k + \ell} 
        = X^{\ell + k} 
        =X^{\ell} \cdot X^k
        = W \cdot Z
    \end{align}
    so $Z \in Z(F(X))$,\\
    so $F(X) \subset Z(F(X))$\\
    since $Z(F(X)) \subset F(X)$\\
    $F(x) = Z(F(X))$.\\

    
    
    


\end{homeworkProblem}

\pagebreak

\begin{homeworkProblem}
    \textbf{For Fun:} Let $F_n = \langle x_1, x_2, \ldots, x_n \rangle$
    denote the free group of rank $n$, i.e.,
    generated by $n$ free generators. 
    Find $a,b,c \in F_2$ such that
    \begin{align}
        \lvert F_2 : \langle a, b, c \rangle \rvert = 2
    \end{align}

    \solution

    Find $a,b,c \in F_2 $ such that $\lvert F_2: \left\langle a,b,c \right\rangle \rvert = 2$\\
    \begin{align}
        f: x &\xrightarrow[]{} 0 + 2 \mathbb{Z}\\
        y &\xrightarrow[]{} 1 + 2 \mathbb{Z}
    \end{align}
    $f$ is a map that 
    \begin{align}
        x &\xrightarrow[]{} 0 + 2 \mathbb{Z}\\
        y &\xrightarrow[]{} 1 + 2 \mathbb{Z}
    \end{align}
    $F_2 = \langle x, y \rangle$ is the free group
    generated by $x,y$.
    Universal property gives a surjective homomorphism $\varphi$,
    and by 1st iso thm,
    \begin{align}
        &F_2 / (ker \varphi) \cong \mathbb{Z} / 2 \mathbb{Z} \cong \mathbb{Z}_2\\
        &\implies \lvert F_2 : ker \varphi \rvert = 2
    \end{align}
    claim: 
    \begin{align}
        ker \varphi &= \left\{ x^{k_1} y^{\ell_1} x^{k_2} y^{\ell_2} \cdots x^{k_m} y^{\ell_m} \mid \sum_i \ell_i = even \right\}\\
        &= \left\langle x, y^2, yxy \right\rangle
    \end{align}
    $x \in ker \varphi$ since $\varphi (x) = f(x) = 0 + 2 \mathbb{Z} \implies x \in ker \varphi$ \\
    $ y^2 \in ker \varphi$ since 
    $\varphi (y^2 ) = \varphi(y) \varphi(y) = f(y)f(y) = (1+2 \mathbb{Z}) + (1+ 2 \mathbb{Z}) = 0 + 2 \mathbb{Z}$\\
    $y x y \in ker \varphi$ similarly.\\


    
    
    
    

\end{homeworkProblem}

\end{document}
