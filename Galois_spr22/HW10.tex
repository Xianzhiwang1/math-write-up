\documentclass{article}

%
% Homework Details
%   - Title
%   - Due date
%   - Class
%   - Section/Time
%   - Instructor
%   - Author
%

\newcommand{\hmwkTitle}{GAL \ \#10}
\newcommand{\hmwkDueDate}{May 6, 2022}
\newcommand{\hmwkClass}{Galois Theory}
\newcommand{\hmwkClassTime}{Section 12 \& 15}
\newcommand{\hmwkClassInstructor}{Prof Matyas Domokos}
\newcommand{\hmwkAuthorName}{\textbf{Xianzhi} }
\newcommand{\hmwkAuthor}{\textit{Xianzhi Wang}}


\usepackage{fancyhdr}
\usepackage{extramarks}
\usepackage{amsmath}
\usepackage{amsthm}
\usepackage{amsfonts}
\usepackage{tikz}
\usepackage[plain]{algorithm}
\usepackage{algpseudocode}

\usetikzlibrary{automata,positioning}

%
% Basic Document Settings
%

\topmargin=-0.45in
\evensidemargin=0in
\oddsidemargin=0in
\textwidth=6.5in
\textheight=9.0in
\headsep=0.25in

\linespread{1.1}

\pagestyle{fancy}
\lhead{\hmwkAuthorName}
\chead{\hmwkClass\ (\hmwkClassInstructor\ \hmwkClassTime): \hmwkTitle}
\rhead{\firstxmark}
\lfoot{\lastxmark}
\cfoot{\thepage}

\renewcommand\headrulewidth{0.4pt}
\renewcommand\footrulewidth{0.4pt}

\setlength\parindent{0pt}

%
% Create Problem Sections
%

\newcommand{\enterProblemHeader}[1]{
    \nobreak\extramarks{}{Problem \arabic{#1} continued on next page\ldots}\nobreak{}
    \nobreak\extramarks{Problem \arabic{#1} (continued)}{Problem \arabic{#1} continued on next page\ldots}\nobreak{}
}

\newcommand{\exitProblemHeader}[1]{
    \nobreak\extramarks{Problem \arabic{#1} (continued)}{Problem \arabic{#1} continued on next page\ldots}\nobreak{}
    \stepcounter{#1}
    \nobreak\extramarks{Problem \arabic{#1}}{}\nobreak{}
}

\setcounter{secnumdepth}{0}
\newcounter{partCounter}
\newcounter{homeworkProblemCounter}
\setcounter{homeworkProblemCounter}{1}
\nobreak\extramarks{Problem \arabic{homeworkProblemCounter}}{}\nobreak{}

%
% Homework Problem Environment
%
% This environment takes an optional argument. When given, it will adjust the
% problem counter. This is useful for when the problems given for your
% assignment aren't sequential. See the last 3 problems of this template for an
% example.
%
\newenvironment{homeworkProblem}[1][-1]{
    \ifnum#1>0
        \setcounter{homeworkProblemCounter}{#1}
    \fi
    \section{Problem \arabic{homeworkProblemCounter}}
    \setcounter{partCounter}{1}
    \enterProblemHeader{homeworkProblemCounter}
}{
    \exitProblemHeader{homeworkProblemCounter}
}


%
% Title Page
%

\title{
    \vspace{2in}
    \textmd{\textbf{\hmwkClass:\ \hmwkTitle}}\\
    \normalsize\vspace{0.1in}\small{Due\ on\ \hmwkDueDate\ at 11:59PM}\\
    \vspace{0.1in}\large{\textit{\hmwkClassInstructor\ \hmwkClassTime}}
    \vspace{3in}
}

\author{\hmwkAuthorName}
\date{2023}

\renewcommand{\part}[1]{\textbf{\large Part \Alph{partCounter}}\stepcounter{partCounter}\\}

%
% Various Helper Commands
%

% Useful for algorithms
\newcommand{\alg}[1]{\textsc{\bfseries \footnotesize #1}}

% For derivatives
\newcommand{\deriv}[1]{\frac{\mathrm{d}}{\mathrm{d}x} (#1)}

% For partial derivatives
\newcommand{\pderiv}[2]{\frac{\partial}{\partial #1} (#2)}

% Integral dx
\newcommand{\dx}{\mathrm{d}x}

% Alias for the Solution section header
\newcommand{\solution}{\textbf{\large Solution:}}

% Probability commands: Expectation, Variance, Covariance, Bias
\newcommand{\E}{\mathrm{E}}
\newcommand{\Var}{\mathrm{Var}}
\newcommand{\Cov}{\mathrm{Cov}}
\newcommand{\Bias}{\mathrm{Bias}}

% From xianzhi.sty

% Fancy
\newcommand{\cA}{\mathcal A}
\newcommand{\cB}{\mathcal B}
\newcommand{\cC}{\mathcal C}
\newcommand{\cD}{\mathcal D}
\newcommand{\cE}{\mathcal E}
\newcommand{\cF}{\mathcal F}
\newcommand{\cG}{\mathcal G}
\newcommand{\cH}{\mathcal H}
\newcommand{\cI}{\mathcal I}
\newcommand{\cJ}{\mathcal J}
\newcommand{\cL}{\mathcal L}
\newcommand{\cM}{\mathcal M}
\newcommand{\cN}{\mathcal N}
\newcommand{\cO}{\mathcal O}
\newcommand{\cP}{\mathcal P}
\newcommand{\cR}{\mathcal R}
\newcommand{\cS}{\mathcal S}
\newcommand{\cT}{\mathcal T}
\newcommand{\cU}{\mathcal U}
\newcommand{\cW}{\mathcal W}
\newcommand{\cX}{\mathcal X}
\newcommand{\cY}{\mathcal Y}


\newcommand{\bP}{\mathbb{P}}

\newcommand{\C}{\mathbb{C}}
\newcommand{\R}{\mathbb{R}}
\newcommand{\N}{\mathbb{N}}
\newcommand{\Q}{\mathbb{Q}}
\newcommand{\Z}{\mathbb{Z}}
\newcommand{\F}{\mathbb{F}}

% Brackets
\newcommand{\bigp}[1]{\left( #1 \right)} % (x)
\newcommand{\bigb}[1]{\left[ #1 \right]} % [x]
\newcommand{\bigc}[1]{\left\{ #1 \right\}} % {x}
\newcommand{\biga}[1]{\left\langle #1 \right\rangle} % <x>

%norm

% theorem 
\newtheorem{Proposition}{proposition}
\newtheorem{Assumption}{assumption}
\newtheorem{Definition}{definition}
\newtheorem{Corollary}{corollary}
\newtheorem{Question}{question}



% \begin{document}

% \usepackage{xianzhi}

\begin{document}

\maketitle
HW10 \\
Exercise 12.4.12\\
Exercise 12.4.13\\
Exercise 15.1.2\\
\pagebreak

\begin{homeworkProblem}
    \textbf{Exercise 12.4.12} Prove that $X^4 - 10 X^2 + 1$ is irreducible over $\Q$, but it is reducible in $(\Z/p \Z)[X]$ for any prime $p$.\\
    \solution 

    \part 

    \begin{proof}
        We claim the minimum polynomial is $M_{\Q}(\sqrt{2} + \sqrt{3}) = X^4 - 10 X^2 + 1$.
        Observe that
        \begin{align}
            &(\sqrt{2} + \sqrt{3})^4 - 10 (\sqrt{2} + \sqrt{3})^2 + 1 \\
            =& (5 + 2 \sqrt{6})^2 - 10 (5 + 2 \sqrt{6}) + 1 \\
            =& 0
        \end{align}
        Thus, $(\sqrt{2} + \sqrt{3})$ is a root of $X^4 - 10 X^2 + 1$.
        Claim: $\Q (\sqrt{2} + \sqrt{3}) = \Q (\sqrt{2} , \sqrt{3})$, and it is a degree $4$ extension over $\Q$, so
        $X^4 - 10 X^2 + 1$ is the minimal polynomial over $\Q$, hence irreducible.

        Now we show the claim. $\Q(\sqrt{2} + \sqrt{3}) \supset \Q(\sqrt{2}, \sqrt{3})$ because 
        \begin{align}
            5 + 2 \sqrt{6} &= (\sqrt{2} + \sqrt{3})^2 \in \Q(\sqrt{2} + \sqrt{3})\\
            &\implies \sqrt{6}(\sqrt{2} + \sqrt{3}) \in  \Q(\sqrt{2} + \sqrt{3})\\
            &\implies \sqrt{2} = \sqrt{6}(\sqrt{2}+\sqrt{3}) - 2(\sqrt{2}+\sqrt{3}) \in \Q(\sqrt{2} + \sqrt{3})\\
            &\implies \sqrt{3} = \sqrt{2} + \sqrt{3} - \sqrt{2} \in \Q(\sqrt{2} + \sqrt{3})\\
        \end{align}
        $\Q(\sqrt{2} + \sqrt{3}) \subset \Q(\sqrt{2}, \sqrt{3})$ because $\sqrt{2} + \sqrt{3} \in \Q(\sqrt{2}, \sqrt{3})$. 
        Hence, we showed $X^4 - 10X^2 + 1$ is irreducible over $\Q$.\\
    \end{proof}

    \part

    \begin{proof}
        Now, observe
        \begin{align}
            X^4 - 10 X^2 + 1 &= (X^2 - 5)^2 - 2^2 \cdot 6\\
            &= (X^2 -1)^2 - (2X)^2 \cdot 2\\
            &= (X^2 + 1)^2 - (2X)^2 \cdot 3
        \end{align}
        Thus, in $(\Z / p\Z)$, as long as at least one of $6,2,3$ is a square, then $X^4 - 10 X^2 +1$ factors in $\Z/ p \Z [X]$ use formula $a^2 - b^2 = (a+b)(a-b)$.\\
        For any prime $p$, $\Z/ p\Z^{\times} = \F_p^{\times} = \F_p \setminus \{0\}$ is cyclic,
        (multiplicative group of any finite field is cyclic), $\exists$ generator $g$, thus
        \begin{align}
            \{1, g, g^2, g^3, g^4, \cdots, g^{p-2} \} = \F_p^{\times}
        \end{align}
        Those with even power are squares.\\
        Assume for contradiction that all $2,3,6$ are not squares in $\F_p$.\\
        $\implies \ 2 = g^j, \ 3 = g^i$ for some $j,i$ odd.\\
        But $6 = g^{j+i}$ has even power $j+i$, so $6$ should be square in $\F_p$. We have a contradiction.\\
        Thus, $\exists$ at least $1$ square among $2,3,6$. And we are done.
    \end{proof}
    



\end{homeworkProblem}

\pagebreak


\begin{homeworkProblem}
    \textbf{Exercise 12.4.13} Let $K$ be a field of characteristic $p$ (where $p$ is a prime), and suppose 
    that $f = X^p - X -a \in K[X]$ is irreducible. Show that $f$ is separable,
    and determine the Galois group of $f$. 
    \textit{Warning: $K$ is not assumed to be finite.)}\\
    \solution 

    \begin{proof}
        
    \end{proof}
    




\end{homeworkProblem}

\pagebreak

\begin{homeworkProblem}
    \textbf{Exercise 15.1.2} Let $p$ be a prime and $n$ a positive integer. 
    For $d \in \N$ denote by $\overline{\Phi}_d \in (\Z / p \Z)[X]$ the modulo $p$ reduction of the cyclotomic polynomial $\Phi_d \in \Z [X]$.
    Show that the splitting field of $\overline{\Phi}_{p^n -1}$ over $\F_p = \Z/ p \Z$ is the field $\F_{p^n}$.\\
    \solution \\




\end{homeworkProblem}

\end{document}
