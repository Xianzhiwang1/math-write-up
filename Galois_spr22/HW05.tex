\documentclass{article}

%
% Homework Details
%   - Title
%   - Due date
%   - Class
%   - Section/Time
%   - Instructor
%   - Author
%

\newcommand{\hmwkTitle}{GAL \ \#05}
\newcommand{\hmwkDueDate}{Mar 18, 2022}
\newcommand{\hmwkClass}{Galois Theory}
\newcommand{\hmwkClassTime}{Section 7}
\newcommand{\hmwkClassInstructor}{Prof Matyas Domokos}
\newcommand{\hmwkAuthorName}{\textbf{Xianzhi} }
\newcommand{\hmwkAuthor}{\textit{Xianzhi Wang}}


\usepackage{fancyhdr}
\usepackage{extramarks}
\usepackage{amsmath}
\usepackage{amsthm}
\usepackage{amsfonts}
\usepackage{tikz}
\usepackage[plain]{algorithm}
\usepackage{algpseudocode}

\usetikzlibrary{automata,positioning}

%
% Basic Document Settings
%

\topmargin=-0.45in
\evensidemargin=0in
\oddsidemargin=0in
\textwidth=6.5in
\textheight=9.0in
\headsep=0.25in

\linespread{1.1}

\pagestyle{fancy}
\lhead{\hmwkAuthorName}
\chead{\hmwkClass\ (\hmwkClassInstructor\ \hmwkClassTime): \hmwkTitle}
\rhead{\firstxmark}
\lfoot{\lastxmark}
\cfoot{\thepage}

\renewcommand\headrulewidth{0.4pt}
\renewcommand\footrulewidth{0.4pt}

\setlength\parindent{0pt}

%
% Create Problem Sections
%

\newcommand{\enterProblemHeader}[1]{
    \nobreak\extramarks{}{Problem \arabic{#1} continued on next page\ldots}\nobreak{}
    \nobreak\extramarks{Problem \arabic{#1} (continued)}{Problem \arabic{#1} continued on next page\ldots}\nobreak{}
}

\newcommand{\exitProblemHeader}[1]{
    \nobreak\extramarks{Problem \arabic{#1} (continued)}{Problem \arabic{#1} continued on next page\ldots}\nobreak{}
    \stepcounter{#1}
    \nobreak\extramarks{Problem \arabic{#1}}{}\nobreak{}
}

\setcounter{secnumdepth}{0}
\newcounter{partCounter}
\newcounter{homeworkProblemCounter}
\setcounter{homeworkProblemCounter}{1}
\nobreak\extramarks{Problem \arabic{homeworkProblemCounter}}{}\nobreak{}

%
% Homework Problem Environment
%
% This environment takes an optional argument. When given, it will adjust the
% problem counter. This is useful for when the problems given for your
% assignment aren't sequential. See the last 3 problems of this template for an
% example.
%
\newenvironment{homeworkProblem}[1][-1]{
    \ifnum#1>0
        \setcounter{homeworkProblemCounter}{#1}
    \fi
    \section{Problem \arabic{homeworkProblemCounter}}
    \setcounter{partCounter}{1}
    \enterProblemHeader{homeworkProblemCounter}
}{
    \exitProblemHeader{homeworkProblemCounter}
}


%
% Title Page
%

\title{
    \vspace{2in}
    \textmd{\textbf{\hmwkClass:\ \hmwkTitle}}\\
    \normalsize\vspace{0.1in}\small{Due\ on\ \hmwkDueDate\ at 11:59PM}\\
    \vspace{0.1in}\large{\textit{\hmwkClassInstructor\ \hmwkClassTime}}
    \vspace{3in}
}

\author{\hmwkAuthorName}
\date{2023}

\renewcommand{\part}[1]{\textbf{\large Part \Alph{partCounter}}\stepcounter{partCounter}\\}

%
% Various Helper Commands
%

% Useful for algorithms
\newcommand{\alg}[1]{\textsc{\bfseries \footnotesize #1}}

% For derivatives
\newcommand{\deriv}[1]{\frac{\mathrm{d}}{\mathrm{d}x} (#1)}

% For partial derivatives
\newcommand{\pderiv}[2]{\frac{\partial}{\partial #1} (#2)}

% Integral dx
\newcommand{\dx}{\mathrm{d}x}

% Alias for the Solution section header
\newcommand{\solution}{\textbf{\large Solution:}}

% Probability commands: Expectation, Variance, Covariance, Bias
\newcommand{\E}{\mathrm{E}}
\newcommand{\Var}{\mathrm{Var}}
\newcommand{\Cov}{\mathrm{Cov}}
\newcommand{\Bias}{\mathrm{Bias}}

% From xianzhi.sty

% Fancy
\newcommand{\cA}{\mathcal A}
\newcommand{\cB}{\mathcal B}
\newcommand{\cC}{\mathcal C}
\newcommand{\cD}{\mathcal D}
\newcommand{\cE}{\mathcal E}
\newcommand{\cF}{\mathcal F}
\newcommand{\cG}{\mathcal G}
\newcommand{\cH}{\mathcal H}
\newcommand{\cI}{\mathcal I}
\newcommand{\cJ}{\mathcal J}
\newcommand{\cL}{\mathcal L}
\newcommand{\cM}{\mathcal M}
\newcommand{\cN}{\mathcal N}
\newcommand{\cO}{\mathcal O}
\newcommand{\cP}{\mathcal P}
\newcommand{\cR}{\mathcal R}
\newcommand{\cS}{\mathcal S}
\newcommand{\cT}{\mathcal T}
\newcommand{\cU}{\mathcal U}
\newcommand{\cW}{\mathcal W}
\newcommand{\cX}{\mathcal X}
\newcommand{\cY}{\mathcal Y}


\newcommand{\bP}{\mathbb{P}}

\newcommand{\C}{\mathbb{C}}
\newcommand{\R}{\mathbb{R}}
\newcommand{\N}{\mathbb{N}}
\newcommand{\Q}{\mathbb{Q}}
\newcommand{\Z}{\mathbb{Z}}
\newcommand{\F}{\mathbb{F}}

% Brackets
\newcommand{\bigp}[1]{\left( #1 \right)} % (x)
\newcommand{\bigb}[1]{\left[ #1 \right]} % [x]
\newcommand{\bigc}[1]{\left\{ #1 \right\}} % {x}
\newcommand{\biga}[1]{\left\langle #1 \right\rangle} % <x>

%norm

% theorem 
\newtheorem{Proposition}{proposition}
\newtheorem{Assumption}{assumption}
\newtheorem{Definition}{definition}
\newtheorem{Corollary}{corollary}
\newtheorem{Question}{question}



% \begin{document}

% \usepackage{xianzhi}

\begin{document}

\maketitle
HW05 \\
Exercise 7.2.5\\
Exercise 7.2.7\\
Exercise 7.2.8\\
\pagebreak

\begin{homeworkProblem}
    \textbf{Exercise 7.2.5} Let $\gamma = \sqrt{2 + \sqrt{2}}$.
    \begin{enumerate}
        \item Show that $\Q(\gamma):\Q$ is normal with cyclic Galois Group.\\
        \item Show that $\Q(\gamma, i) = \Q(\phi)$ with $\phi^4 = i$.
    \end{enumerate}

    \solution

    \part

    Let 
    \begin{align}
        \sqrt{2 + \sqrt{2}} &= X \\
        \sqrt{2} &= X^2 - 2 \\
        2 &= (X^2 - 2)^2 = X^4 - 4 X^2 + 4
    \end{align}
    Thus, $\sqrt{2+\sqrt{2}}$ is a root of $X^4 - 4X^2 + 2 =: f$
    Since $f$ is irreducible, by Eisenstein $(p=2)$, 
    $\mathbb{Q}(\gamma)$ is degree 4 over $\mathbb{Q}$,
    we could find the roots of $f$:
    \begin{align}
        f = \left(X + \sqrt{2+\sqrt{2}}\right) \left(X - \sqrt{2+\sqrt{2}} \right) \left(X + \sqrt{2-\sqrt{2}}\right) \left(X - \sqrt{2-\sqrt{2}}\right)
    \end{align}
    Let $\phi \in \Gamma(\mathbb{Q}(\gamma):\mathbb{Q})$, we know $\phi$ permutes the roots of $f$.\\
    If we can find an element $\phi$ with order strictly greater than $2$, 
    then since order of the element need to divide the order of the group,
    $\lvert \phi \rvert$ must be 4. Since up to isomorphism, group of order 4
    is $\mathbb{Z}_4$ and $\mathbb{Z}_2 \oplus \mathbb{Z}_2$, we know once 
    there exists $\lvert \phi \rvert = 4, \ \Gamma(\mathbb{Q}(\gamma):\mathbb{Q})$ must be cyclic.\\
    First, we show $ \mathbb{Q}(\gamma) $ is splitting field of $f$
    that has no multiple roots, which would imply it is normal.
    since 
    \begin{align}
        \sqrt{2+\sqrt{2}} = \gamma \in \mathbb{Q}(\gamma) \implies - \sqrt{2+\sqrt{2}} &\in \mathbb{Q}(\gamma)\\
        2+\sqrt{2} &= \gamma^2 \in \mathbb{Q}(\gamma) 
    \end{align}
    so $\sqrt{2} \in \Q(\gamma)$. Since
    \begin{align}
        \sqrt{2+\sqrt{2}} \cdot \sqrt{2 - \sqrt{2}} = \sqrt{2} \in \mathbb{Q}(\gamma) \implies \sqrt{2-\sqrt{2}} \in \mathbb{Q}(\gamma)  
    \end{align}
    so $-\sqrt{2-\sqrt{2}} \in \mathbb{Q}(\gamma)$.\\
    Thus, all 4 roots of $f$ are in $\mathbb{Q}(\gamma)$. And 
    these roots are all distinct. so
    \begin{align}
        \mathbb{Q} \left(\sqrt{2+\sqrt{2}}, -\sqrt{2+\sqrt{2}}, +\sqrt{2-\sqrt{2}}, - \sqrt{2-\sqrt{2}}\right) &\subseteq \mathbb{Q}(\gamma)\\
        \mathbb{Q} \left(\sqrt{2+\sqrt{2}}, -\sqrt{2+\sqrt{2}}, +\sqrt{2-\sqrt{2}}, - \sqrt{2-\sqrt{2}}\right) &\supseteq \mathbb{Q}(\gamma)\\
        \text{so} \ \mathbb{Q} \left(\sqrt{2+\sqrt{2}}, -\sqrt{2+\sqrt{2}}, +\sqrt{2-\sqrt{2}}, - \sqrt{2-\sqrt{2}}\right) &= \mathbb{Q}(\gamma)\\
    \end{align}
    Thus, $\mathbb{Q}(\gamma)$ is indeed splitting field of $f$ with no multiple roots,
    so $\mathbb{Q}(\gamma): \mathbb{Q}$ is Galois extension, 
    and is normal extension. 
    Now we find $\phi \in \Gamma(\mathbb{Q}(\gamma):\mathbb{Q})$ that has order $\geq 3$. Claim:
    \begin{align}
        \sqrt{2 + \sqrt{2}} \mapsto_{\phi} \sqrt{2 - \sqrt{2}}  
    \end{align}
    does the job.\\
    \textit{Why does such an automorphism exist?
    Answer: Galois group acts transitively on the roots of a minimal polynomial}



    Let 
    \begin{align}
    X_1 = - \sqrt{2 + \sqrt{2}}, \ X_2 = \sqrt{2 + \sqrt{2}}, \ X_3 = - \sqrt{2 - \sqrt{2}}, \ X_4 = \sqrt{2 - \sqrt{2}}
    \end{align}
    Hence 
    \begin{align}
        \phi \circ \phi \left( \sqrt{2 + \sqrt{2}}\right) &= \phi \left( \sqrt{2 - \sqrt{2}}\right) \\ 
        &= \phi \left( \frac{ \sqrt{2} }{ \sqrt{2+\sqrt{2}} }\right)\\
        &= \frac{ \phi (\sqrt{2}) }{ \phi \left( \sqrt{2 + \sqrt{2}}\right) }\\
        &= \frac{ - \sqrt{2} }{ \sqrt{2-\sqrt{2}} }\\
        &= - \sqrt{2 + \sqrt{2}}
    \end{align}
    We also have
    \begin{align}
        \phi(2) + \phi(\sqrt{2}) &= \phi(2 + \sqrt{2})\\
        &= \phi \left( \sqrt{2 + \sqrt{2}} \cdot \sqrt{2 + \sqrt{2}} \right)\\
        &= \phi \left( \sqrt{2 + \sqrt{2}} \right) \cdot \phi \left( \sqrt{2 + \sqrt{2}} \right)\\
        &= \left( \sqrt{2 - \sqrt{2}} \right) \cdot \left( \sqrt{2 - \sqrt{2}} \right)\\
        &= 2 - \sqrt{2}
    \end{align}
    so $\phi(2) + \phi(\sqrt{2}) = 2 - \sqrt{2}$,
    since $\phi(2) = 2$ implies $\phi(\sqrt{2}) = - \sqrt{2}$.\\
    Since $\phi^2$ is not identity automorphism, $\lvert \phi \rvert \geq 3,$ 
    and we found the desired $\phi$.\\
    We can check $\phi$ indeed permutes $X_1, X_2, X_3, X_4$.\\
    \begin{align}
        X_2 &\xrightarrow[]{\phi}    X_4\\
        \phi(X_1) &= \phi(-\sqrt[]{2+\sqrt[]{2}} = - \sqrt[]{2-\sqrt[]{2}} = X_3\\
        \phi(X_3) &= \phi(- \sqrt[]{2-\sqrt[]{2}}) = (-1)\cdot (-\sqrt[]{2+\sqrt[]{2}}) = \sqrt[]{2+\sqrt[]{2}} = X_2\\
        \phi(X_4) &= - \sqrt[]{2+\sqrt[]{2}} = X_1\\
        \phi: X_1 &\xrightarrow[]{} X_3 \xrightarrow[]{} X_2 \xrightarrow[]{} X_4 \xrightarrow[]{} X_1
    \end{align}

    \part

    Show $\mathbb{Q}(\gamma, i) = \mathbb{Q}(\phi)$ with $\phi^4 = i$.\\
    Use formula
    \begin{align}
        \cos (A) &= 1 - 2 \sin^2 \frac{ A }{ 2 }\\
        1 - 2 \sin^2(22.5^{\circ}) &= \cos 45^{\circ}\\
        \sin^2 (22.5^{\circ}) &= \frac{ \sqrt[]{2}-1 }{ 2 \sqrt{2} }\\
        \sin^2 (22.5^{\circ}) &= \frac{ \sqrt[]{2-\sqrt[]{2}} }{ 2 } 
    \end{align}
    And also,
    \begin{align}
        \cos (A) &= 2 \cos^2 \frac{ A }{ 2 } - 1\\
        \cos 45^{\circ} &= 2 \cos^2 22.5^{\circ} - 1\\
        \frac{ 1 }{ \sqrt[]{2} }+1 &= 2 \cos^2 (22.5^{\circ})\\
        \sqrt[]{\frac{ 1+\sqrt{2} }{ 2 \sqrt{2} }} &= \cos (22.5^{\circ})
    \end{align}
    
    


    

    
    
    

    
    
    
    

    





\end{homeworkProblem}

\pagebreak


\begin{homeworkProblem}
    \textbf{Exercise 7.2.7} Find the degree of 
    \begin{align}
        \sqrt[5]{81} + 29\sqrt[5]{9} + 17\sqrt[5]{3} - 16
    \end{align}
    over $\Q$.

    \solution

\end{homeworkProblem}

\pagebreak

\begin{homeworkProblem}
    \textbf{Exercise 7.2.8} Find the degree of $\sqrt[5]{81}$ over $\Q(\sqrt[81]{5})$.

    \solution

\end{homeworkProblem}


\end{document}
