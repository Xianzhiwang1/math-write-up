\documentclass{article}

%
% Homework Details
%   - Title
%   - Due date
%   - Class
%   - Section/Time
%   - Instructor
%   - Author
%

\newcommand{\hmwkTitle}{GAL \ \#09}
\newcommand{\hmwkDueDate}{Apr 29, 2022}
\newcommand{\hmwkClass}{Galois Thy}
\newcommand{\hmwkClassTime}{Sec 12 \& 13}
\newcommand{\hmwkClassInstructor}{Prof Matyas Domokos}
\newcommand{\hmwkAuthorName}{\textbf{Xianzhi} }
\newcommand{\hmwkAuthor}{\textit{Xianzhi Wang}}


\usepackage{fancyhdr}
\usepackage{extramarks}
\usepackage{amsmath}
\usepackage{amsthm}
\usepackage{amsfonts}
\usepackage{tikz}
\usepackage[plain]{algorithm}
\usepackage{algpseudocode}

\usetikzlibrary{automata,positioning}

%
% Basic Document Settings
%

\topmargin=-0.45in
\evensidemargin=0in
\oddsidemargin=0in
\textwidth=6.5in
\textheight=9.0in
\headsep=0.25in

\linespread{1.1}

\pagestyle{fancy}
\lhead{\hmwkAuthorName}
\chead{\hmwkClass\ (\hmwkClassInstructor\ \hmwkClassTime): \hmwkTitle}
\rhead{\firstxmark}
\lfoot{\lastxmark}
\cfoot{\thepage}

\renewcommand\headrulewidth{0.4pt}
\renewcommand\footrulewidth{0.4pt}

\setlength\parindent{0pt}

%
% Create Problem Sections
%

\newcommand{\enterProblemHeader}[1]{
    \nobreak\extramarks{}{Problem \arabic{#1} continued on next page\ldots}\nobreak{}
    \nobreak\extramarks{Problem \arabic{#1} (continued)}{Problem \arabic{#1} continued on next page\ldots}\nobreak{}
}

\newcommand{\exitProblemHeader}[1]{
    \nobreak\extramarks{Problem \arabic{#1} (continued)}{Problem \arabic{#1} continued on next page\ldots}\nobreak{}
    \stepcounter{#1}
    \nobreak\extramarks{Problem \arabic{#1}}{}\nobreak{}
}

\setcounter{secnumdepth}{0}
\newcounter{partCounter}
\newcounter{homeworkProblemCounter}
\setcounter{homeworkProblemCounter}{1}
\nobreak\extramarks{Problem \arabic{homeworkProblemCounter}}{}\nobreak{}

%
% Homework Problem Environment
%
% This environment takes an optional argument. When given, it will adjust the
% problem counter. This is useful for when the problems given for your
% assignment aren't sequential. See the last 3 problems of this template for an
% example.
%
\newenvironment{homeworkProblem}[1][-1]{
    \ifnum#1>0
        \setcounter{homeworkProblemCounter}{#1}
    \fi
    \section{Problem \arabic{homeworkProblemCounter}}
    \setcounter{partCounter}{1}
    \enterProblemHeader{homeworkProblemCounter}
}{
    \exitProblemHeader{homeworkProblemCounter}
}


%
% Title Page
%

\title{
    \vspace{2in}
    \textmd{\textbf{\hmwkClass:\ \hmwkTitle}}\\
    \normalsize\vspace{0.1in}\small{Due\ on\ \hmwkDueDate\ at 11:59PM}\\
    \vspace{0.1in}\large{\textit{\hmwkClassInstructor\ \hmwkClassTime}}
    \vspace{3in}
}

\author{\hmwkAuthorName}
\date{2023}

\renewcommand{\part}[1]{\textbf{\large Part \Alph{partCounter}}\stepcounter{partCounter}\\}

%
% Various Helper Commands
%

% Useful for algorithms
\newcommand{\alg}[1]{\textsc{\bfseries \footnotesize #1}}

% For derivatives
\newcommand{\deriv}[1]{\frac{\mathrm{d}}{\mathrm{d}x} (#1)}

% For partial derivatives
\newcommand{\pderiv}[2]{\frac{\partial}{\partial #1} (#2)}

% Integral dx
\newcommand{\dx}{\mathrm{d}x}

% Alias for the Solution section header
\newcommand{\solution}{\textbf{\large Solution:}}

% Probability commands: Expectation, Variance, Covariance, Bias
\newcommand{\E}{\mathrm{E}}
\newcommand{\Var}{\mathrm{Var}}
\newcommand{\Cov}{\mathrm{Cov}}
\newcommand{\Bias}{\mathrm{Bias}}

% From xianzhi.sty

% Fancy
\newcommand{\cA}{\mathcal A}
\newcommand{\cB}{\mathcal B}
\newcommand{\cC}{\mathcal C}
\newcommand{\cD}{\mathcal D}
\newcommand{\cE}{\mathcal E}
\newcommand{\cF}{\mathcal F}
\newcommand{\cG}{\mathcal G}
\newcommand{\cH}{\mathcal H}
\newcommand{\cI}{\mathcal I}
\newcommand{\cJ}{\mathcal J}
\newcommand{\cL}{\mathcal L}
\newcommand{\cM}{\mathcal M}
\newcommand{\cN}{\mathcal N}
\newcommand{\cO}{\mathcal O}
\newcommand{\cP}{\mathcal P}
\newcommand{\cR}{\mathcal R}
\newcommand{\cS}{\mathcal S}
\newcommand{\cT}{\mathcal T}
\newcommand{\cU}{\mathcal U}
\newcommand{\cW}{\mathcal W}
\newcommand{\cX}{\mathcal X}
\newcommand{\cY}{\mathcal Y}


\newcommand{\bP}{\mathbb{P}}

\newcommand{\C}{\mathbb{C}}
\newcommand{\R}{\mathbb{R}}
\newcommand{\N}{\mathbb{N}}
\newcommand{\Q}{\mathbb{Q}}
\newcommand{\Z}{\mathbb{Z}}
\newcommand{\F}{\mathbb{F}}

% Brackets
\newcommand{\bigp}[1]{\left( #1 \right)} % (x)
\newcommand{\bigb}[1]{\left[ #1 \right]} % [x]
\newcommand{\bigc}[1]{\left\{ #1 \right\}} % {x}
\newcommand{\biga}[1]{\left\langle #1 \right\rangle} % <x>

%norm

% theorem 
\newtheorem{Proposition}{proposition}
\newtheorem{Assumption}{assumption}
\newtheorem{Definition}{definition}
\newtheorem{Corollary}{corollary}
\newtheorem{Question}{question}



% \begin{document}

% \usepackage{xianzhi}

\begin{document}

\maketitle
HW09 \\
Apr 29, 2022 \\
Exercise 12.4.11\\
Exercise 13.3.3\\
Exercise 13.3.5\\
% \pagebreak

\begin{homeworkProblem}
\textbf{Exercise 12.4.8} Factor $x^4 + x + 1 \in \F_2[x]$ as a product of irreducibles over $\F_4$.\\
\solution \\

\end{homeworkProblem}

\pagebreak

\begin{homeworkProblem}
\textbf{Exercise 12.4.11} Factor $x^{375} + x^{250} + 2 $ over $ \F_5$ into the product of irreducibles. \\
\solution \\
Observe that since $2 \in \F_5$, Frobenius automorphism fixes $2$,
\begin{align}
    & \F_5 \xrightarrow{\Phi} \F_5\\
    & a \rightarrow a^5 \\
    \text{where} \ &\Phi(2) = 2^5 = 32 = 2 \mod 5.
\end{align}
Thus, $\Phi^3(2) = \Phi \circ \Phi \circ \Phi (2) = 2 \implies 2^{125} = 2$.\\
Thus, because $char \ \F_5 = 5$, we have 
\begin{align}
    X^{375} + X^{250} + 2 &= X^{375} + X^{250} + 2^{125}\\
    &= (X^3 + X^2 + 2)^{125}
\end{align}
If $X^3 + X^2 + 2$ is reducible in $\F_5$, it must have a linear factor, so suffice to check for linear factors.\\
\begin{align}
    \text{when} \ x &= 0, \ x^3 + x^2 + 2 = 2\\
    \text{when} \ x &= 1, \ x^3 + x^2 + 2 = 4\\
    \text{when} \ x &= 2, \ x^3 + x^2 + 2 = 14\\
    \text{when} \ x &= 3 = -2, \ x^3 + x^2 + 2 = 3\\
    \text{when} \ x &= -1, \ x^3 + x^2 + 2 = 2
\end{align}
Thus, $x^3 + x^2 + 2$ has no linear factors, hence irreducible.\\
So $(x^3 + x^2 + 2)^{125}$ is the desired factorization.



\end{homeworkProblem}

\pagebreak


\begin{homeworkProblem}
    \textbf{Exercise 13.3.3} Let $p \in \Q [x]$ be a quartic polynomial with $Gal_{\Q}(p) \cong D_4$, the dihedral group of order $8$.
    \begin{enumerate}
        \item Show that $p$ is irreducible over $\Q$.
        \item Show that the cubic resolvent of $p$ has a rational root.
    \end{enumerate}
    \part
    \textbf{Soln 1: }\\
    Let $p \in \Q [x]$ be quartic satisfying the assumption. Assume for contradiction that $p$ is reducible.\\
    So $p$ factor into 
    \begin{enumerate}
        \item $4=3+1$,
        \item $4 = 2+2$,
        \item $4=2+1+1$,
        \item $4=1+1+1+1$
    \end{enumerate}
    \textbf{Case 1:} $3+1$. a irred. cubic factor $f$ and a linear factor. Then $p$ has a root in $\Q$, and $Gal_{\Q}(p)=Gal_{\Q}(f)$,
    and since degree of extension of the splitting field of a polynomial of degree $n$ is $\leq n!$, we have 
    \begin{align}
        &|Gal_{\Q}(p)| = |Gal_{\Q}(f)| \leq 3! = 6,\\
        \text{but} \ &|Gal_{\Q}(p)| = 8.
    \end{align}
    Hence we have a contradiction.\\
    \textbf{Case 2:} $2+2$. $p$ factor into $2$ quadratic factor, $f$ and $g$. Now, let $L$ be the splitting field of $f\cdot g$ over $\Q$.
    Then $|Gal_{\Q}(p)| = |[L:\Q]| \leq 4$. Reason: when we add the roots of $f$, we get a degree $2$ extension (degree 2 extension is normal). 
    Then if roots of $g$ are already in this extension, $[L:\Q]=2,$ if not, then $[L:\Q]=4$.\\
    \textbf{Case 3:} $2+1+1$ we have $|Gal_{\Q}(p)| = 2$ Contradiction.\\
    \textbf{Case 4:} $1+1+1+1$ we have $|Gal_{\Q}(p)| = 1$ Contradiction.\\ 
    \part
    \textbf{Soln 2:}\\
    \begin{align}
\varphi: Gal_{\Q}(p) \hookrightarrow  S_4
    \end{align}
    We could imbed $Gal_{\Q}(p)$ into $S_4$ since an automorphism $\phi \in Gal_{\Q}(p)$ permutes the roots of $p$, ($p$ is irred.) which we call $\alpha_1, \alpha_2, \alpha_3, \alpha_4$.\\
    Thus, $\phi$ permutes the label $1,2,3,4$. Since every element of $Gal_{\Q}(p)$ defines uniquely a permutation of $1,2,3,4$, we could map $\phi \in Gal_{\Q}(p)$ to 
    its corresponding permutation, which is a homomorphism, because both group operations are composition. This homomorphism $\varphi$ is injective, since each 
    element of $Gal_{\Q}(p)$ corresponds uniquely to a permutation. Thus,\\
\begin{align}
    D_4 \cong Gal_{\Q}(p) \cong Im \varphi \leq S_4
\end{align}
Since groups of order $8$ in $S_4$ are Sylow-2 subgroups of $S_4$, and they are all conjugates of each other, then any conjugate copy of $Im \varphi$
is going to act like $D_4$ on $\alpha_1, \alpha_2, \alpha_3, \alpha_4,$ up to a relabel of the roots.\\
Thus, knowing $Gal_{\Q}(p) \cong D_4$ means $Gal_{\Q}(p)$ will act on the $\alpha$'s just like $D_4$ permutes $1,2,3,4$, up to a relabeling of the roots.\\
Thus, use the formula, let the roots of cubic resolvent of $p$ be denoted by $\beta_1, \beta_2, \beta_3,$ where
\begin{align}
    &\beta_1 = (\alpha_1+\alpha_2)(\alpha_3+\alpha_4)  \\
    &\beta_2 = (\alpha_1+\alpha_3)(\alpha_2+\alpha_4) \\ 
    &\beta_3 = (\alpha_1+\alpha_4)(\alpha_2+\alpha_3)  
\end{align}
Write 
\begin{align}
    D_4 = \{ id, (1324), (12)(34), (1423), (13)(24), (14)(23), (12), (34) \}
\end{align}
Then we check
\begin{align}
    & \beta_1^{id} = \beta_1 = (\alpha_1+\alpha_2)(\alpha_3+\alpha_4)\\
    & \beta_1^{(1324)} = (\alpha_3+\alpha_4)(\alpha_2+\alpha_1) = \beta_1\\
    & \beta_1^{(12)(34)} = (\alpha_2+\alpha_1)(\alpha_4+\alpha_3) = \beta_1\\
    & \beta_1^{(1423)} = (\alpha_4+\alpha_3)(\alpha_1+\alpha_2)=\beta_1\\
    & \beta_1^{(12)} = (\alpha_2+\alpha_1)(\alpha_3+\alpha_4) = \beta_1\\
    & \ldots 
\end{align}
We could go on and check the action of all $8$ elements of $D_4$. However, it would be sufficient to check that $\beta_1$ is fixed by the generators $(1324)$ and $(12)$.\\
Thus, $\beta_1$ is fixed by $Gal_{\Q}(p) \Rightarrow \ \beta_1 \in \Q$\\
Since the labeling of the roots $\alpha_1,\alpha_2,\alpha_3, \alpha_4$ is arbitrary, we showed there is a rational root (of the cubic resolvent of $p$).


    
\end{homeworkProblem}

\pagebreak

\begin{homeworkProblem}
\textbf{Exercise 13.3.5} Let $\alpha$ be a root of $x^2 + ax + b$ and $\beta$ a root of $x^3 + px + q$. 
Write down a polynomial with coefficients in $\Q(a,b,p,q)$ having $\alpha + \beta$ as a root.\\
\solution \\

Let $\alpha_1, \alpha_2$ be roots of $x^2 + ax + b$.\\
Let $\beta_1, \beta_2, \beta_3$ be roots of $x^3 + px + q$. Hence, 
\begin{align}
    & \alpha_1 + \alpha_2 = -a\\
    & \alpha_1 \cdot \alpha_2 = b
\end{align}
We have 
\begin{align}
    & \alpha_1^2 + \alpha_2^2 = (\alpha_1+\alpha_2)^2 - 2 \alpha_1 \alpha_2 = a^2 - 2 b\\
    & \alpha_1^3 + \alpha_2^3 = (\alpha_1+\alpha_2)^3 - 3 \alpha_1 \alpha_2 (\alpha_2 + \alpha_1) = - a^3 + 3 b a
\end{align}
Similarly, we have 
\begin{align}
    & \beta_1 + \beta_2 + \beta_3 = 0\\
    & \beta_1 \beta_2 + \beta_1 \beta_3 + \beta_2 \beta_3 = p\\
    & \beta_1 \beta_2 \beta_3 = -q
\end{align}
A polynomial with coefficients in $\Q (a,b,p,q)$ having $\alpha_i + \beta_j$ as a root is:
\begin{align}
    \prod_{i=1, j=1}^{2,3} (X-\alpha_i - \beta_j)
\end{align}
To see why this works, we expand:
\begin{align}
    & \prod_{i=1, j=1}^{2,3} (X-\alpha_i - \beta_j)\\
    & = \bigb{(X-\alpha_1)^3 - \beta_3(X- \alpha_1)^2 - (\beta_1 + \beta_2)(X- \alpha_1)^2 + (\beta_1 + \beta_2)\beta_3 (X - \alpha_1) + \beta_1 \beta_2 (X - \alpha_1) - \beta_1 \beta_2 \beta_3 } \\
    & \cdot \bigb{\ldots}\\
    & = \bigb{(X-\alpha_1)^3 + p (X-\alpha_1) + q} \bigp{(X-\alpha_2)^3 + p(X-\alpha_2) + q} \\
    & = [X^2 + aX + b]^3 + p^2 [ X^2 + aX + b] + q^2\\
    & + p(X-\alpha_2)(X-\alpha_1)^3 + p(X-\alpha_1)(X-\alpha_2)^3\\
    & + q(X-\alpha_1)^3 + pq (X-\alpha_1) + q (X-\alpha_2)^3 + pq (X - \alpha_2)
\end{align}
expand, we have
\begin{align}
    & = [X^2 + aX + b]^3 + p^2 [ X^2 + aX + b] + q^2\\
    & + p(X^2 + a X + b)(X^2 -2 \alpha_1 X + \alpha_1^2 + X^2 - 2 \alpha_2 X + \alpha_2^2)\\
    & + 2pqX + apqX\\
    & + q[ 2X^3 + 3X(\alpha_1^2 + \alpha_2^2) + 3aX^2 - (\alpha_1^3 + \alpha_2^3)]
\end{align}
Now, use the expression for $\alpha_1 + \alpha_2$, $\alpha_1^2 + \alpha_2^2$, and $\alpha_1^3 + \alpha_2^3$ that we obtained earlier, we see that we obtain 
a degree 6 polynomial with coefficients in $\Q (a,b,p,q)$. And we are done.
    
    

\end{homeworkProblem}


\end{document}
