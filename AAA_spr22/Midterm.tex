\documentclass[12pt,english]{article}
\usepackage{xianzhi}

% \usepackage[a4paper,
%             bindingoffset=0.2in,
%             left=1in,
%             right=1in,
%             top=1in,
%             bottom=1in,
%             footskip=.25in]{geometry}
% \usepackage[utf8]{inputenc}
% \usepackage{soul}
% \usepackage{tikz}
% \usepackage{comment}
% %%%%%%%%%%%%%%%%%%%%%%%%%%
% \usepackage{bbm}
% \usepackage{mathrsfs}
% \usepackage{amsmath}
% \usepackage{empheq}
% \usepackage{amsfonts}
% \usepackage{amsthm}
% \usepackage{amssymb}
% \usepackage{blkarray}
% \usepackage{mathtools}
% \usepackage{physics}
% \usepackage{thmtools}
% \usepackage{thm-restate}
% %%%%%%%%%%%%%%%%%%%%%%%%%%%
% \declaretheorem[name=Proposition]{proposition}
% \declaretheorem[name=Assumption]{assumption}
% \declaretheorem[name=Definition]{definition}
% \declaretheorem[name=Lemma]{lemma}
% \declaretheorem[name=Theorem]{theorem}
% \declaretheorem[name=Corollary]{corollary}


\title{Adv Abstract Algebra Spr2022 midterm}
\author{xianzhi wang}
\date{Sept 2022}

\begin{document}

\maketitle

\section*{Q 1}
\begin{question}
Let $H \leq G$ such that $|G:H| = 4$. Prove: if $g \in G$ and $g$ has order 19, then $g \in H$. 
\end{question}
Let $G$ act on $\bigc{Hb} = \cR$, the set of right cosets by right action. We have homomorphism 
\begin{align*}
    \rho: G &\mapsto Sym(\cR) \cong S_4 \quad \text{since there are 4 cosets of } H.\\
    g &\mapsto g^{\rho},\\
    g^{\rho}: \cR &\mapsto \cR\\
    Hb &\mapsto Hbg
\end{align*}
Let $g \in G$, and $|g| = 19$. Want to show $g \in H$. Since $\rho(g)$ is in $Sym(\cR) \cong S_4$, $|\rho(g)|$ divide $4!$, which is the order of $S_4$. Also, $|\rho(g)|$ divide $|g| = 19$ since $\rho$ is homomorphism, so $|\rho(g)|$ divide $\gcd (24,19) = 1$, so $\rho(g) = id$. So $g$ is mapped to the identity permutation on the right cosets of $H$. Thus, 
\begin{align*}
    g \in \ker \rho = \bigcap_{g \in G} g^{-1}Hg \leq H. 
\end{align*}










\section*{Q 2}
\begin{question}
let a (finite) group $G$ act on $\Omega$ and on $\Delta$. the action on $\Omega$ is transitive, and $|\Omega| = 22$, $|\Delta| = 10$. Prove that the action of $G$ on $\Delta$ is not faithful. (i.e. the kernel of the action on $\Delta$ is non-trivial)
\end{question}
Use O-S, $G$ act on $\Omega$, for $x \in \Omega$, $|\cO(x)| = 22$,
\begin{align*}
    |\cO(x)| \cdot |stab(x)| = |G| \implies 22 \mid |G|
\end{align*}



Let $\rho$ be denote the homomorphism assoicated with the group action.
\begin{align*}
    \rho: G &\mapsto Sym(\Delta) \cong S_{10}\\
    g &\mapsto g^{\rho}
\end{align*}
We have
\begin{align*}
    &G/\ker \rho \cong \Im \rho \leq Sym (\Delta) \\
    & |\Im \rho| \ \text{divide} \ |G| = 22k\\
    & |\Im \rho| \ \text{divide} \ 10!
\end{align*}

If $G$ has finite order, then assume for contradiction $|\ker \rho | =1$, then $|G| = |\Im \rho|$ divide $10!$, but $|G|$ has a factor of 11. Contradiction. So $\ker \rho >1$. \\

If $G$ has infinite order, then since $|\Im \rho| \leq 10! < \infty$, $|\ker \rho|$ must be infinite. 


\section*{Q 3}
\begin{question}
Let $\C^{\times}$ denote the multiplicative group of all non-zero complex numbers (under the ordinary multiplication). Prove that $\C^{\times}$ does not have any non-trivial subgroup of finite index.
\end{question}
Supp that $H$ has finite index in $\C^{\times}$ and $m = [\C^{\times}:H]$. Then for any nonzero complex number $z^m \in H$, we have
\begin{align*}
    a^m H = (zH)^m = H, \implies z^m \in H.
\end{align*} For all $w \in \C$, we can solve $z^m -w =0$ to write $w$ as $z^m$ for some $z$, hence $w\in H$, so $\C \subseteq H$, so $H = \C$. 


\section*{Q 4}
\begin{proposition}
Let a group $G$ have order $2^2 \cdot 5 \cdot 17$. Show that 
\begin{enumerate}
    \item $G$ has a unique Sylow $5$-subgroup and a unique Sylow $17$-subgroup.
    \item $\exists$ an element of order $85 = 5 \cdot 17$ in $G$.
\end{enumerate}

\end{proposition}


\section*{Q 5}
\begin{proposition}
    Let $Z \trianglelefteq G$ such that $ \lvert Z \rvert = 2$ and $ \lvert G:Z \rvert =97 $. Show that
    \begin{enumerate}
        \item $Z \leq Z(G)$
        \item $G$ is cyclic.
    \end{enumerate}
    
\end{proposition}










\end{document}
