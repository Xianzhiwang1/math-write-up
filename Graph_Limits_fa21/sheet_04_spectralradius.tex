\documentclass[12pt]{article}%
\usepackage{amsfonts}
\usepackage{fancyhdr}
\usepackage[a4paper, top=2.5cm, bottom=2.5cm, left=2.2cm, right=2.2cm]%
{geometry}
\usepackage{times}
\usepackage{amsmath}
\usepackage{amssymb}
\usepackage{graphicx}%
\setcounter{MaxMatrixCols}{30}

\newtheorem{theorem}{Theorem}
\newtheorem{acknowledgement}[theorem]{Acknowledgement}
\newtheorem{algorithm}[theorem]{Algorithm}
\newtheorem{axiom}{Axiom}
\newtheorem{case}[theorem]{Case}
\newtheorem{claim}[theorem]{Claim}
\newtheorem{conclusion}[theorem]{Conclusion}
\newtheorem{condition}[theorem]{Condition}
\newtheorem{conjecture}[theorem]{Conjecture}
\newtheorem{corollary}[theorem]{Corollary}
\newtheorem{criterion}[theorem]{Criterion}
\newtheorem{definition}[theorem]{Definition}
\newtheorem{example}[theorem]{Example}
\newtheorem{exercise}[theorem]{Exercise}
\newtheorem{lemma}[theorem]{Lemma}
\newtheorem{notation}[theorem]{Notation}
\newtheorem{problem}[theorem]{Problem}
\newtheorem{proposition}[theorem]{Proposition}
\newtheorem{remark}[theorem]{Remark}
\newtheorem{solution}[theorem]{Solution}
\newtheorem{summary}[theorem]{Summary}
\newenvironment{proof}[1][Proof]{\textbf{#1.} }{\ \rule{0.5em}{0.5em}}
\begin{document}

\title{Sheet 4: Spectral radius and randomness }
\author{Xianzhi Wang}
\date{fall 2021}
\maketitle

Let $G$ be a finite undirected $d$-regular graph on $n$ vertices. Let $A$
denote the adjacency matrix of $G$, and let $b_{0},b_{1},\ldots,b_{n-1}$ be an
orthonormal eigenbasis for $A$ with real eigenvalues $\lambda_{0}\geq
\lambda_{1}\geq\ldots\geq\lambda_{n-1}$.

\begin{definition}
Let the spectal radius of $G$ be
\[
\rho_{0}(G)=\max\left\{  \left\vert \lambda_{i}\right\vert \mid1\leq i\leq
n-1\right\}  \text{.}%
\]

\end{definition}

That is, we exclude the trivial eigenvalue $d$.

\begin{lemma}
If $G$ is connected, then 
$b_{0}=\left(  1/\sqrt{n},\ldots,1/\sqrt{n}\right)$.
\end{lemma}

\begin{solution}
    If $G$ is connected, then 
    \begin{align}
        b_0 = \frac{ 1 }{ \sqrt[]{n} }
        \begin{bmatrix}
            1\\
            1\\
            \vdots\\
            1
        \end{bmatrix}
    \end{align}
    Since $G$ is a finite, undirected, $d$-regular graph on $n$ vertices, 
    thus, by results from the previous sheet, $d$ regular and connectedness
    implies $\lambda_0 (G) = d$, where $d$ is also maximal degree.\\
    The associated eigenvector for eigenvalue $d$ is the all $1$'s vector,     
    \begin{align}
        \begin{bmatrix}
            1\\
            1\\
            \vdots\\
            1
        \end{bmatrix}
    \end{align}
    since every row of the adjacency matrix of a regular graph
    sum to $d$. Thus, we just need to normalize the all one's vector to get
    \begin{align}
        b_0 = \frac{ 1 }{ \sqrt[]{n} }
        \begin{bmatrix}
            1\\
            1\\
            \vdots\\
            1
        \end{bmatrix}
    \end{align}

    
\end{solution}

Let $S,T$ be subsets of $V(G)$. Assume now that the edges of $G$ are chosed randomly.

\begin{exercise}
What value do we expect for $e(S,T)$ for a random $G$?
\end{exercise}

\textbf{solution:}\\
    Originally, $e(S,T) = \chi^T_S A \chi_T$,
    but here, the edges between the 2 vertex set, 
    $S,T$ are chosen randomly.\\
    since the graph $G$ is $d$-regular on $n$ vertices, 
    there must be $\frac{ d \cdot n }{ 2 }$ edges 
    in total, and we have ${n \choose 2}$ possible
    ``slots'' to put those edges. And we are 
    doing this randomly. Thus, for each ``slot'',
    the possibility that there is an edge there, is
    \begin{align}
        \mathbb{P}(\exists \ \text{an edge}) = 
        \frac{ \frac{ d \cdot n }{ 2 } }{ \frac{ n(n-1) }{ 2 } } 
        = \frac{ d }{ n-1 }
    \end{align}
    Thus, we should have a matrix 
    \begin{align}
        P = 
        \begin{bmatrix}
        0 & & & \frac{ d }{ n-1 }\\
        & 0 & & \\
        & & \ddots & \\
        \frac{ d }{ n-1 }& & &0
        \end{bmatrix}
    \end{align}
    with zero on the diagonal and $\frac{ d }{ n-1 }$ everywhere
    except the diagonal. 
    $e(S,T) = \chi_S^T P \chi_T$ would take into account that
    an edge is in place with possibility $\frac{ d }{ n-1 }$.\\
    
\textbf{end solution}\\

It turns out that when the spectral radius is small, the graph mimicks the
random behaviour.

\begin{exercise}
For $S\subseteq V(G)$ decompose the characteristic function $X_{S}$ as
\[
X_{S}=%
%TCIMACRO{\dsum _{i=0}^{n-1}}%
%BeginExpansion
{\displaystyle\sum_{i=0}^{n-1}}
%EndExpansion
\alpha_{i}b_{i}\text{.}%
\]
What is $\alpha_{0}$? What is $%
%TCIMACRO{\dsum _{i=0}^{n-1}}%
%BeginExpansion
{\displaystyle\sum_{i=0}^{n-1}}
%EndExpansion
\alpha_{i}^{2}$?
\end{exercise}

\begin{solution}
    $S \subseteq V(G)$ decompose the characteristic function $\chi_s$
    as $\chi_s = \sum_{i=0}^{n-1} \alpha_i b_i$ where
    \begin{align}
        \chi_s (v) = \begin{cases}
            &1 \ \text{if} \ v \in S\\
            &0 \ \text{if} \ v \not\in S
        \end{cases}
    \end{align}
    We could think of $\chi_s$ as a column vector, 
    where we have the $n$ vertices, and we put 
    $1$ and $0$ according to whether $v_i$ is 
    in $S$ or not.\\
    Since $\chi_s$ is a vector, it could be 
    expressed as a linear combination of eigenvectors, 
    since those eigenvectors form an orthonormal eigenbasis:
    \begin{align}
        b_0, b_1, \ldots, b_{n-1}.
    \end{align}
    Thus, we have
    \begin{align}
        \chi_s = \alpha_0 b_0 + \alpha_1 b_1 + \ldots + \alpha_{n-1} b_{n-1}
    \end{align}
    $\alpha_0$ is the coefficient corresponding to the eigenvector
    \begin{align}
        b_0 = \frac{ 1 }{ \sqrt[]{n} }
        \begin{bmatrix}
            1\\
            1\\
            \vdots\\
            1
        \end{bmatrix}
    \end{align}
    Also, 
    \begin{align}
        \left\langle \chi_s, b_0 \right\rangle &= \left\langle \sum_i \alpha_i b_i, b_0 \right\rangle\\
        &= \sum_i \alpha_i \left\langle b_i, b_0 \right\rangle\\
        &= \sum_i \alpha_i \delta_{i,0}\\
        &= \alpha_0 \\
        \sum_{i=0}^{n-1} \alpha_i^2 &= \alpha_0^2 + \alpha_1^2 + \ldots + \alpha_{n-1}^2
    \end{align}

    
\end{solution}

\begin{theorem}
For all $S,T\subseteq V(G)$ we have
\[
\left\vert e(S,T)-\frac{d\left\vert S\right\vert \left\vert T\right\vert }%
{n}\right\vert \leq\rho_{0}(G)\sqrt{\left\vert S\right\vert \left\vert
T\right\vert }\text{.}%
\]

\end{theorem}

\begin{proof}
    As suggested in exercises that builds up to this theorem, 
    write
    \begin{align}
        \chi_S &= \sum_{i=0}^{n-1} \alpha_i b_i \\
        \chi_T &= \sum_{i=0}^{n-1} \beta_i b_i 
    \end{align}
    in eigenbasis. The $\alpha_i$ and $\beta_i$ are coefficients, and 
    \begin{align}
        \alpha_i &= \left\langle \chi_S, b_i \right\rangle\\
        \beta_i &= \left\langle \chi_T, b_i \right\rangle 
    \end{align}
    also, in perticular, 
    \begin{align}
        \alpha_0 &= \left\langle \chi_S, b_0 \right\rangle\\
        &= \left\langle \chi_S, \frac{ 1 }{ \sqrt[]{n} } \mathbf{1} \right\rangle \ \text{where} \ \mathbf{1} \ \text{is} \ n \times 1 \ \text{vector of all 1's}\\
        &= \frac{ 1 }{ \sqrt[]{n} } \left\langle \chi_S, \mathbf{1} \right\rangle \\
        &= \frac{ 1 }{ \sqrt[]{n} } \lvert S \rvert
    \end{align}
    similarly, $\beta_0 = \frac{ 1 }{ \sqrt[]{n} } \lvert T\rvert$.\\
    Thus 
    \begin{align}
    e (S,T) &= \chi_S^T A \chi_T\\
    &= \sum_{i=0}^{n-1} \alpha_i b_i^T A \beta_j b_j\\
    &= \sum_{i=0}^{n-1} \alpha_i \beta_i \lambda_i \ \text{since} \ b_i \ \text{form orthonormal basis}\\
    &= \alpha_0 \beta_0 \lambda_0 + \sum_{i=1}^{n-1} \alpha_i \beta_i \lambda_i\\
    &= \frac{ \lvert S \rvert }{ \sqrt[]{n} } \frac{ \lvert T \rvert }{ \sqrt[]{n} } d +  \sum_{i=1}^{n-1} \alpha_i \beta_i \lambda_i.
    \end{align}
    Thus, we have
    \begin{align}
        \lvert e(S,T) - \frac{ d \lvert S\rvert \lvert T\rvert }{ n }\rvert = \lvert \sum_{i=1}^{n-1} \alpha_i \beta_i \lambda_i \rvert.
    \end{align}
    Now, LHS of equation we want to show
    \begin{align}
        &= \lvert \sum_{i=1}^{n-1} \alpha_i \beta_i \lambda_i \rvert\\
        &\leq \lvert \sum_{i=1}^{n-1} \alpha_i \beta_i \rvert \rho_0 (G)\\
        &\leq \lvert \sum_{i=0}^{n-1} \alpha_i \beta_i \rvert \rho_0 (G)\\
        &= \lvert \left\langle \chi_S, \chi_T \right\rangle\rvert \rho_0 (G)\\
        &\leq \lVert \chi_S \rVert \lVert \chi_T \rVert \rho_0 (G)\\
        &= \sqrt[]{\lvert S \rvert} \sqrt[]{\lvert T \rvert} \rho_0 (G)\\
        &= \rho_0 (G) \sqrt[]{\lvert S \rvert \lvert T \rvert}\\
        &= RHS
    \end{align}
    observe 
    \begin{align}
        &\lvert \left\langle \chi_S, \chi_T \right\rangle\rvert\\
        &= \lvert \left\langle \sum_i \alpha_i b_i, \sum_j \beta_j b_j \right\rangle\rvert\\
        &= \lvert \sum_{i,j} \alpha_i \beta_j \left\langle b_i, b_j \right\rangle\rvert\\
        &= \lvert \sum_{i,j} \alpha_i \beta_j \delta_{i,j} \rvert\\
        &= \lvert \sum_i \alpha_i \beta_i \rvert
    \end{align}
    and 
    \begin{align}
        \lVert \chi_S \rVert &= \sqrt[]{\left\langle \chi_S, \chi_S \right\rangle}\\
        &= \sqrt[]{\lvert S \rvert}\\
        \lVert \chi_T \rVert &= \sqrt[]{\left\langle \chi_T, \chi_T \right\rangle}\\
        &= \sqrt[]{\lvert T \rvert}\\
    \end{align}
    
    
    
    
\end{proof}


A subset $S\subseteq V(G)$ is \emph{independent}, if $E(S,S)$ is empy. Using
the above theorem, one can easily get an upper bound for the maximal size of
an independent subset of $G$.


\begin{corollary}
An independent subset of $G$ has size at most
\[
\frac{\rho_{0}(G)}{d}n\text{.}%
\]

\end{corollary}

\begin{proof}
   From the previous theorem, we have $\forall S,T \subseteq V(G)$,
   \begin{align}
    \lvert e(S,T) - \frac{ d \lvert S \rvert \lvert T \rvert }{ n }\rvert 
    \leq \rho_0 (G) \sqrt[]{\lvert S \rvert \lvert T \rvert}
   \end{align}
   Let $S \subseteq V(G)$ be independent, 
   we have $e(S,S) = 0$.
   So the above formula reduce to 
   \begin{align}
       \lvert e(S,S) - \frac{ d \lvert S \rvert \lvert S \rvert }{ n }\rvert
       &\leq \rho_0 (G) \sqrt[]{\lvert S \rvert \lvert S \rvert}\\
       \implies \lvert \frac{ d }{ n }\rvert \lvert S \rvert^2 &\leq \rho_0 (G) \lvert S \rvert\\
       \implies \lvert \frac{ d }{ n }\rvert \lvert S \rvert &\leq \rho_0 (G) \\
       \lvert S \rvert &\leq \rho_0 (G) \frac{ n }{ d }.
   \end{align}
   Thus, an independent subset $S$ 
   has size at most $\frac{ \rho_0 (G) }{ d }n$.
   
   
\end{proof}


Of course, this bound is only meaningful, if $\rho_{0}(G)$ can go below, say,
$d/2$. It is also not clear how small $\rho_{0}(G)$ can get for large $d
$-regular graphs -- we will get a good estimate on this later, using graph limits.


\end{document}


