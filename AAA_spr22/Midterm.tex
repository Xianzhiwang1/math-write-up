\documentclass[12pt,english]{article}
\usepackage{xianzhi}

% \usepackage[a4paper,
%             bindingoffset=0.2in,
%             left=1in,
%             right=1in,
%             top=1in,
%             bottom=1in,
%             footskip=.25in]{geometry}
% \usepackage[utf8]{inputenc}
% \usepackage{soul}
% \usepackage{tikz}
% \usepackage{comment}
% %%%%%%%%%%%%%%%%%%%%%%%%%%
% \usepackage{bbm}
% \usepackage{mathrsfs}
% \usepackage{amsmath}
% \usepackage{empheq}
% \usepackage{amsfonts}
% \usepackage{amsthm}
% \usepackage{amssymb}
% \usepackage{blkarray}
% \usepackage{mathtools}
% \usepackage{physics}
% \usepackage{thmtools}
% \usepackage{thm-restate}
% %%%%%%%%%%%%%%%%%%%%%%%%%%%
% \declaretheorem[name=Proposition]{proposition}
% \declaretheorem[name=Assumption]{assumption}
% \declaretheorem[name=Definition]{definition}
% \declaretheorem[name=Lemma]{lemma}
% \declaretheorem[name=Theorem]{theorem}
% \declaretheorem[name=Corollary]{corollary}


\title{Adv Abstract Algebra Spr2022 midterm}
\author{xianzhi wang}
\date{Sept 2022}

\begin{document}

\maketitle

\section*{Q 1}
\begin{question}
Let $H \leq G$ such that $|G:H| = 4$. Prove: if $g \in G$ and $g$ has order 19, then $g \in H$. 
\end{question}
Let $G$ act on $\bigc{Hb} = \cR$, the set of right cosets by right action. We have homomorphism 
\begin{align*}
    \rho: G &\mapsto Sym(\cR) \cong S_4 \quad \text{since there are 4 cosets of } H.\\
    g &\mapsto g^{\rho},\\
    g^{\rho}: \cR &\mapsto \cR\\
    Hb &\mapsto Hbg
\end{align*}
Let $g \in G$, and $|g| = 19$. Want to show $g \in H$. Since $\rho(g)$ is in $Sym(\cR) \cong S_4$, $|\rho(g)|$ divide $4!$, which is the order of $S_4$. Also, $|\rho(g)|$ divide $|g| = 19$ since $\rho$ is homomorphism, so $|\rho(g)|$ divide $\gcd (24,19) = 1$, so $\rho(g) = id$. So $g$ is mapped to the identity permutation on the right cosets of $H$. Thus, 
\begin{align*}
    g \in \ker \rho = \bigcap_{g \in G} g^{-1}Hg \leq H. 
\end{align*}










\section*{Q 2}
\begin{question}
let a (finite) group $G$ act on $\Omega$ and on $\Delta$. the action on $\Omega$ is transitive, and $|\Omega| = 22$, $|\Delta| = 10$. Prove that the action of $G$ on $\Delta$ is not faithful. (i.e. the kernel of the action on $\Delta$ is non-trivial)
\end{question}
Use O-S, $G$ act on $\Omega$, for $x \in \Omega$, $|\cO(x)| = 22$,
\begin{align*}
    |\cO(x)| \cdot |stab(x)| = |G| \implies 22 \mid |G|
\end{align*}



Let $\rho$ be denote the homomorphism assoicated with the group action.
\begin{align*}
    \rho: G &\mapsto Sym(\Delta) \cong S_{10}\\
    g &\mapsto g^{\rho}
\end{align*}
We have
\begin{align*}
    &G/\ker \rho \cong \Im \rho \leq Sym (\Delta) \\
    & |\Im \rho| \ \text{divide} \ |G| = 22k\\
    & |\Im \rho| \ \text{divide} \ 10!
\end{align*}

If $G$ has finite order, then assume for contradiction $|\ker \rho | =1$, then $|G| = |\Im \rho|$ divide $10!$, but $|G|$ has a factor of 11. Contradiction. So $\ker \rho >1$. \\

If $G$ has infinite order, then since $|\Im \rho| \leq 10! < \infty$, $|\ker \rho|$ must be infinite. 


\section*{Q 3}
\begin{question}
Let $\C^{\times}$ denote the multiplicative group of all non-zero complex numbers (under the ordinary multiplication). Prove that $\C^{\times}$ does not have any non-trivial subgroup of finite index.
\end{question}
Supp that $H$ has finite index in $\C^{\times}$ and $m = [\C^{\times}:H]$. Then for any nonzero complex number $z^m \in H$, we have
\begin{align*}
    z^m H = (zH)^m = H, \implies z^m \in H.
\end{align*} For all $w \in \C$, we can solve $z^m -w =0$ to write $w$ as $z^m$ for some $z$, hence $w\in H$, so $\C \subseteq H$, so $H = \C$. 


\section*{Q 4}
\begin{proposition}
Let a group $G$ have order $2^2 \cdot 5 \cdot 17$. Show that 
\begin{enumerate}
    \item $G$ has a unique Sylow $5$-subgroup and a unique Sylow $17$-subgroup.
    \item $\exists$ an element of order $85 = 5 \cdot 17$ in $G$.
\end{enumerate}

\end{proposition}

\textbf{Soln 1: }
$\lvert G \rvert = 2^2 \cdot 5 \cdot 17$. Let $n_5 = \#$ of Sylow $5$ subgroup.
By Sylow's theorem, we have:
\begin{align}
    &n_5 \equiv 1 \mod 5\\
    &n_5 \ \vert \ 2^2 \cdot 17
\end{align}
hence, $n_5$ can be $1,2,4,17,34, \ 4\cdot 17 = 68$. 
Only $1 \equiv 1 \mod 5$ among them. Thus, $n_5 = 1$, so Sylow 5 subgroup is unique.\\

Let $n_{17} = \#$ of Sylow $17$ subgroup. We have
\begin{align}
    &n_{17} \equiv 1 \mod 17\\
    &n_{17} \ \vert \ 2\cdot 2 \cdot 5
\end{align}
Thus, $n_{17}$ can be $1,2,5,4,10,20$, and only $1$ satisfy
\begin{align}
    n_{17} \equiv 1 \mod 17
\end{align} among them. So $n_{17} = 1$, and the Sylow 17 subgroup is unique.

\textbf{Soln 2: }
Since there exists unique Sylow 5 subgroup $=: H$
and there exists unique Sylow 17 subgroup $=: N$. 
We know that $H \triangleleft G, \ N \triangleleft G$, so we know that $H\cdot N$ is a group (as long as one of $H$ or $N$ is normal).
\begin{align*}
    & H \cdot N \leq G\\
    & |H| = 5\\
    & |N| = 17\\
\end{align*}
Thus, since there are of prime order, they are cyclic. 
$$\exists h \in H, \ |h|=5$$ (take the generator for example.)
$$ \exists n \in N, \ |n|=17.$$
Also, $H \cap N = \{ 1 \}$, since for cyclic group of prime order, every element $\neq 1$ has same order, 
so if $1\neq x \in H \cap N$, then $x \in H$, $x$ has order $5$, but $x \in N$ implies $|x|$ divide $N=17$, but $|x| = 5 \nmid 17$.\\

$N \cdot N$ is the internal direct product of $N$ and $N$, 
\begin{align}
    H \cdot N \cong H \times N \cong \Z^{+}_5 \times \Z^{+}_{17} \cong \Z^+_{5\cdot 17} = \Z^+_{85}
\end{align}
Since $5$ and $17$ are coprime, so $\exists$ an element $x \in H \cdot N$ of order $85$,
since Isomorphism preserve the order, so $x \in H \cdot N \leq G$ implies $x \in G,$ and $x$ has order $85$. 





\section*{Q 5}
\begin{proposition}
    Let $Z \trianglelefteq G$ such that $ \lvert Z \rvert = 2$ and $ \lvert G:Z \rvert =97 $. Show that
    \begin{enumerate}
        \item $Z \leq Z(G)$
        \item $G$ is cyclic.
    \end{enumerate}
    
\end{proposition}

\textbf{soln to 1:}\\
$Z \triangleleft G$ and $|Z| = 2$. $97$ is prime. $G$ act on $Z$ by conjugation.
\begin{align}
    &G \xmapsto{\rho} Aut(Z)\\
    &g \mapsto g^{\rho}
\end{align}
$g^{\rho} \in Aut(Z)$ so $g^{\rho}: Z \rightarrow Z$ is isomorphism from $Z$ to $Z$, 

\begin{align}
    & g^{\rho}: Z \rightarrow Z\\
    & z \rightarrow g^{-1} z g \in Z, 
\end{align} since $Z \triangleleft G$.
Recall that ($g$ in the center of $G$) $g \in Z(G) \iff g^{\rho}$ is trivial $\iff g^{\rho} = id.$  \\
WTS: $x^{\rho} = id \ \forall \ x \in G$ \\
We have
\begin{align}
    & x^{\rho} \in Im \rho \leq Aut(Z) \\
    & |Aut(z)| = 1 \ \text{since} \ |z|=2
\end{align}
Thus, $|x^{\rho}|$ divides $|Aut(z)|=1$.\\
This implies that $x^{\rho} = id$, and any $x \rightarrow x^{\rho}: \ x^{-1} z x = z$ for $z \in Z$.\\
$1 \in Z(G)$ automatically, so $Z \subset Z(G)$.

\textbf{soln to 2 (not complete):}\\
We have $Z \triangleleft G$ and $|G/Z| = 97$. Since $|Z|=2$ is prime, $Z$ is cyclic.\\
$G/Z$ is cyclic? $G$ abelian? See HW01.
\begin{align}
    & G \xmapsto{\eta} G/Z\\
    & g \mapsto gz
\end{align}
Consider the following map?
\begin{align}
    f: G &\rightarrow Z \subset Z(G)\\
    g &\rightarrow g^{97}
\end{align}
so we have 
\begin{align}
    Z g^{97} = (Z g)^{97} = Z, \ \implies g^{97} \in Z.
\end{align}
Do we have 
\begin{align}
    G \cong \Z_2 \otimes \Z_{97} \cong \Z_{194} ?
\end{align}














\end{document}
