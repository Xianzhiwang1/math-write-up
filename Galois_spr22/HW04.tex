\documentclass[12pt,english]{article}
\usepackage{xianzhi}

\title{Galois Theory HW04}
\author{Xianzhi}
\date{Due March 11, 2022}

\begin{document}

\maketitle

\section*{Exercise 5.3.9}
\begin{question}
    Is the polynomial $X^4 -2$ irreducible over the field $\Q(\sqrt{3})$?
\end{question}

\textbf{Soln}

Assume $X^4 - 2$ is reducible over $\Q(\sqrt{3})$.
Then $X^4 -2$ either factor into 1 degree one factor and 1 degree three factor, or 
factor into 2 degree two factor (factor means polynomial).

\textbf{Case 1}

$X^4 -2$ has a degree one factor in $\Q(\sqrt{3})$. 
so it has a root in $\Q(\sqrt{3})$. The roots of $X^4 - 2$ are
\begin{align}
    \sqrt[4]{2}, \ \sqrt[4]{2}, \ - \sqrt[4]{2}, \ - i \sqrt[4]{2}. 
\end{align}
since $\Q (\sqrt{3})$ is a degree $2$ extension, 
(Because $\sqrt{3}$ has minimal polynomial $X^2 -3$, which is irreducible by Eisenstein,)




\section*{Exercise 6.4.6}
\begin{question}
    Let $L$ be the splitting field over $\Q$ of $X^5 - 2$ over $\Q$. 
    Show that the Galois group $G:=\Gamma(L:\Q)$ has order $20$,
    and $G$ has a normal subgroup $N$ with $\lvert N \rvert = 5$
    such that the factor group $G/N$ is cyclic.
\end{question}

\section*{Exercise 6.4.7}
\begin{question}
    Let $p$ be an irreducible polynomial over a subfield $K$ of $\C$, and denote by $L$ the splitting field of $p$ over $K$. 
    Show that if the Galois group $\Gamma (L:K)$ is abelian (i.e. commutative), 
    then its order equals the degree of $p$.
\end{question}

\begin{proof}
    Let $p$ be irreducible polynomial over $K \subseteq \C$.
    Let $L$ be the splitting field of $p$ over $K$.
    Let $\alpha$ be a root of $p$.
    Let $m = m_K^{\alpha}$ be the minimal polynomial having $\alpha$ as a root over $K$.
    Then $m$ divide $p$. But $p$ is already irreducible, so we conclude that $m = p$. 
    (We can assume $p$ is monic, because if not, we could scale by a constant
    from $K$ to make it monic.)
    Since $L$ is the splitting field of $p$ over $K$, and $K \subseteq L \subseteq \C$, 
    so $p$ has no multiple roots in $L$, we apply the equivalence theorem
    to say $L$ of $K$ is a Galois extension. Since $\Gamma(L:K)$ is abelian,
    all subgroups are normal. We apply Galois correspondence.
    \begin{align}
        \Gamma(K(\alpha):K) \cong \Gamma(L:K)/\Gamma(L:K(\alpha))
    \end{align}
    and $K(\alpha):K$ is Galois extension by Galois correspondence.
    so $K(\alpha):K$ is normal and separable.
    Thus, since we established $m_K^{\alpha} = p$, $K(\alpha)$ is normal,
    so $K(\alpha)$ contain all the roots of $m_K^{\alpha}=p$, 
    so $K(\alpha) \supset L,$ and since $K(\alpha) \subseteq L$,
    we conclude $K(\alpha) = L$. Thus,
    \begin{align}
        \lvert \Gamma(L:K) \rvert = [L:K] = [K(\alpha):K] = deg \ m_K^{\alpha} = deg \ p
    \end{align}
    and the first equal sign is because extension is Galois.
    
\end{proof}





\section*{A question from HW02}
\begin{question}
Show number of automorphisms of a finite degree field extension divides the degree of the field extension. 
\end{question}
Let $K \subset L, L:K$ be a finite degree field extension. Recall
\begin{align*}
    \Gamma(L:K) &= \{g \in \Gamma(L): g(x) = x \quad \forall x \in K\}\\
    \text{WTS:} \quad |\Gamma(L:K)| &\mid [L:K].
\end{align*} Recall Artin's theorem, let $\Gamma(L:K)$ be the finite subgroup. (Since $|\Gamma(L:K)|$ is bounded by $[L:K]<\infty$.) and 
\begin{align*}
    M = \{x \in L: \forall g \in \Gamma(L:K): g(x)= x\}
\end{align*} so $K \subset M$, and $[L:M]=|\Gamma(L:K)|$. Thus, consider $K \subset M \subset L$,
\begin{align*}
    [L:K] = [L:M][M:K]
\end{align*} where $[L:M] = |\Gamma(L:K)|$, so $|\Gamma(L:K)|$ divides $[L:K]$.










\end{document}
