\documentclass{article}

%
% Homework Details
%   - Title
%   - Due date
%   - Class
%   - Section/Time
%   - Instructor
%   - Author
%

\newcommand{\hmwkTitle}{AAA \ \#HW03}
\newcommand{\hmwkDueDate}{2022}
\newcommand{\hmwkClass}{Adv Abstract Algebra}
\newcommand{\hmwkClassTime}{Spr 2022}
\newcommand{\hmwkClassInstructor}{Prof. Peter Hermann}
\newcommand{\hmwkAuthorName}{\textbf{Xianzhi} }
\newcommand{\hmwkAuthor}{\textit{Xianzhi Wang}}


\usepackage{fancyhdr}
\usepackage{extramarks}
\usepackage{amsmath}
\usepackage{amsthm}
\usepackage{amsfonts}
\usepackage{tikz}
\usepackage[plain]{algorithm}
\usepackage{algpseudocode}

\usetikzlibrary{automata,positioning}

%
% Basic Document Settings
%

\topmargin=-0.45in
\evensidemargin=0in
\oddsidemargin=0in
\textwidth=6.5in
\textheight=9.0in
\headsep=0.25in

\linespread{1.1}

\pagestyle{fancy}
\lhead{\hmwkAuthorName}
\chead{\hmwkClass\ (\hmwkClassInstructor\ \hmwkClassTime): \hmwkTitle}
\rhead{\firstxmark}
\lfoot{\lastxmark}
\cfoot{\thepage}

\renewcommand\headrulewidth{0.4pt}
\renewcommand\footrulewidth{0.4pt}

\setlength\parindent{0pt}

%
% Create Problem Sections
%

\newcommand{\enterProblemHeader}[1]{
    \nobreak\extramarks{}{Problem \arabic{#1} continued on next page\ldots}\nobreak{}
    \nobreak\extramarks{Problem \arabic{#1} (continued)}{Problem \arabic{#1} continued on next page\ldots}\nobreak{}
}

\newcommand{\exitProblemHeader}[1]{
    \nobreak\extramarks{Problem \arabic{#1} (continued)}{Problem \arabic{#1} continued on next page\ldots}\nobreak{}
    \stepcounter{#1}
    \nobreak\extramarks{Problem \arabic{#1}}{}\nobreak{}
}

\setcounter{secnumdepth}{0}
\newcounter{partCounter}
\newcounter{homeworkProblemCounter}
\setcounter{homeworkProblemCounter}{1}
\nobreak\extramarks{Problem \arabic{homeworkProblemCounter}}{}\nobreak{}

%
% Homework Problem Environment
%
% This environment takes an optional argument. When given, it will adjust the
% problem counter. This is useful for when the problems given for your
% assignment aren't sequential. See the last 3 problems of this template for an
% example.
%
\newenvironment{homeworkProblem}[1][-1]{
    \ifnum#1>0
        \setcounter{homeworkProblemCounter}{#1}
    \fi
    \section{Problem \arabic{homeworkProblemCounter}}
    \setcounter{partCounter}{1}
    \enterProblemHeader{homeworkProblemCounter}
}{
    \exitProblemHeader{homeworkProblemCounter}
}


%
% Title Page
%

\title{
    \vspace{2in}
    \textmd{\textbf{\hmwkClass:\ \hmwkTitle}}\\
    \normalsize\vspace{0.1in}\small{Due\ on\ \hmwkDueDate\ at 11:59PM}\\
    \vspace{0.1in}\large{\textit{\hmwkClassInstructor\ \hmwkClassTime}}
    \vspace{3in}
}

\author{\hmwkAuthorName}
\date{2023}

\renewcommand{\part}[1]{\textbf{\large Part \Alph{partCounter}}\stepcounter{partCounter}\\}

%
% Various Helper Commands
%

% Useful for algorithms
\newcommand{\alg}[1]{\textsc{\bfseries \footnotesize #1}}

% For derivatives
\newcommand{\deriv}[1]{\frac{\mathrm{d}}{\mathrm{d}x} (#1)}

% For partial derivatives
\newcommand{\pderiv}[2]{\frac{\partial}{\partial #1} (#2)}

% Integral dx
\newcommand{\dx}{\mathrm{d}x}

% Alias for the Solution section header
\newcommand{\solution}{\textbf{\large Solution:}}

% Probability commands: Expectation, Variance, Covariance, Bias
\newcommand{\E}{\mathrm{E}}
\newcommand{\Var}{\mathrm{Var}}
\newcommand{\Cov}{\mathrm{Cov}}
\newcommand{\Bias}{\mathrm{Bias}}

% From xianzhi.sty

% Fancy
\newcommand{\cA}{\mathcal A}
\newcommand{\cB}{\mathcal B}
\newcommand{\cC}{\mathcal C}
\newcommand{\cD}{\mathcal D}
\newcommand{\cE}{\mathcal E}
\newcommand{\cF}{\mathcal F}
\newcommand{\cG}{\mathcal G}
\newcommand{\cH}{\mathcal H}
\newcommand{\cI}{\mathcal I}
\newcommand{\cJ}{\mathcal J}
\newcommand{\cL}{\mathcal L}
\newcommand{\cM}{\mathcal M}
\newcommand{\cN}{\mathcal N}
\newcommand{\cO}{\mathcal O}
\newcommand{\cP}{\mathcal P}
\newcommand{\cR}{\mathcal R}
\newcommand{\cS}{\mathcal S}
\newcommand{\cT}{\mathcal T}
\newcommand{\cU}{\mathcal U}
\newcommand{\cW}{\mathcal W}
\newcommand{\cX}{\mathcal X}
\newcommand{\cY}{\mathcal Y}


\newcommand{\bP}{\mathbb{P}}

\newcommand{\C}{\mathbb{C}}
\newcommand{\R}{\mathbb{R}}
\newcommand{\N}{\mathbb{N}}
\newcommand{\Q}{\mathbb{Q}}
\newcommand{\Z}{\mathbb{Z}}
\newcommand{\F}{\mathbb{F}}

% Brackets
\newcommand{\bigp}[1]{\left( #1 \right)} % (x)
\newcommand{\bigb}[1]{\left[ #1 \right]} % [x]
\newcommand{\bigc}[1]{\left\{ #1 \right\}} % {x}
\newcommand{\biga}[1]{\left\langle #1 \right\rangle} % <x>

%norm

% theorem 
\newtheorem{Proposition}{proposition}
\newtheorem{Assumption}{assumption}
\newtheorem{Definition}{definition}
\newtheorem{Corollary}{corollary}
\newtheorem{Question}{question}



% \begin{document}

% \usepackage{xianzhi}

\begin{document}

\maketitle
Homework set 3

\pagebreak

\begin{homeworkProblem}
    Suppose that a group $G$ has order $312$. 
    Prove that $G$ has a proper normal subgroup.\\
    \solution

    $312 = 2^3 \times 3 \times 13$.\\
    The number of Sylow $p=2$ subgroup, $n$, has several possibilities 
    \begin{align}
        1,3,13,39 &\equiv \mod p=2
    \end{align}
    and they divide $m = 3 \times 13$.\\
    The number of Sylow $p = 3$ subgroup, $n$ has several possibilities.
    \begin{align}
        1,4,13 &\equiv 1 \mod p=3\\
    \end{align}
    and they divide $m = 8 \cdot 13$.\\
    However, the number of Sylow $p=13$ subgroup
    is one, since $1$ is the only number $\equiv 1 \mod p=13$
    and divide $m = 2^3 \cdot 3$ at the same time.\\
    By Sylow's theorem, (and corollary)
    $G$ has a proper normal subgroup of order $13$.\\
    The unique Sylow $13$ subgroup.\\




\end{homeworkProblem}

\pagebreak

\begin{homeworkProblem}
    Suppose that a group $G$ has order $1960$. Prove that $G$ has a proper normal subgroup.\\
    \solution

    Suppose a group has order $1960 = 2^3 \cdot 5 \cdot 7^2$, 
    the number of Sylow $p=2$ subgroup has several possibilites 
    \begin{align}
        1,5,7 \equiv 1 \mod p = 2
    \end{align}
    and divide $m = 5 \cdot 7 \cdot 7$.\\
    The number of Sylow $p=5$ subgroup has several possibilities, 
    for example, $1,56, 196 \equiv 1 \mod p = 5$
    and divide $m = 2^3 \cdot 7^2$.\\
    But the number $n$ of Sylow $p = 7$ subgroup has $2$ 
    possibilities 
    \begin{align}
        1,8 &\equiv 1 \mod p = 7\\
    \end{align}
    and divide $m = 2^3 \cdot 5 = 40$.\\
    Suppose $n = 8$. (If $n=1$, then we are done)\\
    Since $G$ acts on $Syl_7 (G)$ by conjugation 
    and the action is transitive, $G$ is 
    essentially permuting the $8 Syl_7 (G)$ subgroups.\\
    Thus, we could define a homomorphism
    \begin{align}
        G \xrightarrow[]{\psi} S_8
    \end{align}
    ($\psi$ is indeed a homomorphism 
    because of the definition of group action)\\
    $ker \psi$ cannot be the whole group $G$, 
    since $G$ acts transitively on the 
    $Syl_7 (G)$ groups, and we are assuming 
    there is $8$ of them, 
    so $\psi$ cannot map everything in $G$ to the identity permutation.\\

    Assuming $ker \psi = \left\{ e \right\}$, $\implies \psi$ is one to one,
    $\lvert G \rvert = \lvert Im \psi \rvert$ need to divide 
    $\lvert S_8 \rvert$ since $Im \psi \leq S_8$.\\
    So $\lvert G \rvert$ need to divide $\lvert S_8 \rvert$.\\
    But $\lvert G \rvert = 2^3 \cdot 5 \cdot 7^2$.
    and $\lvert S_8\rvert = 8!$.
    Contradiction.\\

    Thus $ker \psi$ is not the whole group
    $G$ and is not just the identity,
    so $ker \psi \lhd G$
    is the proper normal subgroup we seek.
    


    
\end{homeworkProblem}

\pagebreak

\begin{homeworkProblem}
For $A \leq G, \ \lvert G:A \rvert$ finite and $A$ abelian,
let $\tau_{G \mapsto A}$ denote the transfer 
homomorphism from $G$ to $A$. Let $g \in G$
and $b \in N_G(A)$. Show that 
$\tau_{G \mapsto A} (g)$ commutes with $b$.\\
Hint: If $ h_1, \ldots, h_{n} $ is a set of right coset representatives 
of $A$ then show that $ bh_1, \ldots, bh_n $ is also a set of right
coset representatives of $A$.\\

\solution

Let $G = \dot{\coprod}_{i=1}^{n} A h_i$ and $b \in N_G(A) = \left\{ g\in G: g A = A g\right\} $,\\
then $b^{-1} A b = A \implies Ab = bA$.\\
Thus 
\begin{align}
    G &= b G = b \dot{\coprod}_{i=1}^{n} A h_i \\
    &= \dot{\coprod}_{i=1}^n b A h_i\\
    &= \dot{\coprod}_{i=1}^{n} A b h_i\\
\end{align}
since $Ab = bA$ set wise, not element wise.\\
so if $h_i$'s are coset representatives,
then $bh_i$'s are also coset representatives.\\
Now, let $g \in G$, we have 
\begin{align}
    A h_i \cdot g &= A h_{i^g} \\
    \implies &\exists a_{i,g} \ \forall i \ \text{such that}\\
    h_i \cdot g &= a_{i,g} \cdot h_{i^g}\\
    h_i \cdot g \cdot h_{i^g} &= a_{i,g}
\end{align}
similarly,
\begin{align}
    A b \cdot h_i \cdot g &= A b \cdot h_{i^g} \implies \exists a^{*}_{i,g} \ \forall i \ \text{such that}\\
    b \cdot h_i \cdot g &= a^{*}_{i,g} \cdot b \cdot h_{i^g}\\
    b \cdot \underbrace{h_i \cdot g \cdot h_{i^g}}_{a_{i,g}} \cdot b^{-1} &= a^{*}_{i,g} \in A\\
    b a_{i,g} b^{-1} &= a^{*}_{i,g} \ \forall \ i \in \left\{ 1, \ldots, n \right\}\\
    \implies a_{i,g} &= b^{-1} a^{*}_{i,g} b \ \forall \ i
\end{align}
$\left\{ h_i \right\}, \left\{ b h_i \right\}$ are both coset representations\\
the $\tau (g)$ is independent of the representatives we pick,
so $\tau (g) = \prod_{i=1}^{n} a_{i,g} = \prod_{i=1}^{n} a^{*}_{i,g}$. Hence,\\
\begin{align}
    \tau (g) &= \prod_{i=1}^n a_{i,g} \\
    &= \prod_{i=1}^{n} \left( b^{-1} a^*_{i,g} b \right)\\
    &= b^{-1} \left( \prod_{i=1}^n a^*_{i,g} \right) b\\
    &= b^{-1} \left( \tau (g) \right) b
\end{align}
so $b \tau (g) = \tau (g) b$. Thus commute.





    
\end{homeworkProblem}

\pagebreak

\end{document}







