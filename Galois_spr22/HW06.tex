\documentclass{article}

%
% Homework Details
%   - Title
%   - Due date
%   - Class
%   - Section/Time
%   - Instructor
%   - Author
%

\newcommand{\hmwkTitle}{GAL \ \#06}
\newcommand{\hmwkDueDate}{Apr 01, 2022}
\newcommand{\hmwkClass}{Galois Theory}
\newcommand{\hmwkClassTime}{Section 9}
\newcommand{\hmwkClassInstructor}{Prof Matyas Domokos}
\newcommand{\hmwkAuthorName}{\textbf{Xianzhi} }
\newcommand{\hmwkAuthor}{\textit{Xianzhi Wang}}


\usepackage{fancyhdr}
\usepackage{extramarks}
\usepackage{amsmath}
\usepackage{amsthm}
\usepackage{amsfonts}
\usepackage{tikz}
\usepackage[plain]{algorithm}
\usepackage{algpseudocode}

\usetikzlibrary{automata,positioning}

%
% Basic Document Settings
%

\topmargin=-0.45in
\evensidemargin=0in
\oddsidemargin=0in
\textwidth=6.5in
\textheight=9.0in
\headsep=0.25in

\linespread{1.1}

\pagestyle{fancy}
\lhead{\hmwkAuthorName}
\chead{\hmwkClass\ (\hmwkClassInstructor\ \hmwkClassTime): \hmwkTitle}
\rhead{\firstxmark}
\lfoot{\lastxmark}
\cfoot{\thepage}

\renewcommand\headrulewidth{0.4pt}
\renewcommand\footrulewidth{0.4pt}

\setlength\parindent{0pt}

%
% Create Problem Sections
%

\newcommand{\enterProblemHeader}[1]{
    \nobreak\extramarks{}{Problem \arabic{#1} continued on next page\ldots}\nobreak{}
    \nobreak\extramarks{Problem \arabic{#1} (continued)}{Problem \arabic{#1} continued on next page\ldots}\nobreak{}
}

\newcommand{\exitProblemHeader}[1]{
    \nobreak\extramarks{Problem \arabic{#1} (continued)}{Problem \arabic{#1} continued on next page\ldots}\nobreak{}
    \stepcounter{#1}
    \nobreak\extramarks{Problem \arabic{#1}}{}\nobreak{}
}

\setcounter{secnumdepth}{0}
\newcounter{partCounter}
\newcounter{homeworkProblemCounter}
\setcounter{homeworkProblemCounter}{1}
\nobreak\extramarks{Problem \arabic{homeworkProblemCounter}}{}\nobreak{}

%
% Homework Problem Environment
%
% This environment takes an optional argument. When given, it will adjust the
% problem counter. This is useful for when the problems given for your
% assignment aren't sequential. See the last 3 problems of this template for an
% example.
%
\newenvironment{homeworkProblem}[1][-1]{
    \ifnum#1>0
        \setcounter{homeworkProblemCounter}{#1}
    \fi
    \section{Problem \arabic{homeworkProblemCounter}}
    \setcounter{partCounter}{1}
    \enterProblemHeader{homeworkProblemCounter}
}{
    \exitProblemHeader{homeworkProblemCounter}
}


%
% Title Page
%

\title{
    \vspace{2in}
    \textmd{\textbf{\hmwkClass:\ \hmwkTitle}}\\
    \normalsize\vspace{0.1in}\small{Due\ on\ \hmwkDueDate\ at 11:59PM}\\
    \vspace{0.1in}\large{\textit{\hmwkClassInstructor\ \hmwkClassTime}}
    \vspace{3in}
}

\author{\hmwkAuthorName}
\date{2023}

\renewcommand{\part}[1]{\textbf{\large Part \Alph{partCounter}}\stepcounter{partCounter}\\}

%
% Various Helper Commands
%

% Useful for algorithms
\newcommand{\alg}[1]{\textsc{\bfseries \footnotesize #1}}

% For derivatives
\newcommand{\deriv}[1]{\frac{\mathrm{d}}{\mathrm{d}x} (#1)}

% For partial derivatives
\newcommand{\pderiv}[2]{\frac{\partial}{\partial #1} (#2)}

% Integral dx
\newcommand{\dx}{\mathrm{d}x}

% Alias for the Solution section header
\newcommand{\solution}{\textbf{\large Solution:}}

% Probability commands: Expectation, Variance, Covariance, Bias
\newcommand{\E}{\mathrm{E}}
\newcommand{\Var}{\mathrm{Var}}
\newcommand{\Cov}{\mathrm{Cov}}
\newcommand{\Bias}{\mathrm{Bias}}

% From xianzhi.sty

% Fancy
\newcommand{\cA}{\mathcal A}
\newcommand{\cB}{\mathcal B}
\newcommand{\cC}{\mathcal C}
\newcommand{\cD}{\mathcal D}
\newcommand{\cE}{\mathcal E}
\newcommand{\cF}{\mathcal F}
\newcommand{\cG}{\mathcal G}
\newcommand{\cH}{\mathcal H}
\newcommand{\cI}{\mathcal I}
\newcommand{\cJ}{\mathcal J}
\newcommand{\cL}{\mathcal L}
\newcommand{\cM}{\mathcal M}
\newcommand{\cN}{\mathcal N}
\newcommand{\cO}{\mathcal O}
\newcommand{\cP}{\mathcal P}
\newcommand{\cR}{\mathcal R}
\newcommand{\cS}{\mathcal S}
\newcommand{\cT}{\mathcal T}
\newcommand{\cU}{\mathcal U}
\newcommand{\cW}{\mathcal W}
\newcommand{\cX}{\mathcal X}
\newcommand{\cY}{\mathcal Y}


\newcommand{\bP}{\mathbb{P}}

\newcommand{\C}{\mathbb{C}}
\newcommand{\R}{\mathbb{R}}
\newcommand{\N}{\mathbb{N}}
\newcommand{\Q}{\mathbb{Q}}
\newcommand{\Z}{\mathbb{Z}}
\newcommand{\F}{\mathbb{F}}

% Brackets
\newcommand{\bigp}[1]{\left( #1 \right)} % (x)
\newcommand{\bigb}[1]{\left[ #1 \right]} % [x]
\newcommand{\bigc}[1]{\left\{ #1 \right\}} % {x}
\newcommand{\biga}[1]{\left\langle #1 \right\rangle} % <x>

%norm

% theorem 
\newtheorem{Proposition}{proposition}
\newtheorem{Assumption}{assumption}
\newtheorem{Definition}{definition}
\newtheorem{Corollary}{corollary}
\newtheorem{Question}{question}



% \begin{document}

% \usepackage{xianzhi}

\begin{document}

\maketitle
HW06 \\
Apr 01, 2022 \\
Exercise 9.4.1\\
Exercise 9.4.2\\
Exercise 9.4.3\\
\pagebreak

\begin{homeworkProblem}
    \textbf{Exercise 9.4.1} Let $L$ be the splitting field over $\Q$ of a cubic polynomial with rational coefficients,
    and $\omega$ a primitive cubic root of unity. Show that $L(\omega)$ is a radical extension of $\Q$, by exhibiting explicitly a radical sequence.\\
    \textit{(Hint: recall Cardano's Method.)}\\
    \solution \\
    Let $L$ be splitting field over $\Q$ of a cubic polynomial with radical coefficients $aX^3 + bX^2 + cX + d, a,b,c,d \in \Q$.
    WLOG, $L$ is the same splitting field if the polynomial is monic
    \begin{align}
        X^3 + \frac{ b }{ a }X^2 + \frac{ c }{ a }X + \frac{ d }{ a }, 
    \end{align}
    so we could assume $a=1$ from the beginning.\\
    Also, $L$ is the same if we shift by a rational amount $b/3$ of all the roots of this polynomial, because
    \begin{align}
        X^3 + b X^2 + cX+d=(X + \frac{ b }{ 3 })^3 + (c - \frac{ b^2 }{ 3 })(X + \frac{ b }{ 3 }) + d + \frac{ b^3 }{ 9 } - \frac{ cb }{ 3 } - \frac{ b^3 }{ 27 }
    \end{align}
    Thus, we could assume our polynomial is of the form:
    \begin{align}
        X^3 + pX + q, \ p,q \in \Q.
    \end{align}
    $L(\omega)$ is radical extension, since
    \begin{align}
        E := \Q \left(\sqrt{-3}, \sqrt{\frac{ q^2 }{ 4 }+\frac{ p^3 }{ 27 }}, \sqrt[3]{\frac{ -q }{ 2 } + \sqrt{\frac{ q^2 }{ 4 }+\frac{ p^3 }{ 27 }} } \right) = L(\omega). \ (*)
    \end{align}
    LHS of $(*)$ is a radical sequence, since 
    \begin{align}
        & ( \sqrt{-3})^2 \in \Q \\
        & \left( \sqrt{\frac{ q^2 }{ 4 }+\frac{ p^3 }{ 27 }} \right)^2 \in \Q(\sqrt{-3}) \\
        & u^3 := \left( \sqrt[3]{\frac{ -q }{ 2 } + \sqrt{\frac{ q^2 }{ 4 }+\frac{ p^3 }{ 27 }} } \right)^3 \in \Q \left(\sqrt{-3}, \sqrt{\frac{ q^2 }{ 4 }+\frac{ p^3 }{ 27 }} \right) \\
        & v := \frac{ -p }{ 3u } = \sqrt[3]{\frac{ -q }{ 2 } - \sqrt{\frac{ q^2 }{ 4 }+\frac{ p^3 }{ 27 }} }\\
        & \omega := - \frac{ 1 }{ 2 } + \frac{ \sqrt{-3} }{ 2 }
    \end{align}
    Now, we show the equality in $(*)$. By definition of $u,v,\omega$, we have
    \begin{align}
        L(\omega) = \Q (u+v, \omega u + \omega^2 v, \omega^2 u + \omega v, \omega)
    \end{align}
    First, we show $E \subset L(\omega)$, then we show $E \supset L(\omega)$.
    Since $\sqrt{-3} \in L(\omega), \ \sqrt{-3} = 2 \omega + 1$, observe
    \begin{align}
        & u+v, \frac{ \omega u + \omega^2 v }{ \omega } = u + \omega v \in L(\omega)\\
        & \implies (u + \omega v) - (u + v) = (\omega -1)v \in L(\omega)
    \end{align}
    since $\omega - 1 \in L(\omega) \implies \ v \in L(\omega)$, similarly, we have $u \in L(\omega)$ or $u = (u+v) - v \in L(\omega)$. 
    Hence, 
    \begin{align}
    \sqrt{ \frac{ q^2 }{ 4 } + \frac{ p^3 }{ 27 } }= u^3 + \frac{ q }{ 2 }  \in L(\omega)
    \end{align}
    Now, we show the other direction $E \supset L(\omega)$.
    We have $w = - \frac{ 1 }{ 2 } + \frac{ 1 }{ 2 }\cdot \sqrt{-3} \in E$,
    and $u \in E$, by definition, $v = \frac{ -p }{ 3u } \in E$.
    Thus, $L(\omega) \subset E$. Hence, $E = L(\omega)$, 
    and we have shown that $L(\omega)$ is a radical extension.
    



\end{homeworkProblem}

\pagebreak

\begin{homeworkProblem}
    \textbf{Exercise 9.4.2} Let $L$ be the splitting field over $\Q$ of a monic irreducible cubic polynomial $f$ in $\Q[x]$.
    \begin{enumerate}
        \item Show that $\Gamma (L : \Q)$ has order 3 iff the discriminant of $f$ is the square of a rational number. Recall that the discriminant of $f$ is
            \begin{align}
                \prod_{1 \leq i < j \leq 3}(\alpha_i - \alpha_j)^2,
            \end{align}
            where $\alpha_i$ are the complex roots of $f$.
        \item Give an example of a monic cubic polynomial $f$ with $|\Gamma(L:\Q)|=3$.\\
            \textit{You may want to use the fact that the discriminant of} $X^3 + pX + q \in \Q[X]$ \textit{is} $-4p^3 -27q^2$.
            
    \end{enumerate}
    \solution \\
    \part 
    
    $L$ is splitting field over $\Q$ of a monic irreducible cubic polynomial $X^3 + a X^2 + b X + c, \ a,b,c \in \Q,$ let $\alpha_1, \alpha_2,\alpha_3$ denote roots. 
    Then, by Vieta's theorem,
    \begin{align}
        & \alpha_1 + \alpha_2 + \alpha_3 = -a \ \implies \alpha_2 = -a - \alpha_1 - \alpha_3 \\
        & \alpha_1 \alpha_2 + \alpha_2 \alpha_3 + \alpha_1 \alpha_3 = b \ \implies \alpha_2 \alpha_3 - b = - \alpha_1 \alpha_2 - \alpha_1 \alpha_3\\
        & \alpha_1 \alpha_2 \alpha_3 = -c \ \implies \alpha_2 = \frac{ -c }{ \alpha_1 \alpha_3 }
    \end{align}
    
    ``$\Leftarrow$''\\
    Assume discriminant of $f$ is $r^2$ for some $r \in \Q$. Then
    \begin{align}
        &(\alpha_1 - \alpha_2)(\alpha_1 - \alpha_3)(\alpha_2 - \alpha_3) = r\\
        &(\alpha_1^2 - \alpha_2 \alpha_1 - \alpha_1 \alpha_3 + \alpha_2 \alpha_3)(\alpha_2 - \alpha_3) = r\\
        &(\alpha_1^2 + \alpha_2 \alpha_3 + \alpha_2 \alpha_3 - b)(-a - \alpha_1 - 2 \alpha_3) = r\\
        &(\alpha_1^2 + 2 \frac{ -c }{ \alpha_1 \alpha_3 } \cdot \alpha_3 - b) ( a + \alpha_1 + \alpha_3) = -r\\
        &\alpha_3 = \frac{ 1 }{ 2 } \bigp{ \frac{ -r }{ \alpha_1^2 + 2 \frac{ -c }{ \alpha_1 } - b } - a - \alpha_1}
    \end{align}
    Thus, $\Q(\alpha_1)$ already has $\alpha_3$ in it. Since Vieta's equations are symetrical equations, and $(\alpha_1 - \alpha_2)(\alpha_1 - \alpha_3)(\alpha_2 - \alpha_3)$
    is symmetrical except a minus sign for $\alpha_2$ and $\alpha_3$, we can also express $\alpha_2$ using rational numbers and $\alpha_1$, 
    so $\alpha_2, \alpha_3 \in \Q(\alpha_1)$, and $\Q(\alpha_1)$ is degree 3 since $\alpha_1$ is root of a irreducible cubic polynomial.
    $\Gamma(L:\Q)$ is Galois extension since $L$ is splitting field, so
    \begin{align}
        \lvert \Gamma (L:\Q) \rvert = [L:\Q] = [\Q(\alpha_1):\Q] = 3
    \end{align}
    ``$\Rightarrow$''\\
    Assume $\Gamma(L:\Q)$ has order 3. $L$ is splitting field over $\Q$ of a monic irreducible cubic polynomial, so $L:\Q$ is Galois, since in $\C$, 
    irreducible polynomial has no multiple roots. Thus, $\phi \in \Gamma (L:\Q)$ acts transitively on the roots, which we call $\alpha_1, \alpha_2, \alpha_3$.
    (take $\phi \neq id$. If $\phi (\alpha_1) = \alpha_2$, then $\phi (\alpha_2) = \alpha_3$, since if $\phi (\alpha_2) = \alpha_1$, then
    $\phi^2(\alpha_1) = \alpha_1$, then $\phi$ has order 2, which does not divide 3.)
    \begin{align}
        \phi_1(\alpha_1) &= \alpha_2 \\
        \phi_1 (\alpha_2) &= \alpha_3 \\
        \phi_1 (\alpha_3) &= \alpha_1,
    \end{align}
    and 
    \begin{align}
        \phi (\alpha_1) &= \alpha_3\\
        \phi (\alpha_3) &= \alpha_2 \\
        \phi (\alpha_2) &= \alpha_1,
    \end{align}
    we have 
    \begin{align}
        &\Gamma (L : \Q) \leq S_3, \\
        &\lvert \Gamma (L:\Q) \rvert = 3, \\
        \Gamma (L:\Q) &\cong A_3 = \{ id, (123), (132) \} 
    \end{align}
    so $\Gamma (L:\Q)$ consists of identity automorphism, and automorphism that do not fix any roots, the action is faithful. 
    Since the determinant is symmetric, 
    \begin{align}
        ((\alpha_1-\alpha_2)(\alpha_1-\alpha_3)(\alpha_2 - \alpha_3))^2
    \end{align}
    is fixed by elements of $\Gamma (L:\Q)$. So the discriminant is in the fixed field of $\Gamma (L:\Q)$,
    so the discriminant is in $\Q$.

    \textit{Comment from Professor: the problem was to show that the square root of the discriminant is in} $\Q$.

    \part 

    example: $X^3 - 3 X + 1$
    \begin{align}
        -4 (-3)^3 - 27 \cdot 1 = 4 \cdot 27 - 27 = 81
    \end{align}
    by the first part of the problem, $\lvert \Gamma (L:\Q) \rvert = 3$.\\
    \textit{Comment: Why is this polynomial irreducible?}

\end{homeworkProblem}

\pagebreak

\begin{homeworkProblem}
    \textbf{Exercise 9.4.3} Let $L$ be a subfield of $\C$ such that $\Gamma(L)$ is the dihedral group $D_4$ (having 8 elements), 
    and $L$ a Galois extension of $\Q$. Show that $L$ is a radical extension of $\Q$.\\
    \begin{proof}
    Let us denote the dihedral group $D_4$ this way:
    \begin{align}
        D_4 = \langle f,t \vert \ f^4 = 1 = t^2, f t = t f^3 \rangle
    \end{align}
    We know degree 2 extension is obtained by adjoining an square root from previous homework.
    Since $\Gamma(L:\Q)$ is $\cong D_4$, and is Galois, we can use Galois Correspondence. Normal subgroup corresponds 
    to normal (Galois) extension.
    \begin{align}
        \begin{matrix}
            \{e\} & \triangleleft & \langle t \rangle & \triangleleft & \langle t, f^2 \rangle & \triangleleft  & D_4 \\
            \updownarrow & & \updownarrow & & \updownarrow & & \updownarrow \\ 
            L & \supset & E & \supset & F & \supset & \Q \\
        \end{matrix}
    \end{align}
    Where the up-down arrow $\updownarrow$ indicates the relationship being the field is fixed field of the group. 
    Each of the subgroup has index $2$ in the previous one, and it's a subnormal chain. So we have
    \begin{align}
       \Gamma (F:\Q) \cong \Gamma (L: \Q) / \Gamma(L:F) = \lvert D_4 / \langle t, f^2 \rangle \rvert = 2
    \end{align}
    So $\Gamma(F:\Q)$ has order 2, so $F:\Q$ has degree 2. So $F=\Q(\alpha)$ where $\alpha^2 \in \Q$.
    Similarly, 
    \begin{align}
        \Gamma(E:F) \cong \Gamma(L:F)/\Gamma(L:E) = \lvert \langle t, f^2 \rangle / \langle t \rangle \rvert = 2
    \end{align}
    So $\lvert \Gamma (E:F) \rvert = 2$ implies that $E:F$ has degree 2, so $E = F(\beta)$, where $\beta^2 \in F$.\\
    $\Gamma (L:E) \cong \langle t \rangle$, $\lvert \langle t \rangle \rvert = 2$ implies that $L:E$ has degree 2.\\
    $ \implies L = E( \gamma)$, where $\gamma^2 \in E$.\\
    $\implies L = \Q (\alpha, \beta, \gamma)$, so $L$ is a radical extension.

        
    \end{proof}
    


\end{homeworkProblem}

\pagebreak


% Go back to where we left off
\begin{homeworkProblem}[6]
    Evaluate the integrals
    \(\int_0^1 (1 - x^2) \dx\)
    and
    \(\int_1^{\infty} \frac{1}{x^2} \dx\).
\end{homeworkProblem}

\end{document}
