\documentclass[12pt]{article}
\usepackage{amsfonts}
\usepackage{fancyhdr}
\usepackage[a4paper, top=2.5cm, bottom=2.5cm, left=2.2cm, right=2.2cm]{geometry}
\usepackage{times}
\usepackage{amsmath}
\usepackage{amssymb}
\usepackage{graphicx}

\newtheorem{theorem}{Theorem}
\newtheorem{acknowledgement}[theorem]{Acknowledgement}
\newtheorem{algorithm}[theorem]{Algorithm}
\newtheorem{axiom}{Axiom}
\newtheorem{case}[theorem]{Case}
\newtheorem{claim}[theorem]{Claim}
\newtheorem{conclusion}[theorem]{Conclusion}
\newtheorem{condition}[theorem]{Condition}
\newtheorem{conjecture}[theorem]{Conjecture}
\newtheorem{corollary}[theorem]{Corollary}
\newtheorem{criterion}[theorem]{Criterion}
\newtheorem{definition}[theorem]{Definition}
\newtheorem{example}[theorem]{Example}
\newtheorem{exercise}[theorem]{Exercise}
\newtheorem{lemma}[theorem]{Lemma}
\newtheorem{notation}[theorem]{Notation}
\newtheorem{problem}[theorem]{Problem}
\newtheorem{proposition}[theorem]{Proposition}
\newtheorem{remark}[theorem]{Remark}
\newtheorem{solution}[theorem]{Solution}
\newtheorem{summary}[theorem]{Summary}
\newenvironment{proof}[1][Proof]{\textbf{#1.} }{\ \rule{0.5em}{0.5em}}
\input{tcilatex}

\begin{document}

\title{Sheet 3: The Adjacency matrix}
\author{Xianzhi Wang}
\date{fall 2021}
\maketitle

All graphs are finite on this sheet. Let $G$ be a directed graph on the
vertex set $\{1,\ldots,n\}$. Let us define the \emph{adjacency matrix} $A=%
\mathrm{Adj}(G)$ by setting 
\begin{equation*}
A_{i,j}=\text{number of edges from }i\text{ to }j\text{ in }G\text{.}
\end{equation*}
So, we allow multiple edges and even loops in $G$.

\begin{exercise}
Express the following in linear algebra terms, using $A$: \newline
1)\ the degrees of a vertex;\newline
2) the number of edges in $G$; \newline
3) $e(X,Y)$ for $X,Y\subseteq V(G)$.
\end{exercise}

One of the main reasons why we look at the adjacency (or neighboring)
relation as a matrix is the following correspondance between matrix
multiplication and walks in $G$.

\begin{definition}
A \emph{directed walk of length }$n$ in $G$ is a sequence of directed edges $%
e_{1},\ldots,e_{n}$ such that $e_{i}^{+}=e_{i+1}^{-}$ ($1\leq i\leq n-1$).
The walk is a \emph{loop (or returning)}, if $e_{1}^{-}=e_{n}^{+}$.
\end{definition}

Note that we redefine the notion of walk here: it is a sequence of edges
rather than vertices.

\begin{theorem}
For every $k>0$, $(A^{k})_{i,j}$ equals the number of directed walks of
length $k$ from $i$ to $j$.
\end{theorem}

\begin{proof}
    Let's prove by induction.\\
    When $k=1, \ (A)_{ij}$ is indeed the number of
    directed walks of length $1$,
    (which is just directed edge) from $i$ to $j$ by definition.\\
    Assume true for $k-1$.\\
    Denote the $ij$th entry of $A$ by $a_{ij}$ and $ij$th 
    entry of $A^{k-1}$ by $b_{ij}$
    \begin{align}
        (A^k)_{i,j} = (A \cdot A^{k-1})_{i,j} = \sum_{\alpha = 1}^{n} a_{i \alpha} b_{\alpha j}
    \end{align}
    Thus, for fixed $\alpha$, we multiply together $a_{i \alpha} b_{\alpha j}$,
    which means multiply number of directed walks of length $1$
    (which are directed edges) from $i$ to $\alpha$ and the 
    number of directed walks of length $k-1$ from $\alpha$ to $j$.
    After this, we sum over $\alpha$, which run through $1$ to $n$,
    and the result $\sum_{\alpha = 1}^{n} a_{i \alpha} b_{\alpha j}$
    is indeed number of directed walks of length $k$.\\
    Thus, we have proved the theorem using induction.
\end{proof}


\begin{corollary}
For every $k>0$, the trace $\mathrm{tr}(A^{k})$ equals the number of loops
of length $k$ in $G$.
\end{corollary}

\begin{proof}
    Since $(A^k)_{ii}$ equals the number of directed walks
    of length $k$ from $i$ to $i$, (i.e., a loop),
    we could sum over the $n$ entries in the diagonal
    to obtain all the loops of length $k$ in $G$.
    This is exactly $tr (A^k)$.
    
\end{proof}


Now assume that $G$ is undirected.

This turns $A$ to be a symmetric real matrix. Using the spectral theorem, it
follows that $A$ admits an orthonormal eigenbasis $b_{0}(G),b_{1}(G),%
\ldots,b_{n-1}(G)$ with real eigenvalues $\lambda_{0}(G)\geq\lambda
_{1}(G)\geq\ldots\geq\lambda_{n-1}(G)$. That is, we have 
\begin{equation*}
Ab_{i}=\lambda_{i}b_{i}\text{ \ }(0\leq i<n). 
\end{equation*}
Note that the $\lambda_{i}$ are well defined, but the $b_{i}$ are not. Also:

\begin{lemma}
The eigenvalues are graph invariants, that is, isomorphic graphs have the
same eigenvalues.
\end{lemma}
\begin{proof}
    Isomorphic graphs have the same eigenvalues. Since 
    Isomorphic graphs are structurally the same, 
    they have the exact same adjacency matrix, 
    thus the same eigenvalues.
\end{proof}


\begin{exercise}
Compute $\lambda_{i}$ and $b_{i}$ for the triangle.
\end{exercise}

One way to visualize the adjacency matrix as an operator is as follows.
Write real numbers on the vertices of $G$, call this function $f$. Now $A$
acts by taking all neighbors of the vertex $x$, add up the $f$-values there
and write it to the position $x$. This will be the value of $Af$ at $x$: 
\begin{equation*}
(Af)(x)=\dsum \limits_{(x,y)\in E(G)}f(y)\text{.}
\end{equation*}
Actually this is how we will \emph{define} the adjacency operator for
infinite graphs. Using this image, one can prove.

\begin{theorem}
Let $G$ be an undirected graph with maximal degree $d$. Then $\left\vert
\lambda_{i}(G)\right\vert \leq d$ ($0\leq i<n$). When $G$ is $d$-regular, we
have $\lambda_{0}(G)=d$.
\end{theorem}

Hint: take an eigenvector. Find a particular vertex for it..

\begin{proof}
    Let $G$ be undirected graph with max degree $d$.\\
    First we show $\lvert \lambda_k \rvert := \lvert \lambda_k (G) \rvert \leq d$
    for $k \in \left\{ 0, \ldots, n-1 \right\}$.
    Let $\lambda_k$ be an eigenvalue of adjacency matrix 
    $A_G$, and let 
    \begin{align}
        x = 
        \begin{bmatrix}
            x_1\\
            \vdots\\
            x_n
        \end{bmatrix}
    \end{align}
    be a corresponding eigenvector. 
    Choose $x_i$ the max component $\left\{ x_1, \ldots, x_n\right\}$. WLOG $x_i > 0$. 
    We have 
    \begin{align}
        \lvert \lambda_k x_i \rvert &= A \cdot x\\
        &= \lvert \sum^{n}_{j=1} a_{ij}x_j \rvert\\
        &\leq \lvert \sum_{j=1}^{n} a_{ij} x_i \rvert\\
        &\leq \lvert d x_i \rvert \\
        &= \lvert x_i \rvert d
    \end{align}
    since summing over row gives the degree,
    which is less than the degree bound.
    Thus, we have $\lvert \lambda_k \rvert \lvert x_i \rvert \leq \lvert x_i \rvert d$
    so $\lvert \lambda_k \rvert \leq d$ as desired.\\
    Now, when $G$ is $d$-regular $\implies \lambda_0 (G) = d$.\\
    We need an eigenvector that has $d$ as its eigenvalue.\\
    Consider the vector 
    \begin{align}
        x = 
        \begin{bmatrix}
            1\\
            1\\
            \vdots\\
            1\\
        \end{bmatrix}
    \end{align}
    with $n$ copies of $1$.
    Since $G$ is $d$-regular,
    every row of $A_G$ sum to $d$. 
    Thus,
    \begin{align}
        A_G \cdot x =
        \begin{bmatrix}
            a_{11} & \ldots & a_{1n} \\
            \vdots & \ddots & \vdots \\
            a_{n1} & \ldots & a_{n n}\\
        \end{bmatrix} 
        \begin{bmatrix}
            1\\
            \vdots\\
            1\\
        \end{bmatrix} =
        \begin{bmatrix}
            d\\
            \vdots\\
            d\\
        \end{bmatrix} 
    \end{align}
    Thus, $d$ is an eigenvalue.
    Since all eigenvalues are $\leq d$,
    $d$ is $\lambda_0 (G)$, the biggest. 
    
\end{proof}


\begin{exercise}
Assume that $G$ is undirected and connected with maximal degree $d$. Then $%
\lambda_{0}(G)=d$ if and only if $G$ is $d$-regular.
\end{exercise}

\begin{proof}
    $G$ is $d$ regular $\implies \lambda_0 (G) = d$.
    We already proved this in the previous theorem. \\
    Let $G$ undirected, connected, max degree is $d$.
    $\lambda_0(G) = d \implies G$ is $d$-regular. Let 
    \begin{align}
        x = 
        \begin{bmatrix}
            x_1\\
            \vdots\\
            x_n
        \end{bmatrix}
    \end{align}
    be an eigenvector for eigenvalue $d$.
    Again, WLOG, pick a max component $x_i >$ 
    among the entries $x_i, \ldots, x_n$. 
    With $A_G = \left[ a_{ij} \right]$ we have 
    \begin{align}
        \lvert d x_i \rvert &= \lvert \sum_{j=1}^n a_{ij} x_j \rvert\\
        &\leq \lvert \sum_{j=1}^n a_{ij} x_i \rvert\\
        &\leq \lvert d x_i \rvert
    \end{align}
    meaning we have equal signs above.
    so $x$ is a vector of constants of value $x_i$.
    Thus, we could rescale to have 
    \begin{align}
        x =
        \begin{bmatrix}
            1\\
            \vdots\\
            1\\
        \end{bmatrix}
    \end{align}
    so $\sum_{j=1}^n a_{ij} = d$,
    thus, vertex $i$, corresponding the row 
    \begin{align}
        \begin{bmatrix}
            a_{i1} & \ldots & a_{in}\\
        \end{bmatrix}
    \end{align}
    has degree $d$, and since $G$ connected,
    we could apply 
    \begin{align}
        x =
        \begin{bmatrix}
            1\\
            \vdots\\
            1\\
        \end{bmatrix}
    \end{align}
    to other row $i$
    to obtain $v_i$ 
    has degree $d$.
    Thus, $G$ is $d$-regular.
    
\end{proof}


\begin{lemma}
Let $G$ be a $d$-regular undirected graph. Then the multiplicity of $d$ as
an eigenvalue of $G$ equals the number of connected components of $G$.
\end{lemma}

The eigenvalue $-d$ also comes into the picture naturally.

\begin{proof}
    Let $G$ be $d$-regular, undirected, graph that has $n$ vertices.
    $\implies$ multiplicity of $d$ as an eigenvalue of $G = \#$ of connected components of $G$.
    Let $G$ have $i$ connected components $C_i$ with $ i \in \left\{ 1, \ldots, k \right\}$ 
    for some $k$.
    We define vectors $X^i$ corresponding to $C_i$ in this way:
    \begin{align}
        X^i_j &= \begin{cases}
            &1 \ \text{if} \ v_j \in C_i \\
            &0 \ \text{else}
        \end{cases} \\
        j &\in \left\{ 1, \ldots, n \right\}
    \end{align}
    The adjacency matrix of $G$, $A_G$ looks like this:
    \begin{align}
        A_G \cdot X_2 = 
        \begin{bmatrix}
            C_1 & 0 & \ldots & 0\\
            0 & C_2 & & 0\\
            \vdots & & \ddots & \vdots \\
            0 & 0 & & C_n \\
        \end{bmatrix} 
        \begin{bmatrix}
            0\\
            1\\
            \vdots\\
            0
        \end{bmatrix}
        = d
        \begin{bmatrix}
            0\\
            1\\
            \vdots\\
            0
        \end{bmatrix}
    \end{align}
    since $d$-regular row sum to $d$.
    We could see that $X^i$ are eigenvectors 
    of eigenvalue $d$. 
    And $x^i$'s are orthogonal to each other.
    Thus, the multiplicity of $d$
    is at least the number of connected 
    components of $G$.\\
    We want to show that those $X^i$ actually
    span the eigenbasis of $d$, so there are 
    no more eigenvectors of $d$ that we 
    didn't already take into account.\\
    So let $y$ be an eigenvector of $d$,
    we want to express it as a linear combination 
    of $X^i$'s.
    \begin{align}
        i \text{-th row of } A_G \cdot
        \begin{bmatrix}
            y_1\\
            y_2\\
            y_3\\
            \vdots\\
            y_n
        \end{bmatrix} =
        i \text{-th row of }
        d \begin{bmatrix}
            y_1\\
            y_2\\
            y_3\\
            \vdots\\
            y_n
        \end{bmatrix}
    \end{align}
    We assume $\lvert y_i \rvert$
    is the max component in vector $y$, 
    and $y_i \in C_i$.
    \begin{align}
        \sum_{(i,j) \in \text{Edge Set}} y_j = d y_i
    \end{align}
    LHS has $d$ summands, since $G$ is 
    $d$-regular. Also,
    \begin{align}
        \lvert y_j \rvert \leq \lvert y_i \rvert \forall j
    \end{align}
    so in fact all $y_j$ are $y_i$.
    So those underlying vertices $v_j$ are all connected
    to the $v_i$, so they are in component $C_i$.
    Thus, for vector $y$, the ``block'' corresponding 
    to $C_i$ could be expressed as a linear combination 
    of $X^i$, like $\alpha X^i$.
    
    
\end{proof}


\begin{lemma}
Let $G$ be a $d$-regular undirected, connected graph. Then $\lambda_{n-1}=-d 
$ if and only if $G$ is bipartite.
\end{lemma}

\begin{proof}
    We know from Ex 8 that since $G$ is $d$-regular, undirected, 
    connected, $d$ is an eigenvalue. Want To Show:
    $G$ bipartite $\implies -d$ is also eigenvalue. 
    (Then $\lambda_{n-1} = -d$ since $\lvert \lambda_i (G) \rvert \leq d$
    eigenvalue cannot get smaller than $-d$.)\\
    Since $G$ is bipartite, $A_G$ could be expressed as:
    \begin{align}
        \begin{bmatrix} 
            0 & M \\
            M^T & 0
        \end{bmatrix}
    \end{align}
    for some matrix $M$ that is $k \times (n-k)$.
    Now let 
    \begin{align}
        \begin{bmatrix}
            x_1\\
            \vdots\\
            x_k\\
            x_{k+1}\\
            \vdots\\
            x_n
        \end{bmatrix}
        =
        \begin{bmatrix}
            X\\
            Y
        \end{bmatrix}
    \end{align}
    be an eigenvector of eigenvalue $d$.\\
    Therefore, we get
    \begin{align}
        d
        \begin{bmatrix}
            X\\
            Y\\
        \end{bmatrix}
        =
        \begin{bmatrix}
            0 & M\\
            M^T & 0\\
        \end{bmatrix}
        \cdot
        \begin{bmatrix}
            X\\
            Y\\
        \end{bmatrix}
        =
        \begin{bmatrix}
            M Y \\
            M^T X
        \end{bmatrix}
    \end{align}
    Thus:
    \begin{align}
        dX &= MY \\
        dY &= M^T X
    \end{align}
    Claim: $-d$ is eigenvalue with eigenvector 
    \begin{align}
        \begin{bmatrix}
            X\\
            -Y\\
        \end{bmatrix} =
        \begin{bmatrix}
            x_1\\
            \vdots\\
            x_k\\
            -x_{k+1}\\
            \vdots\\
            -x_n
        \end{bmatrix}
    \end{align}
    so we have
    \begin{align}
        \begin{bmatrix}
            0 & M \\
            M^T & 0 \\
        \end{bmatrix}
        \begin{bmatrix}
            X \\
            -Y
        \end{bmatrix}
        =
        \begin{bmatrix}
            M(-Y)\\
            M^T X
        \end{bmatrix}
        =
        \begin{bmatrix}
            -M Y\\
            M^T X
        \end{bmatrix}
        =
        \begin{bmatrix}
            -dX\\
            dY
        \end{bmatrix}
        =
        \begin{bmatrix}
            -dX \\
            -d(-Y)
        \end{bmatrix}
        = -d
        \begin{bmatrix}
            X\\
            -Y
        \end{bmatrix}
        = -d
        \begin{bmatrix}
            x_1\\
            \vdots\\
            x_k\\
            -x_{k+1}\\
            \vdots\\
            -x_n\\
        \end{bmatrix}
    \end{align}
    Thus, $-d$ is eigenvalue as wanted.\\
    $\lambda_{n-1} = -d$,
    $-d$ is also eigenvalue $\implies G$ bipartite.
    
\end{proof}












\begin{exercise}
Let $G$ be a $d$-regular undirected graph. What is the multiplicity of $-d$
as an eigenvalue of $G$?
\end{exercise}

Now we look at some simple examples.

\begin{exercise}
Compute the eigenvalues and eigenvectors for the cycle of length $n$.
\end{exercise}

Hint: What are the eigenvalues for the directed cycle of length $n$?

\begin{exercise}
Compute the eigenvalues and an orthonormal eigenbasis for the complete graph
on $d+1$ points.
\end{exercise}

Hint: What happens to the eigenvalues and eigenvectors of $A$ if you add a
scalar matrix to $A$?

\end{document}
