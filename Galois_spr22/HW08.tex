\documentclass{article}

%
% Homework Details
%   - Title
%   - Due date
%   - Class
%   - Section/Time
%   - Instructor
%   - Author
%

\newcommand{\hmwkTitle}{GAL \ \#08}
\newcommand{\hmwkDueDate}{Apr 22, 2022}
\newcommand{\hmwkClass}{Galois Theory}
\newcommand{\hmwkClassTime}{Section 12}
\newcommand{\hmwkClassInstructor}{Prof Matyas Domokos}
\newcommand{\hmwkAuthorName}{\textbf{Xianzhi} }
\newcommand{\hmwkAuthor}{\textit{Xianzhi Wang}}


\usepackage{fancyhdr}
\usepackage{extramarks}
\usepackage{amsmath}
\usepackage{amsthm}
\usepackage{amsfonts}
\usepackage{tikz}
\usepackage[plain]{algorithm}
\usepackage{algpseudocode}

\usetikzlibrary{automata,positioning}

%
% Basic Document Settings
%

\topmargin=-0.45in
\evensidemargin=0in
\oddsidemargin=0in
\textwidth=6.5in
\textheight=9.0in
\headsep=0.25in

\linespread{1.1}

\pagestyle{fancy}
\lhead{\hmwkAuthorName}
\chead{\hmwkClass\ (\hmwkClassInstructor\ \hmwkClassTime): \hmwkTitle}
\rhead{\firstxmark}
\lfoot{\lastxmark}
\cfoot{\thepage}

\renewcommand\headrulewidth{0.4pt}
\renewcommand\footrulewidth{0.4pt}

\setlength\parindent{0pt}

%
% Create Problem Sections
%

\newcommand{\enterProblemHeader}[1]{
    \nobreak\extramarks{}{Problem \arabic{#1} continued on next page\ldots}\nobreak{}
    \nobreak\extramarks{Problem \arabic{#1} (continued)}{Problem \arabic{#1} continued on next page\ldots}\nobreak{}
}

\newcommand{\exitProblemHeader}[1]{
    \nobreak\extramarks{Problem \arabic{#1} (continued)}{Problem \arabic{#1} continued on next page\ldots}\nobreak{}
    \stepcounter{#1}
    \nobreak\extramarks{Problem \arabic{#1}}{}\nobreak{}
}

\setcounter{secnumdepth}{0}
\newcounter{partCounter}
\newcounter{homeworkProblemCounter}
\setcounter{homeworkProblemCounter}{1}
\nobreak\extramarks{Problem \arabic{homeworkProblemCounter}}{}\nobreak{}

%
% Homework Problem Environment
%
% This environment takes an optional argument. When given, it will adjust the
% problem counter. This is useful for when the problems given for your
% assignment aren't sequential. See the last 3 problems of this template for an
% example.
%
\newenvironment{homeworkProblem}[1][-1]{
    \ifnum#1>0
        \setcounter{homeworkProblemCounter}{#1}
    \fi
    \section{Problem \arabic{homeworkProblemCounter}}
    \setcounter{partCounter}{1}
    \enterProblemHeader{homeworkProblemCounter}
}{
    \exitProblemHeader{homeworkProblemCounter}
}


%
% Title Page
%

\title{
    \vspace{2in}
    \textmd{\textbf{\hmwkClass:\ \hmwkTitle}}\\
    \normalsize\vspace{0.1in}\small{Due\ on\ \hmwkDueDate\ at 11:59PM}\\
    \vspace{0.1in}\large{\textit{\hmwkClassInstructor\ \hmwkClassTime}}
    \vspace{3in}
}

\author{\hmwkAuthorName}
\date{2023}

\renewcommand{\part}[1]{\textbf{\large Part \Alph{partCounter}}\stepcounter{partCounter}\\}

%
% Various Helper Commands
%

% Useful for algorithms
\newcommand{\alg}[1]{\textsc{\bfseries \footnotesize #1}}

% For derivatives
\newcommand{\deriv}[1]{\frac{\mathrm{d}}{\mathrm{d}x} (#1)}

% For partial derivatives
\newcommand{\pderiv}[2]{\frac{\partial}{\partial #1} (#2)}

% Integral dx
\newcommand{\dx}{\mathrm{d}x}

% Alias for the Solution section header
\newcommand{\solution}{\textbf{\large Solution:}}

% Probability commands: Expectation, Variance, Covariance, Bias
\newcommand{\E}{\mathrm{E}}
\newcommand{\Var}{\mathrm{Var}}
\newcommand{\Cov}{\mathrm{Cov}}
\newcommand{\Bias}{\mathrm{Bias}}

% From xianzhi.sty

% Fancy
\newcommand{\cA}{\mathcal A}
\newcommand{\cB}{\mathcal B}
\newcommand{\cC}{\mathcal C}
\newcommand{\cD}{\mathcal D}
\newcommand{\cE}{\mathcal E}
\newcommand{\cF}{\mathcal F}
\newcommand{\cG}{\mathcal G}
\newcommand{\cH}{\mathcal H}
\newcommand{\cI}{\mathcal I}
\newcommand{\cJ}{\mathcal J}
\newcommand{\cL}{\mathcal L}
\newcommand{\cM}{\mathcal M}
\newcommand{\cN}{\mathcal N}
\newcommand{\cO}{\mathcal O}
\newcommand{\cP}{\mathcal P}
\newcommand{\cR}{\mathcal R}
\newcommand{\cS}{\mathcal S}
\newcommand{\cT}{\mathcal T}
\newcommand{\cU}{\mathcal U}
\newcommand{\cW}{\mathcal W}
\newcommand{\cX}{\mathcal X}
\newcommand{\cY}{\mathcal Y}


\newcommand{\bP}{\mathbb{P}}

\newcommand{\C}{\mathbb{C}}
\newcommand{\R}{\mathbb{R}}
\newcommand{\N}{\mathbb{N}}
\newcommand{\Q}{\mathbb{Q}}
\newcommand{\Z}{\mathbb{Z}}
\newcommand{\F}{\mathbb{F}}

% Brackets
\newcommand{\bigp}[1]{\left( #1 \right)} % (x)
\newcommand{\bigb}[1]{\left[ #1 \right]} % [x]
\newcommand{\bigc}[1]{\left\{ #1 \right\}} % {x}
\newcommand{\biga}[1]{\left\langle #1 \right\rangle} % <x>

%norm

% theorem 
\newtheorem{Proposition}{proposition}
\newtheorem{Assumption}{assumption}
\newtheorem{Definition}{definition}
\newtheorem{Corollary}{corollary}
\newtheorem{Question}{question}



% \begin{document}

% \usepackage{xianzhi}

\begin{document}

\maketitle
HW08 \\
Apr 29, 2022 \\
Exercise 12.4.8\\
Exercise 12.4.9\\
Exercise 12.4.10\\
\pagebreak

\begin{homeworkProblem}
    \textbf{Exercise 12.4.8} Factor $X^4 + X + 1 \in \F_{2}[X]$ as a product of irreducibles over $\F_4$.

    \solution 

    $\F_4 = \F_{2^2} = \F_2 (\alpha)$ where $\alpha$ is a root of the irreducible polynomial
    $X^2 + X + 1$. Thus, $\alpha^2 + \alpha + 1 = 0$, so $\alpha^2 = \alpha + 1$. Hence,
    \begin{align}
        \F_2 (\alpha) = \{ 0,1,\alpha,\alpha+1 \}
    \end{align}
    We want to factor $X^4 + X + 1 \in \F_2[X]$. First, we check if it has
    roots in $\F_2 (\alpha)$: 
    \begin{align}
        X &= 1 \implies X^4 + X + 1 = 1\\
        X &= 0 \implies X^4 + X + 1 = 1\\
        X &= \alpha \implies X^4 + X + 1 = (\alpha+1)^2 + \alpha + 1 = \alpha^2 + 1 + \alpha + 1 = \alpha + 1 + \alpha = 1\\
        X &= \alpha + 1 \implies (\alpha + 1)^4 + \alpha + 1 + 1 = (\alpha^2 + 1)^2 + \alpha = \alpha^2 + \alpha = 1
    \end{align}
    so $X^4 + X + 1$ does not have linear factors in $\F_2(\alpha)$, so $X^4 + X + 1$
    can only factor into 
    \begin{align}
        (X^2 + \alpha_1 X + \alpha_0)(X^2 + \beta_1 X + \beta_0)
    \end{align}
    for $\alpha_1, \alpha_0, \beta_1, \beta_0 \in \F_2 (\alpha)$. Thus,
    \begin{align}
        X^4 + X + 1 = X^4 + (\beta_1 + \alpha_1)X^3 + 
        (\alpha_0 + \beta_0 + \alpha_1 \beta_1) X^2 +
        (\alpha_1 \beta_0 + \alpha_0 \beta_1) X +
        \alpha_0 \beta_0
    \end{align}
    Hence,
    \begin{align}
        \beta_1 + \alpha_1 &= 0\\
        \alpha_0 + \beta_0 + \alpha_1 \beta_1 &= 0\\
        \alpha_1 \beta_0 + \alpha_0 \beta_1 &= 1\\
        \alpha_0 \beta_0 &= 1
    \end{align}
    implies 
    \begin{align}
        \alpha_1 &= 1 \\
        \beta_1 &= 1\\
        \alpha_0 &= \alpha\\
        \beta_0 &= \alpha + 1
    \end{align}
    is one possible factorization. Hence,
    \begin{align}
        X^4 + X + 1 = (X^2 + X + \alpha)(X^2 +  X + \alpha + 1)
    \end{align}
    Now, in $\F_2 (\alpha)$, degree $1$ irreducible polynomials are
    \begin{align}
        X, X+1, X+\alpha, X + \alpha + 1
    \end{align}
    degree $2$ NOT irreducible:
    \begin{align}
        &X^2 + X, X^2 + \alpha X, X^2 + (\alpha + 1) X\\
        &X^2 + (\alpha+1)X + \alpha, X^2 + (\alpha+1)X + X + \alpha + 1 = X^2 + \alpha X + \alpha + 1\\
        &X^2 + \alpha X + (\alpha + 1)X + 1 = X^2 + X + 1
    \end{align}
    In $F_2(\alpha)$, we listed all 6 reducible polynomial of degree $2$, since 
    \begin{align}
        X^2 + X + \alpha, \ X^2 + X + \alpha + 1
    \end{align}
    are not among them, we know they must be irreducible. Hence,
    $X^4 + X + 1$ factors as product of irreducible over $\F_4 = \F_2 (\alpha)$.\\

    $\F_4 = \F_{2^2} = \F_2 (\alpha)$ \textit{Why?} Since $\alpha$ is root of 
    degree 2 irreducible polynomial, adjoining $\alpha$ to $\F_2$
    gives degree 2 extension, since $\F_2 (\alpha) : \F_2 = 2$,
    $\F_2 (\alpha)$ has 4 elements, and 
    since the finite field of a prime power ($4$ in this case) is unique,
    we indeed get $\F_4$ by adjoin $\alpha$ to $\F_2$.
    
  

\end{homeworkProblem}

\pagebreak


\begin{homeworkProblem}
    \textbf{Exercise 12.4.9}
    \begin{enumerate}
        \item What is the splitting field of $X^4 + X + 1$ over $\F_{64}$?
        \item Factor $X^4 + X + 1$ into the product of irreducibles over $\F_{64}$.
    \end{enumerate}
    
    \solution 
    
    \part

    Let $\beta$ be root of $X^4 + X + 1$ over $\F_2$.\\
    Let $\alpha$ be root of $X^6 + X + 1$ over $\F_2$.
    \begin{align}
        \F_2 &\subseteq \F_{16} = \F_2 (\beta) \subseteq \F_{2^{12}} = \F_2 (\alpha, \beta)\\
        \F_2 &\subseteq \F_{64} = \F_2 (\alpha) \subseteq \F_{2^{12}} = \F_2 (\alpha, \beta).
    \end{align}
    In class we established that $X^6 + X + 1$ is irreducible over $\F_2$, so let 
    $\alpha$ be a root of $X^6 + X + 1$ over $\F_2$.\\
    Then $\F_{64} = \F_{2}(\alpha)$  since adjoining $\alpha$ gives a degree 6 extension 
    over $\F_2$, and there is only one field of $2^6$ elements up to isomorphism,
    so $\F_2 (\alpha)$ is indeed $\F_{64}$.\\
    We also established that $X^4 + X + 1$ is irreducible over $\F_2$, and
    since finite extension of finite field is Galois, it is normal,
    so adding one root $\beta$ of $X^4 + X + 1$ to $\F_2$ automatically adds
    all the roots. Thus, the splitting field of $X^4 + X + 1$ over $\F_2$
    is $\F_2 (\beta) = \F_{2^4}$ again because $\exists !$ field of 16 elements (up to isomorphism).\\
    
    The splitting field of $X^4 + X + 1$ over $\F_{64}$ must contain $\F_2 (\alpha) = \F_{64}$ and $\beta$
    so the splitting field is $\F_2 (\alpha, \beta)$, (since adding one root
    automatically adds all other roots of $X^4 + X + 1$.) Since
    \begin{align}
        &\F_{p^r} \subseteq \F_{p^s} \ \iff \ r \mid s\\
        & lcm(4,6) = 12,
    \end{align}
    $\F_{2^{12}}$ is the smallest field containing both $\F_{2^4}$ and $\F_{2^6}$.\\
        So $\F_{2^{12}} = \F_2 (\alpha, \beta)$ is the splitting field of $X^4 + X + 1$ over $\F_{2^6}$. 

    \part

    Claim: $X^4 + X + 1$ cannot have any root in $\F_{64}$.\\
    Assume $X^4 + X + 1$ has a root in $\F_{64}$, Then $\F_{64}$ contains all the roots
    of $X^4 + X + 1$. Thus, $\F_64$ is itself the splitting field of $X^4 + X + 1$
    over $\F_{64}$, which is a contradiction.\\

    Thus, $X^4 + X + 1$ does not have linear factors in $\F_{64}$, so factorization $3 + 1$ and $1+1+2$
    cannot happen. Also, since splitting field of $X^4 + X + 1$, $\F_{2^{12}}$, is degree 2 extension
    over $\F_{2^6}$, we deduce that the only way to factor $X^4 + X + 1$ into 
    irred. factors over $\F_{64}$ is 
    \begin{align}
        (X^2 + a_1 X + a_0) (X^2 + b_1 X + b_0)
    \end{align}
    In perticular, $X^4 + X + 1$ must be reducible over $\F_{64}$.\\
    since $\F_{2^2} \subseteq \F_{2^6}$ because $2 \mid 6$,\\
    the same factorization in \textbf{12.4.8} works here.
    \begin{align}
        X^4 + X + 1 = (X^2 + X + \alpha)(X^2 +  X + \alpha + 1)
    \end{align}
    for $\alpha^2 = \alpha + 1$, where $\alpha$ is a root of irred. polynomial 
    $X^2 + X + 1$ over $\F_2$.\\
    If we let $\F_{2^6}$ be $\F_2 (\gamma)$ where $\gamma$ is a root of 
    irred. polynomial $X^6 + X + 1$ over $\F_2$, we should be able to
    identify $\alpha$ with an element in $\F_2 (\gamma)$ of the form
    \begin{align}
        a_5 \gamma^5 + a_4 \gamma^4 + \ldots + a_1 \gamma + a_0
    \end{align}
    where $a_i = 0 \ \text{or} \ 1$, since $\F_4 \subseteq \F_{64}$.\\
    
    Since $\F_{64}^{\times}$ has $63$ elements, and $\F_2 (\alpha) = \F_4$, where $\alpha$ is root of $X^2 + X + 1$,
    we deduce $\alpha$ is a primitive $3$rd root of unity, so 
    if $\F_{64} = \F_2 (\gamma)$ where $\gamma$ is root of irreducible polynomial
    $X^6 + X + 1$, we have
    \begin{align}
        \alpha &\leftrightarrow \gamma^{21}\\
        \alpha^2 = \alpha + 1 &\leftrightarrow \gamma^{42}\\
        \alpha^3 = (\alpha + 1)\alpha = \alpha^2 + \alpha = 1 &\leftrightarrow \gamma^{63} = 1\\
        \text{thus}, \ X^4 + X + 1 &= (X^2 + X + \gamma^{21}) (X^2 + X + \gamma^{42}) \label{eqn:factor}
    \end{align}
    in $\F_2(\gamma) = \F_{64}$.\\
    We double check: RHS of equation (\ref{eqn:factor}) is
    \begin{align}
        X^4 + (\gamma^{42} + \gamma^{21} + 1)X^2 + (\gamma^{42} +  \gamma^{21})X +  1
    \end{align}
    Since $\gamma^6 = \gamma + 1$, we have
    \begin{align}
        \gamma^{21} = (\gamma^6)^3 \cdot \gamma^3 
        = (\gamma+ 1)^3 \cdot \gamma^3 
        = (\gamma^2 + 1)(\gamma+1)\gamma^3
        = \gamma + 1 + \gamma^5 + \gamma^4 + \gamma^3
    \end{align}
    Similarly, we have
    \begin{align}
        \gamma^{42} = (\gamma+1)^7 = (\gamma^2 + 1)^3 (\gamma+1)
        = (\gamma^6 + 1 + 3 \gamma^4 + 3 \gamma^2)(\gamma+1)
        =(\gamma + 3 \gamma^4 + 3 \gamma^2)(\gamma+1)
        = \gamma^5 + \gamma^4 + \gamma^3 + \gamma
    \end{align}
    Hence, we have
    \begin{align}
        \gamma^{42} + \gamma^{21} &= 1\\
        \gamma^{42} + \gamma^{21} + 1 &= 0
    \end{align}
    as wanted.
    

\end{homeworkProblem}

\pagebreak

\begin{homeworkProblem}
    \textbf{Exercise 12.4.10} 
    \begin{enumerate}
        \item Show that the polynomial $X^2 + 1$ is irreducible over $\F_7$.
        \item Consider the field $\F_7 (\alpha)$, where $\alpha$ is a root of $X^2 + 1$.
            Show that all quadratic polynomials over $\F_7$ have a root in $\F_7 (\alpha)$.
        \item Determine explicitly the roots in $\F_7(\alpha)$ of $5 X^2 + 3 X + 1 \in \F_7 [X]$.
    \end{enumerate}
    
    \solution 

    \part 
    To show $X^2+1$ is irreducible over $\F_7$, we show $X^2 + 1$ does not
    factor into linear factors which is equivalent to show $X^2 + 1$ does not have root in $\F_7$.
    \begin{align}
        X = 0 &\rightarrow X^2 + 1 = 1\\
        X = 1 &\rightarrow X^2 + 1 = 2\\
        X = 2 &\rightarrow X^2 + 1 = 5\\
        X = 3 &\rightarrow X^2 + 1 = 3 \mod 7\\
        X = 4 &\rightarrow X^2 + 1 = 3 \mod 7\\
        X = 5 &\rightarrow X^2 + 1 = 5 \mod 7\\
        X = 6 &\rightarrow X^2 + 1 = 2 \mod 7
    \end{align}
    None of them is zero, so $X^2 + 1$ does not have root in $\F_7$, 
    so $X^2 + 1$ is irreducible over $\F_7$. 

    \part
    first, assume quadratic polynomial $f$ is not irreducible. Thus,
    $f$ can be factored into $2$ linear factors, $\implies$ the roots
    of $f$ are in $\F_7$, and $\F_7$ is certainly contained in $\F_7 (\alpha)$.\\
    Now, assume $f$ is irred. Let $\beta$ be a root of $f$. Since $f$ is quadratic,
    $\F_7 (\beta)$ is degree 2 extension of $\F_7$, (which is also normal,
    since finite extensions of finite fields are Galois.) Thus,
    $\F_7(\beta)$ contains all roots of $f$. Since
    \begin{align}
        &\F_7 \subseteq \F_7(\alpha) \subseteq L\\
        &\F_7 \subseteq \F_7(\beta) \subseteq L
    \end{align}
    where $L$ is the algebraic closure of $\F_7$. Then $\F_7(\alpha)$
    and $\F_7(\beta)$ are the same field, since the
    finite field of any prime power order is unique.

    \part
    Determine roots in $\F_7 (\alpha)$ of $5 X^2 + 3 X + 1 \in \F_7 [X]$ 
    Let $a,b \in \F_7$, remember $\alpha^2 + 1 = 0$. $\F_7 (\alpha)$ is
    degree 2 vector space over $\F_7$, so a general element is of the form $a \alpha + b$.
    \begin{align}
        5(a \alpha + b)^2 + 3(a \alpha + b) + 1 &= 0\\
        5(a^2 \alpha^2 + 2 ab \alpha + b^2) + 3a \alpha + 3 b + 1 &= 0\\
        5a^2 (-1) + 10 a b \alpha + 5 b^2 + 3 a \alpha + 3b + 1 &= 0\\
        2a^2 + 3 ab \alpha + 5 b^2 + 3 a \alpha + 3 b + 1 &= 0\\
    \end{align}
    Thus,
    \begin{align}
        3 ab + 3a &= 0\\
        2 a^2 + 5 b^2 + 3b + 1 &= 0
    \end{align}
    Hence, we have case (1) 
    \begin{align}
        a &= 0\\
        5b^2 + 3b + 1 &= 0
    \end{align}
    OR case (2)
    \begin{align}
        b &= -1\\
        2a^2 + 5 -3 + 1 &=0
    \end{align}
    If case (1), then we could stop because 
    $5X^2 + 3X + 1$ has no roots in $\F_7$.
    \begin{align}
        X=1 &\rightarrow 5 X^2 + 3 X +1 =2\\
        X=2 &\rightarrow 5 X^2 + 3 X +1 =6\\
        X=3 &\rightarrow 5 X^2 + 3 X +1 =6 \mod 7 \\
        X=4 &\rightarrow 5 X^2 + 3 X +1 =2 \mod 7 \\
        X=5 &\rightarrow 5 X^2 + 3 X +1 =1 \mod 7 \\
        X=6 &\rightarrow 5 X^2 + 3 X +1 =3 \mod 7 \\
    \end{align}
   so no roots in $\F_7$.\\
   In case (2), we have
   \begin{align*}
       b &= -1\\
       2a^2 &= -3\\
       \text{which implies}& \\
       b &= 6\\
       2a^2 &= 4
   \end{align*}
   Hence, we have $b=6,\ a=3$ (since $18 \equiv 4$) or $b=6, \ a=4$ (since $32 \equiv 4 \mod 7$).\\
   Thus, the roots are $3 \alpha + 6$ and $4 \alpha + 6$.\\
   We could double check
   \begin{align}
       5(X-6-3 \alpha)(X - 6 - 4 \alpha) &= 5 (X^2 -12X + 36 + 12 \alpha^2 -7 \alpha (X-6))\\
       &= 5(X^2 + 2 X + 1 -5)\\
       &= 5 X^2 + 3X + 1 \mod 7.
   \end{align}
   

   
    
    
    
    



\end{homeworkProblem}

\end{document}
